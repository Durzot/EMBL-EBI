\documentclass[a4paper, 11pt]{article}

\usepackage[left=1.5cm, right=1.5cm, top=2cm, bottom=2cm]{geometry}

\usepackage[utf8]{inputenc}
\usepackage[T1]{fontenc}
\usepackage[english]{babel}  
\usepackage{lmodern}

\usepackage{amsmath, amsthm, amssymb}
\usepackage{mathtools}

\newtheorem{innercustomgeneric}{\customgenericname}
\providecommand{\customgenericname}{}
\newcommand{\newcustomtheorem}[2]{%
  \newenvironment{#1}[1]
  {%
   \renewcommand\customgenericname{#2}%
   \renewcommand\theinnercustomgeneric{##1}%
   \innercustomgeneric}
  {\endinnercustomgeneric}
}

\newcustomtheorem{thm}{Theorem}
\newcustomtheorem{lem}{Lemma}
\newcustomtheorem{cor}{Corollary}
\newcustomtheorem{deftn}{Definition}
\newcustomtheorem{prop}{Proposition}
\DeclareMathOperator*{\argmin}{argmin} 
\DeclareMathOperator*{\argmax}{argmax} 

\begin{document}
\title{Summary paper 3: Ellipse-preserving Hermite interpolation and subdivision}
\author{Yoann Pradat}
\maketitle

\noindent\underline{Active contour (snake)} = computational tool for detecting and outlining objects in digital 
images.\\ 

The study in the paper is motivated by the observation that usual active contours as cubic Hermite splines are unable to 
reproduce elementary shapes. The latter have a lot of attractive properties including fourth-order approximation, 
continuously differentiable, multiresolution properties and minimal support. However a lot of control points are needed 
to approximate ellipses. In this paper authors devised a new Hermite subdivision scheme that can perfectly reproduce 
ellipses while retaining attractive properties of cubic Hermite splines. 

\paragraph{Motivation} \mbox{} \\

[11] Delgado et al.\ proposed a first solution based on cardinal exponential B-splines. The latter work is refined to 
include control over the tangents. The parametric representation of the curve $r(t) = (x(t), y(t))$ is 

\begin{equation}
  \forall t \quad r(t) = \sum_{n \in \mathbb{Z}} r(n) \phi_1(t-n) + r'(n) \phi_2(t-n)
\end{equation}

where $r(t)$ and its derivative $r'(t)$ are assumed to be M-periodic with M being the number of control points. \\

Of course in order for this to be an Hermite interpolation $\phi_1$ and $\phi_2$ must satisfy Hermite intepolation 
properties. We refer to \textbf{shape space} as the collection of all curves that can be generated by the parametric 
representation above when varying ${\left(r(n), r'(n)\right)}_{n=0}^{M-1}$. 3 fundamental requirements on the 
specification of the space shape

\begin{enumerate}
  \item Representation should be unambiguous and stable w.r.t variation of shape parameters (Riesz-basis property)
  \item Shape space should be invariant to affine transformation (partition of unity of $\phi_1$)
  \item Shape space should include ellipses
\end{enumerate}

The latter is satisfied if, for $w_0 = \frac{2\pi}{N} \in [0, \pi]$, one can write for all $t$

\begin{align}
  \cos (w_0t) &= \sum_{n \in \mathbb{Z}} \cos (w_0n) \phi_1(t-n) - w_0 \sin (w_0n) \phi_2(t-n) \\
  \sin (w_0t) &= \sum_{n \in \mathbb{Z}} \sin (w_0n) \phi_1(t-n) + w_0 \cos (w_0n) \phi_2(t-n)
\end{align}

\paragraph{Cardinal Hermite Cycloidal splines} \mbox{} \\

Determine $\phi_{1, w_0}, \phi_{2, w_0}$ by first focusing on the interval $[0, 1]$. Given the boundary constraints, 
\begin{equation}
  \phi_{1,w_0}, \phi_{2, w_0} \in \mathcal{E}_4 = <1, x, e^{-iw_0x}, e^{iw_0x}>
\end{equation}

Given the reproduction requirements, the basis generators are given by
\begin{equation}
  \phi_{1, w_0}(x) =
  \begin{dcases}
    g_{1, w_0}(x) &\text{for } x \geq 0 \\
    g_{1, w_0}(-x) &\text{for } x < 0\\
  \end{dcases} \quad
  \hfill
  \phi_{2, w_0}(x) =
  \begin{dcases}
    g_{2, w_0}(x) &\text{for } x \geq 0 \\
    -g_{2, w_0}(-x) &\text{for } x < 0\\
  \end{dcases} \quad
\end{equation}

Clearly, any combination of integers shifts of the generators is piecewise polynomial function continously 
differentiable at joining points. It can also be seen as exponential splines with double knots at the integers. The 
space of cardinal Hermite cycloidal splines is given by

\begin{equation}
  S^1_{\mathcal{E}_4}(\mathbb{Z}) = \left\{ s(x) = \sum_{n \in \mathbb{Z}} {a[n]}^T \Phi_{w_0}(x-n) \ | \ a \in 
  l^{2\times1}_2(\mathbb{Z}) \right\}
\end{equation}

\paragraph{Connection with standard exponential splines} \mbox{} \\

Looking at the Fourier transform of our basis functions, they can be linked to Green's functions of operators $L_1 = D^4 
+ w_0^2 D^2$ and $L_2 = D^3 + w_0^2 D$ as follows

\begin{equation}
  \forall w \quad \hat{\Phi}_{w_0}(w) = \hat{R}(e^{iw}) \hat{\rho}_{w_0}(w)
\end{equation}

Let $\hat{P}$ the inverse of $\hat{R}$ then we have 

\begin{equation}
  \forall x \quad \rho_{w_0}(x) = \sum_{n \in \mathbb{Z}} P[n] \Phi_{w_0}(x-n)
\end{equation}

Note that (to be cleanly proved using tempered distribution)
\begin{align*}
  \rho_{1,w_0}(x)&=\mathcal{F}^{-1} \left(\frac{1}{w^2(w^2-w_0^2)}\right) = \frac{w_0x - \sin (w_0x)}{2 w_0^3} sgn(x) \\
  \rho_{2,w_0}(x)&=\mathcal{F}^{-1} \left(\frac{i}{w(w^2-w_0^2)}\right) = \frac{1 - \cos (w_0x)}{2 w_0^2} sgn(x)
\end{align*}

\begin{deftn}{1}
  For $w_j \in [0, \pi], j=0, \ldots, m$ the discrete annihilation operator for frequencies $(w_0, \ldots, w_m)$ is 
  defined recursively as 
  
  \begin{equation}
    \Delta_{w_0} f(x) = f(x) - e^{iw_0}f(x-1) \qquad \Delta_{(w_0, \ldots, w_m)} = \Delta_{w_0} \Delta_{(w_1, \ldots, 
    w_m)}
  \end{equation}
\end{deftn}

Normalized order four and order three exponential B-spline, basis for $S_{\mathcal{E}_4}(\mathbb{Z})$ and 
$S_{\mathcal{E}_3}(\mathbb{Z})$ respectively, are in the span of integer shifts of basis generators $\phi_{1, w_0}, 
\phi_{2, w_0}$. As a consequence all functions of $\mathcal{E}_4$ are in $S^{1}_{\mathcal{E}_4}(\mathbb{Z})$.

\begin{prop}{1}
  The exponential spline space $S^{1}_{\mathcal{E}_4}(\mathbb{Z})$ can be written $S^{1}_{\mathcal{E}_4}(\mathbb{Z}) = 
  S_{\mathcal{E}_3}(\mathbb{Z}) + S_{\mathcal{E}_4}(\mathbb{Z})$.
\end{prop}

This is proved by observating that any function in $S_{\mathcal{E}_4}(\mathbb{Z})$ admits a unique representation in the 
span of integers shifts of $\rho_{1, w_0}$, same holds for $S_{\mathcal{E}_3}(\mathbb{Z})$  and $\rho_{2, w_0}$. 

\paragraph{Riesz-basis property} \mbox{} \\

\begin{thm}{1}
  The system of ${\left\{\Phi_{w_0}(.-n)\right\}}_{n \in \mathbb{Z}}$ forms a Riesz-basis.
\end{thm}

This is proved using the Hermitian Fourier Gram matrix of the basis. 

\paragraph{Re-scaled Hermite representation} \mbox{} \\

We define the Hermite functions on the grid $h\mathbb{Z}$ with $h > 0 $ as 

\begin{align*}
  \phi^h_{1,w_0}(x) &= \phi_{1, hw_0}(x/h) \\
  \phi^h_{2,w_0}(x) &= h\phi_{2, hw_0}(x/h)
\end{align*}

Note that these refined basis functions still satisfy the fundamental Hermite interpolation properties on the grid 
$h\mathbb{Z}$. Now is an important result that identify the asymptotic properties of cycloidal Hermite splines with that 
of cubic Hermite spline as $h \to 0$

\begin{prop}{2}
  The rescaled Hermite functions are such that
  \begin{align*}
    \lim_{h \to 0} \phi^h_{1, w_0}(hx) &=
      \begin{dcases}
        (-2x+1){(x+1)}^2 &\text{for } -1 \leq  x \leq 0 \\
        (2x+1){(x-1)}^2 &\text{for } 0 < x \leq 1\\
      \end{dcases} \\
    \lim_{h \to 0} \frac{1}{h} \phi^h_{2, w_0}(hx) &=
      \begin{dcases}
        x{(x+1)}^2 &\text{for } -1 \leq  x \leq 0 \\
        x{(x-1)}^2 &\text{for } 0 < x \leq 1\\
      \end{dcases} 
    \end{align*}
\end{prop}

The convolution relation between basis functions and Green's functions is also valid on $h\mathbb{Z}$ as follows
\begin{equation}
  \Phi^h_{w_0}(x) = \sum_{n \in \mathbb{Z}} R_h[n] \rho_{w_0}(x-nh)
\end{equation}

Then any function $s_h$ in $S^1_{\mathcal{E}_4}(h\mathbb{Z})$ given by $s_h = \sum_{n \in \mathbb{Z}} {a_h[n]}^T
\Phi^h_{w_0}(.-nh)$ can also be written as

\begin{equation}
  s_h = \sum_{n \in \mathbb{Z}} {b_h[n]}^T \rho_{w_0}(.-nh)
\end{equation}

with ${b_h[n]}^T = (a_h^T * R_h)[n]$. Note now that basis functions in the representation above are independent from 
$h$! Therefore for any $m > 1$, $S^1_{\mathcal{E}_4}(h\mathbb{Z}) \subseteq S^1_{\mathcal{E}_4}(\frac{h}{m}\mathbb{Z})$.  
\\

Writing the function $\Phi^h_{w_0}$ in the representation at scale $\frac{h}{m} \mathbb{Z}$ one has

\begin{equation}
  \Phi^h_{w_0} = \sum_{n \in \mathbb{Z}} H_{h \to \frac{h}{m}}[n] \Phi^{\frac{h}{m}}_{w_0}(.-n\frac{h}{m})
\end{equation}

with $H_{h \to \frac{h}{m}}[n] = \begin{pmatrix}
  \phi^h_{1, w_0}(\frac{n}{m}) & {\phi^{h}_{1, w_0}}'(\frac{n}{m})  \\
  \phi^h_{2, w_0}(\frac{n}{m}) & {\phi^{h}_{2, w_0}}'(\frac{n}{m})
\end{pmatrix}$

In the end the author gives explicit expression for the four Bersntein basis functions for $\mathcal{E}_4$ satisfying 
symmetry, endpoint conditions, partition of unity and non-negativity. The conversion between two types of representation 
is given by

\begin{equation}
  \begin{pmatrix} \phi_{1, w_0}(t) \\  \phi_{2, w_0}(t) \\ \phi_{1, w_0}(t-1) \\ \phi_{2, w_0}(t-1) \end{pmatrix}
  =
  \begin{pmatrix}
  1 & 1 & 0 & 0 \\
  0 & \frac{r(w_0)}{r(w_0)-p(w_0)} & 0 & 0 \\
  0 & 0 & 1 & 1 \\
  0 & 0 & - \frac{r(w_0)}{r(w_0)-p(w_0)} & 0 \\
  \end{pmatrix}
  \begin{pmatrix} b_{0, w_0}(t) \\ b_{1, w_0}(t) \\ b_{2, w_0}(t)  \\ b_{3, w_0}(t) \end{pmatrix}
\end{equation}

\end{document}




