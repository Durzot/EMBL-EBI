\documentclass[a4paper, 11pt]{article}

\usepackage[left=1.5cm, right=1.5cm, top=2cm, bottom=2cm]{geometry}

\usepackage[utf8]{inputenc}
\usepackage[T1]{fontenc}
\usepackage[english]{babel}  
\usepackage{lmodern}

\usepackage{amsmath, amsthm, amssymb}
\usepackage{mathtools}

\newtheorem{innercustomgeneric}{\customgenericname}
\providecommand{\customgenericname}{}
\newcommand{\newcustomtheorem}[2]{%
  \newenvironment{#1}[1]
  {%
   \renewcommand\customgenericname{#2}%
   \renewcommand\theinnercustomgeneric{##1}%
   \innercustomgeneric}
  {\endinnercustomgeneric}
}

\newcustomtheorem{thm}{Theorem}
\newcustomtheorem{lem}{Lemma}
\newcustomtheorem{cor}{Corollary}
\newcustomtheorem{deftn}{Definition}
\newcustomtheorem{prop}{Proposition}
\DeclareMathOperator*{\argmin}{argmin} 
\DeclareMathOperator*{\argmax}{argmax}
\DeclareMathOperator*{\Tr}{Tr}

\begin{document}
\title{Summary paper 7: Spline-based deforming ellipsoids for 3D bioimage segmentation Delgado al}
\author{Yoann Pradat}
\maketitle

In this paper the authors derive an exponential B-splines-based model that allow to reproduce ellipsoids. The model 
proposed is computationally efficient because basis functions are of the shortest possible support and volume integrals 
are advantageously transformed into surface integrals using Gauss's theorem. The model can approximate very well blobs 
and perfectly spheres and ellipsoids. A plugin named “Active cells 3D” is provided for the software Icy. \\

Other references in this subject include [2] (simplex meshes), [4] (fully parametric snakes AGF), [5] (refinement of AGF 
using polynomial B-splines), [6] spline-based parametric model with sphere topology. \\

The parametric snake model in the article have the following important features: perfectly reproduce ellipsoids, 
shortest possible support, refinable. 

\paragraph{Splines surfaces for object segmentation} \mbox{} \\

The work here is an extension to 3D of 2D snakes presented in [7] that also have ellipse-reproducing properties. A 
surface is represented by two continuous parameters as such

\begin{equation}
  \sigma(u, v) = \sum_{(i,j) \in \mathbb{Z}^2} c[i,j] \Phi(\frac{u}{T_1}-i, \frac{v}{T_2}-j)
\end{equation}

where $T_1, T_2$ are sampling steps for each parametric dimension. A common strategy is to use tensor products for the 
construction of $\Phi$ that is $\Phi(u, v) = \phi_1(u)\phi_2(v)$. Various choices of $\phi_1, \phi_2$ have been 
considered such as polynomials, polynomial B-splines and trigonometric B-splines. This tensor decomposition allow for 
fast and stable interpolation algorithms. Limits to what surfaces can represented by such as $\Phi$? \\ 

At point ${\bf p} = \sigma(u_0, v_0)$ we define the tangent plane as space spanned by the vectors 

\begin{align}
  {\bf T_1} &= \frac{\partial \sigma}{\partial u}(u_0, v_0) \\
  {\bf T_2} &= \frac{\partial \sigma}{\partial v}(u_0, v_0)
\end{align}

The tangent bundle is the union of all tangent spaces at points on the surface $S$. It is said to well-defined if all 
tangent spaces have dimension 2. In that case $S$ is regular i.e it is not self-intersecting nor does it have borders.  
Under these conditions, the normal vector to $S$ at ${\bf p}$ is ${\bf n} = {\bf T_1} \times {\bf T_2}$ where the 
product is the cross-product. \\

There are three properties we would like our basis functions to have that are: uniqueness of representation and 
numerical stability, affine invariance and well-defined gaussian curvature. A sufficient condition for the first 
property is to have $\phi_1, \phi_2$ to be Riesz generators. The second property is ensured if $\Phi$ has the partition 
of unity property which holds iif $\phi_1, \phi_2$ themselves have it. The gaussian curvature is at point ${\bf p} = 
\sigma(u_0, v_0)$ is $K = \frac{\det II}{\det I}$ where

\begin{align}
  I &= \begin{bmatrix} {\bf T_1.T_1} & {\bf T_1.T_2} \\ {\bf T_1.T_2} & {\bf T_2.T_2} \\ \end{bmatrix} \\
  II &= \begin{bmatrix} \frac{\partial^2 \sigma}{\partial u^2}.{\bf \hat{n}} & \frac{\partial^2 \sigma}{\partial 
  u\partial v}.{\bf \hat{n}} \\ \frac{\partial^2 \sigma}{\partial u \partial v}.{\bf \hat{n}} & \frac{\partial^2 
\sigma}{\partial v^2}.{\bf \hat{n}} \end{bmatrix}
\end{align}

The 3D parametric model represents closed parametric surfaces $\sigma$ and thus it is enough to consider a compact set 
$\Omega \subseteq \mathbb{R}^2$ for the continuous parameters. The range of continuous parameters are normalized so that 
$\Omega = {[0,1]}^2$ that is 

\begin{equation}
  \forall (u, v) \in {[0,1]}^2 \quad \sigma(u,v) = \sum_{(i,j) \in \mathbb{Z}^2} c[i,j] \phi_1(M_1u-i)\phi_2(M_2v-j)
\end{equation}

With the appropriate choice of $\phi_1, \phi_2$ we can force the surface to take the topology of an ellipsoid with 
appropriate boundary conditions on ${(c[i,j])}_{(i,j) \in \mathbb{Z}^2}$. We choose to adopt earth-like terminology and 
refer to $u$ as latitude parameter while referring to $v$ as meridian parameter. For fixed $v$, $\sigma_{v}(.)$ is a 
closed circle.  This is achieved by letting functions of each component of $\sigma$  in $u$ to be 1-periodic for $v$ 
constant. Thus it is necessary to have the sequence of coefficients to be $M_1$-periodic in the first variable. Then

\begin{equation}
  \forall (u, v) \in {[0,1]}^2 \quad \sigma(u,v) = \sum_{i=0}^{M_1-1} \sum_{j \in \mathbb{Z}} c[i,j] \phi_{1, per} 
  (M_1u-i)\phi_2(M_2v-j)
\end{equation}

Note that curves for $u$ constant $\sigma_{u}(.)$ are open curves from north pole ${\bf c_N}$ to south pole ${\bf c_S}$.  
\\

The classical parametrization of the unit sphere is

\begin{equation*}
  \forall (u, v) \in {[0,1]}^2 \quad \sigma(u, v) = 
  \begin{bmatrix} \cos (2\pi u) \sin(\pi v) \\ \sin (2\pi u) \sin(\pi v) \\ \cos (\pi v) \end{bmatrix} 
\end{equation*}

For $\Phi$ to satisfy previous properties and reproduce ellipsoids, $\phi_1$ must reproduce constants and sinusoids of 
period 1 while $\phi_2$ must reproduce constants and sinusoids of period 2. The optimal choice of $\phi_1, \phi_2$ is 
dictated by spline theory about reproduction of exponential polynomials [17]

\begin{thm}{1}
  The centered generating function with minimal support, maximum smoothness satisfying Riesz-basis property, partition 
  of unity property and that reproduces sinusoids of unit period with $M$ coefficients is
  \begin{equation}
    \varphi_{M}(.) = \sum_{k=0}^3 {(-1)}^k h_{M}[k] \varsigma_M(. + \frac{3}{2} - k)
  \end{equation}
  where $\varsigma_M(.) = \frac{1}{4} sgn(.) \frac{\sin^2(\frac{\pi}{M}.)}{\sin^2(\frac{\pi}{M})}$ and $h_M = [1, 
1+2\cos(\frac{2\pi}{M}), 1+2\cos(\frac{2\pi}{M}), 1]$ 
\end{thm}

Therefore we take $\phi_1 = \varphi_{M_1}, \phi_2 = \varphi_{2M_2}$. Note that these functions have support of size 3 
and that they converge to quadratic B-spline as $M$ goes to infinity. The control points that make the snake take the 
shape of a perfect unit square are given by

\begin{equation}
  c[i, j]  = \begin{bmatrix} c_{M_1}[i]s_{2M_2}[j] \\ s_{M_1}[i]s_{2M_2}[j] \\ c_{2M_2}[j] \end{bmatrix}
\end{equation}

Note that this is not unique as we arbitrarily choose the origin of the parametrization. The chosen basis functions are 
twice differentiable with bounded second derivative. However the model has two singular points at the poles where 
continuity of the representation and of its derivative is not guaranteed. In order to be well-defined we require 
$\sigma(u, 0), \sigma(u, 1)$ to be independent of $u$. A sufficient condition for continuity of the tangent plane was 
proven in [8] to be 

\begin{align}
  \frac{\partial \sigma}{\partial u}(u, 0) &= {\bf T_{1, N}} \cos (2\pi u) + {\bf T_{2, N}} \sin (2\pi u) \\
  \frac{\partial \sigma}{\partial u}(u, 1) &= {\bf T_{1, S}} \cos (2\pi u) + {\bf T_{2, S}} \sin (2\pi u) \\
\end{align}

Tensor-product polynomial splines on the sphere have already been considered but they cannot reproduce perfectly 
ellipsoids. No attempt was made to deal with the pole problem and take advantage of B-splines. Interpolation conditions 
at the poles translate into 

\begin{align*}
  \forall i=0, \ldots, M-1 \quad {\bf c_N} &= \phi_2(-1) c[i, 1] + \phi_2(0) c[i, 0] + \phi_2(1) c[i, -1] \\
  \forall i=0, \ldots, M-1 \quad {\bf c_S} &= \phi_2(-1) c[i, M_2+1] + \phi_2(0) c[i, M_2] + \phi_2(1) c[i, M_2-1]
\end{align*}

This is proved from Riesz-basis property and partition of unity of $\phi_1$. Sufficient conditions for continuity of the 
tangent plane translate into

\begin{align*}
   {\bf T_{1, N}} \cos (2\pi u) + {\bf T_{2, N}} \sin (2\pi u) &= M_2\sum_{i=0}^{M_1-1} \sum_{j \in \mathbb{Z}} c[i,j] 
   \phi_{1, per} (M_1u-i)\phi_2(-j)' \\
   {\bf T_{1, S}} \cos (2\pi u) + {\bf T_{2, S}} \sin (2\pi u) &= M_2\sum_{i=0}^{M_1-1} \sum_{j \in \mathbb{Z}} c[i,j] 
   \phi_{1, per} (M_1u-i)\phi_2(M_2-j)' 
 \end{align*}

Given these conditions we can now explicitly express our 3D snake model as

\begin{thm}{2}
  A parametric splines-bases surface with a sphere-like topology, $C^1$ continuity and ellipsoid-reproducing 
  capabilities (all positions and orientations) is given by
  \begin{equation}
    \forall (u, v) \in {[0,1]}^2 \quad \sigma(u,v) = \sum_{i=0}^{M_1-1} \sum_{j=-1}^{M_2+1} c[i,j] \phi_{1, per} 
    (M_1u-i)\phi_2(M_2v-j)
  \end{equation}
  
  where ${\{c[i,j]\}}_{i \in [0, \ldots, M_1-1], j \in [1, \ldots, M_2-1]}, {\bf T_{1, N}}, {\bf T_{2, N}}, {\bf T_{1, 
  S}} , {\bf T_{1, S}}, {\bf c_{N}}, {\bf c_{S}}$ are free parameters that is $M_1(M_2-1) + 6$ control points. $c[i,-1], 
  c[i, 0], c[i, M_2], c[i, M_2+1]$ are constrained by the values of the free parameters.
\end{thm}

\paragraph{Energies and implementation} \mbox{} \\

The rest of the article then discusses snake energies to be minimized in order to detect and outline objects in the 
image of interest. The global energy is the sum of three energies: image energy (edge or region-based), internal energy 
(ensures smoothness), constrained energy (in practice hard constraints on parameters). 

\end{document}




