\documentclass[a4paper, 11pt]{article}

\usepackage[left=1.5cm, right=1.5cm, top=2cm, bottom=2cm]{geometry}

\usepackage[utf8]{inputenc}
\usepackage[T1]{fontenc}
\usepackage[english]{babel}  
\usepackage{lmodern}

\usepackage{amsmath, amsthm, amssymb}
\usepackage{mathtools}

\newtheorem{innercustomgeneric}{\customgenericname}
\providecommand{\customgenericname}{}
\newcommand{\newcustomtheorem}[2]{%
  \newenvironment{#1}[1]
  {%
   \renewcommand\customgenericname{#2}%
   \renewcommand\theinnercustomgeneric{##1}%
   \innercustomgeneric}
  {\endinnercustomgeneric}
}

\newcustomtheorem{thm}{Theorem}
\newcustomtheorem{lem}{Lemma}
\newcustomtheorem{cor}{Corollary}
\newcustomtheorem{deftn}{Definition}
\newcustomtheorem{prop}{Proposition}
\DeclareMathOperator*{\argmin}{argmin} 
\DeclareMathOperator*{\argmax}{argmax}
\DeclareMathOperator*{\Tr}{Tr}

\begin{document}
\title{Summary paper 8: Compactly-supported smooth interpolators for shape modeling with varying resolution Schmitter et 
al}
\author{Yoann Pradat}
\maketitle

In all applications of shape-modeling there is a need for intuitive user interaction which is best achieved when the 
representation is interpolatory. Domains related to interactive curves or shape modeling include computer graphics, 
biomedical imaging, industrial design, modeling of animated surfaces etc. Techniques can be categorized between discrete 
and continuous methods. Discrete methods include polygonal meshes and subdivision scheme and they allow to locally 
refine a shape.  The most used continuous method is the NURBS model but it generally cannot be smooth and interpolatory 
at the same time. \\ 

In the paper authors propose a 3D shape modeling in continuous domain that also has the advantages of discrete domain 
representation that is to say refinement capabilities and interpolation properties. The interpolatory functions are 
written as linear combinations of shifts on the half integer grid of exponential B-splines. This representation allow to 
transfer good properties of B-splines to the basis functions used for modeling. 

\paragraph{Review of exponential B-splines} \mbox{} \\
 
To a vector ${\bf \alpha} \in \mathbb{R}^n$ we can associate a causal B-spline of order $n$ $\beta^+_{\alpha}$. Indeed
if $r^n + a_{n-1}r^{n-1} + \ldots, a_{0}$ is the polynomial of unit leading coefficient whose roots are $\alpha$ then 
the differential operator

\begin{equation}
  L = D^n + a_{n-1}D^{n-1} + \ldots, a_{0}
\end{equation}

can factorize into 

\begin{equation}
  L = \prod_{i=1}^n (D-\alpha_i I)
\end{equation}

The null space of that LSI operator is $\mathcal{N}_{\alpha} = \text{span}{\{t^n e^{\alpha_{m}t}\}}_{m=1, \ldots, n_d ; 
n=1, \ldots, n_{(m)}}$ where $n_d$ is the number of unique components in ${\bf \alpha}$. The Green function of $L$ is 
defined as the unique causal function $\rho_{\alpha}$ that satisfies $\delta = L\{\rho\}$. To each elementary operator 
$D - \alpha_i I$ we associate the Green's function $\rho_{\alpha_i}$ and we define the order 1 exponential B-spline as 

\begin{equation}
  \beta^+_{\alpha_i}(t) = \rho_{\alpha_i}(t) - e^{\alpha_i}\rho_{\alpha_i}(t-1)
\end{equation}

and the order n exponential B-spline as

\begin{equation}
  \beta^+_{\alpha}(t) = (\beta^+_{\alpha_1} * \ldots * \beta^+_{\alpha_n})(t)
\end{equation}

As a consequence $\displaystyle \hat{\beta}^+_{\alpha}(w) = \prod_{i=1}^n \hat{\beta}^+_{\alpha_i}(w)$. From 
$\hat{\rho}_{\alpha_i}(w) = \frac{1}{jw-\alpha_i}$ and the equation defining the order exponential B-spline we deduce

\begin{equation}
  \hat{\beta}^+_{\alpha}(w) = \prod_{i=1}^n \frac{1-e^{\alpha_i-jw} }{jw-\alpha_i}
\end{equation}

The function $\beta^+_{\alpha}$ is compactly supported on $[0, n]$. In interpolation schemes we prefer having basis 
functions symmetric around 0 so let's define $\beta_{\alpha}(t) = \beta^+_{\alpha}(t+n/2)$. This order n exponential 
B-spline can reproduce functions in the nullspace $\mathcal{N}_{\alpha}$. 

\paragraph{Characterization of the interpolator}

\begin{deftn}{1}
  For a sequence $\lambda \in l_1(\mathbb{Z})$ and a vector ${\bf \alpha}$ we define the interpolatory function
  \begin{equation}
    \phi_{\lambda, \alpha}(t) = \sum_{k \in \mathbb{Z}} \lambda[k] \beta_{\alpha}(t-k/2)
  \end{equation}
\end{deftn}

The resulting scheme is then, for suitable functions $f$, given by $f(t) = \sum_{k \in \mathbb{Z}}f[k] \phi_{\lambda, 
\alpha}(t-k)$.  The desired properties for the basis function are the following

\begin{enumerate}
  \item ${\{ \phi_{\lambda, \alpha}(.-k) \}}_{k \in \mathbb{Z}}$ is interpolatory
  \item $\phi_{\lambda, \alpha}$ is compactly supported i.e $\lambda$ has finite number of non-zero values
  \item $\phi_{\lambda, \alpha}$ is smooth i.e at least $\mathcal{C}^1$
  \item ${\{ \phi_{\lambda, \alpha}(.-k) \}}_{k \in \mathbb{Z}}$ is a Riesz basis
  \item Can reproduce the nullspace $\mathcal{N}_{\alpha}$
  \item Can reproduce shapes at various resolutions
\end{enumerate}

\begin{prop}{1}
  Let ${\bf \alpha}$ a vector be such that $\alpha_m-\alpha_n \not\in 2j\pi\mathbb{Z}$ for any pair of purely imaginary 
numbers $(\alpha_m, \alpha_n)$ in ${\bf \alpha}$. For any sequence $\lambda \in l_1(\mathbb{Z})$ if $\phi_{\lambda, 
\alpha}$ is interpolatory then ${\{ \phi_{\lambda, \alpha}(.-k) \}}_{k \in \mathbb{Z}}$ is a Riesz basis.  \end{prop}

\begin{prop}{2}
  Let ${\bf \alpha}$ be a vector of roots and suppose $\lambda \in l_1(\mathbb{Z})$ satisfies
  \begin{align*}
    &\sum_{k \in \mathbb{Z}} |\lambda[k]| e^{-\alpha_i \frac{k}{2}} < \infty \\
    &\sum_{k \in \mathbb{Z}} \lambda[k] e^{-\alpha_i \frac{k}{2}} = 0
  \end{align*}
  for all $i=1, \ldots n$ then the basis function $\phi_{\lambda, \alpha}$ has the same reproduction properties as 
  $\beta_{\alpha}$.
\end{prop}

The dilation by an integer $m \in \mathbb{N}^*$ of an exponential B-spline is expressed as
\begin{equation*}
  \beta^+_{\alpha}(\frac{t}{m}) = \sum_{k \in \mathbb{Z}} h_{\frac{\alpha}{m}, m}[k] \beta^+_{\frac{\alpha}{m}}(t-k)
\end{equation*}

\begin{prop}{3}
  Let ${\bf \alpha}$ be a vector of roots, $\lambda \in l_1(\mathbb{Z})$ and $m_0$ an even integer. Then we have
  \begin{equation}
    \phi^+_{\lambda, \alpha}(\frac{t}{m_0}) = \sum_{k \in \mathbb{Z}} g_{\lambda, \frac{\alpha}{m_0}, m_0}[k] 
    \beta^+_{\frac{\alpha}{m_0}}(t-k)
  \end{equation}
\end{prop}

\begin{prop}{4}
  Let ${\bf \alpha}$ be a vector of roots, $\lambda \in l_1(\mathbb{Z})$ and $m_0$ an even integer, $m$ an integer. For 
  a continuous function $f$ with samples $\{c[k] = f(k) \ k \in \mathbb{Z}\}$ we consider the iterative scheme specified 
  by
  \begin{enumerate}
    \item pre-filter step: $c_0[k] = \left(g_{\lambda, \frac{\alpha}{m_0}, m_0}*c_{\uparrow m_0}\right)[k]$
    \item iterative steps: $c_n[k] = \left(g_{\lambda, \frac{\alpha}{m_0m^n}, m}*{(c_{n-1})}_{\uparrow m}\right)[k]$
  \end{enumerate}

  is convergent in the sense that $\displaystyle \lim_{n \to \infty} \sum_{k \in \mathbb{Z}} c_n[k]\delta(m_0 m^n t-k) = 
  f(t)$
\end{prop}

\paragraph{Construction of family of compactly supported interpolators in practice} \mbox{} \\

It is known that there exists no exponential B-spline that $\beta_{\alpha}$ that is interpolatory and smooth. The goal 
here is to construct a compactly supported generator that has the same properties as $\beta_{\alpha}$ while also being 
intepolatory. For it to be real-valued and symmetric, elements of $\alpha$ must be 0 or come in complex conjugate pairs.  
\\

We assume additionally that condition of proposition 1 are satisfied. For such an $\alpha$ the interpolatory function 
is of the form
\begin{equation}
  \phi_{\lambda, \alpha}(t) = \lambda[0]\beta_{\alpha}(t) + \sum_{i=1}^{n-2} \lambda[n](\beta_{\alpha}(t-n/2) + 
\beta_{\alpha}(t+n/2)) \end{equation}

whose support is included in $[-(n-1), (n-1)]$. This function is interpolatory if and only if $\phi_{\lambda, 
\alpha}(0)=1$ and $\phi_{\lambda, \alpha}(1)=\cdots=\phi_{\lambda, \alpha}(n-2)=0$. This defines a system of $n-1$ 
equations with $n-1$ unknowns. The system has a solution if the matrix defined by $k,l = 0, \ldots, n-2$

\begin{equation}
  {[A_{\alpha}]}_{k+1, l+1} = \begin{dcases}
    \beta_{\alpha}(k) &\text{if } l=0 \\
    \beta_{\alpha}(k-\frac{l}{2}) + \beta_{\alpha}(k+\frac{l}{2})  &\text{else }      
  \end{dcases}
\end{equation}

is invertible. In that case $\lambda = A_{\alpha}^{-1}(1, 0, \ldots, 0)$.

\begin{deftn}{3}
  Let $\alpha$ be a vector of roots whose elements are 0 or come in complex conjugate pairs with no pair of imaginary 
  roots separated by a multiple of $2j\pi$. In case the matrix $A_{\alpha}$ is invertible define $\phi_{\alpha} = 
  \phi_{\lambda, \alpha}$.  
\end{deftn}

A parametric curve ${\bf r(t)} = {[r_x(t), r_y(t), r_z(t)]}^T$ can then be readily expressed as

\begin{equation}
  {\bf r(t)} = \sum_{k \in \mathbb{Z}} r[k] \phi_{\alpha}(t-k)
\end{equation}

This can be extended to represent tensor-product surfaces that

\begin{equation}
  {\bf \sigma(u, v)} = \sum_{k \in \mathbb{Z}}\sum_{l \in \mathbb{Z}} {\bf \sigma[k,l]} \phi_{\alpha_1}(u-k) 
  \phi_{\alpha_2}(v-l)
\end{equation}


\end{document}




