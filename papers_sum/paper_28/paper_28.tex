\documentclass[a4paper, 11pt]{article}

\usepackage[left=1.5cm, right=1.5cm, top=2cm, bottom=2cm]{geometry}

\usepackage[utf8]{inputenc}
\usepackage[T1]{fontenc}
\usepackage[english]{babel}  
\usepackage{lmodern}

\usepackage{amsmath, amsthm, amssymb}
\usepackage{mathtools}
\usepackage{dsfont}

\newtheorem{innercustomgeneric}{\customgenericname}
\providecommand{\customgenericname}{}
\newcommand{\newcustomtheorem}[2]{%
  \newenvironment{#1}[1]
  {%
   \renewcommand\customgenericname{#2}%
   \renewcommand\theinnercustomgeneric{##1}%
   \innercustomgeneric}
  {\endinnercustomgeneric}
}

\newcustomtheorem{thm}{Theorem}
\newcustomtheorem{lem}{Lemma}
\newcustomtheorem{cor}{Corollary}
\newcustomtheorem{deftn}{Definition}
\newcustomtheorem{prop}{Proposition}
\newcustomtheorem{remark}{Remark}
\DeclareMathOperator*{\argmin}{argmin} 
\DeclareMathOperator*{\argmax}{argmax}
\DeclareMathOperator*{\Tr}{Tr}
\DeclareMathOperator*{\sinc}{sinc} 
\DeclareMathOperator*{\supp}{supp} 
\DeclareMathOperator*{\essinf}{ess\ inf}
\DeclareMathOperator*{\esssup}{ess\ sup} 

\begin{document}
\title{Summary paper 28: Riesz bases of splines and regularized splines with multiple knots}
\author{Yoann Pradat}
\maketitle

This paper is interested in characterizing $L_2$-norm and Sobolev norm stability of polynomial splines with multiple 
knots, that is to say osculatory interpolation. Stability estimates are reformulated in terms of Riesz basis property.  
Authors said their work was motivated by talk paper of C. de Boor, \emph{The quasi-interpolant as a tool in elementary 
polynomial spline theory} that seemingly has the same interrogation as I have for a lower bound on the Gram matrix. In 
their previous paper published in 1995, \emph{A generalization of de Boor's stability result and symmetric
preconditioning}, authors developed a generalization for regularized splines with simples knots by tempered 
distributions. \\

This paper achieves three different results
\begin{enumerate}
  \item Extends $L_2$ stability estimates to Sobolev norms
  \item Extends stability estimates to regularized splines based on tempered distributions
  \item Prove that scattered Hermite interpolation can be regular provided a given condition is satisfied.
\end{enumerate}

\section{Notations} 

$X = \{x_i\}$ with $x_i < x_{i+m}$ and $\lim_{i \to -\infty} x_i = -\infty,  \lim_{i \to \infty} x_i = \infty$ denotes a 
knot sequence of order $m$. Divided differences are renoted $\nu_{i,k}(f) = [x_i, \ldots, x_{i+k}]f$ and the B-splines 
are defined with Schoenberg's normalization that is $B_{i,m}(t) = m \nu_{i,m}({(.-t)}_{+}^{m-1})$. Let $\beta_{i,m}$ the 
$i^{th}$ B-spline with de Boor's normalization. Then

\begin{equation*}
  \beta_{i,m}(t) = (x_{i+m} - x_i) [x_i, \ldots, x_{i+m}]{(.-t)}_+^{m-1} \quad \text{while} \quad B_{i,m}(t) = m[x_i, 
  \cdots, x_{i+m}]{(.-t)}_+^{m-1}
\end{equation*}

\begin{equation*}
  \sum_{i} \beta_{i,m}(t) = 1 \quad \text{while} \quad \int_{\mathbb{R}} B_{i,m}(t) = 1
\end{equation*}

Note that for Schoenberg's normalization, we have the following property for $f \in \mathcal{C}^{m-1}$

\begin{equation}
  [x_i, \ldots, x_{i+m}]f = \int B_{i,m} \frac{f^{(m)}}{m!}
\end{equation}

Let $f(t) = \mathds{1}_{+}(t)$ the Heaviside function. Then for any test function $\psi$, $\langle f', \psi \rangle = 
\psi(0) = \langle \delta, \psi \rangle$. Therefore $\hat{f'}(w) = 1 = (jw)\hat{f}(w)$ i.e $\hat{f}(w) = \frac{1}{jw}$.  
Note now that $\frac{t_+^{m-1}}{(m-1)!} = f * \cdots * f$ with $f$ appearing $m$ times. Consequently the function $h(t) 
= \frac{t_+^{m-1}}{(m-1)!}$ is the fundamental solution to the differential operator $D^m$ that is $D^m h = \delta$. \\

The equality $\nu_{i,m}({(.-t)}_+^{m-1}) = {(-1)}^m \nu_{i,m}({(t-.)}_+^{m-1})$ is not so straightforward to me but can 
be verified manually for $m=2$. Maybe using $t_+^m = t^{m-1} t_+$? $\nu_{i,m}$ is a \textbf{functional} operator that 
can be viewed as a tempered distribution. Indeed, $\nu_{i,m}$ takes values in $\mathbb{C}$, $\nu_{i,m}$ is 
$\mathbb{C}$-linear and $\nu_{i,m}$ is continuous. For the latter, notice that from properties of divided differences we 
have

\begin{equation}
  \forall \phi, \psi \in \mathcal{S}(\mathbb{R}), \quad |\nu_{i,m}(\phi-\psi)| \leq \frac{1}{(m-1)!} {\|\phi-\psi\|}_{0, 
  m-1}
\end{equation}

where ${\|\phi\|}_{a,b}$ is the semi-norm $\sup_{t \in \mathbb{R}} |t^a \partial^b \phi(t)|$. As convergence in the 
Schwartz space is defined as the convergence in all such semi-norms, continuity of our operator follows. Consequently, 
$\nu_{i,m} \in \mathcal{S}'(\mathbb{R})$. \\

For any function $\phi$ in the Schwartz space, convolution with the tempered distribution $\nu_{i,m}$ is defined as 
$\phi * \nu_{i,m}(t) = \langle \phi(t-.), \nu_{i,m} \rangle := \nu_{i,m}(\phi(t-.))$. Now our function $h$ is not 
strictly speaking an element of the Schwartz space but can be viewed as the tempered distribution that associates to any 
test function $\phi$ the scalar quantity $\langle \phi, h \rangle = \int_{0}^{+\infty} \frac{t^{m-1}}{(m-1)!} 
\phi(t)dt$.  \textbf{How to define then convolution between two tempered distributions? And how to prove that $h * 
\nu_{i,m} = \nu_{i,m}(\frac{{(t-.)}^{m-1}_+}{(m-1)!})$}? \\

\textbf{Distribution with compact support}. A distribution $T$ is said to vanish on an open set $V$ if for every test 
function $\phi$ with compact support $\supp \phi \subseteq V$, $\langle \phi, T \rangle = 0$. Then the support of $T$ is 
the complement of the maximal open set on which the distribution vanishes. For example, the support of the Dirac 
distribution is $\{0\}$. \\

\textbf{It is possible to define the convolution of two distributions provided one has compact support}. From wikipedia, 
if $T,S$ are distributions with $T$ compactly supported, then for any test function $\phi$, $\langle \phi, S*T \rangle = 
\langle \psi, S \rangle$ with $\psi(t) = \langle \phi(t+.), T \rangle$ a test function itself. Applying this to the 
distributions $h$ and $\nu_{i,m}$ leads to $\langle \phi, h*\nu_{i,m} \rangle = \int_{0}^{\infty} \frac{t^{m-1}}{(m-1)!} 
\nu_{i,m}(\phi(t+.)) dt$ which I cannot relate to the formula in the article.

\begin{lem}{1}
  $\exists D_m > 0$ such that for any sequence $v = {(v_j)}_{j \in \mathbb{Z}}$
  \begin{equation*}
    D_m \|v\|_{l_2} \leq {\left\| \sum_{j \in \mathbb{Z}} v_j \sqrt{d_{j,m}} B_{j,m} \right\|}_{L_2} \leq \|v\|_{l_2}
  \end{equation*}
\end{lem}

\begin{proof}
  In de Boor's talk paper, the property holds with the basis $N_{im2} = d_{j,m}^{-\frac{1}{2}} \beta_{i,m}$. Now note 
  that $d_{j,m}^{\frac{1}{2}} B_{i,m} = d_{j,m}^{\frac{1}{2}} d_{j,m}^{-1} \beta_{i,m} = N_{im2}$.
\end{proof}

Let $\sigma_{v,m} = \sum_{j \in \mathbb{Z}} v_j \sqrt{d_{j,m}} B_{j,m}$. Another result proven again by de Boor in his 
book \emph{A Practical Guide to splines}, explicitly gives the derivative of a spline as

\begin{equation*}
  \sigma_{v, m}' = \sigma_{w, m-1}
\end{equation*}

with $w = D_m v$ and $D_m$ the lower two-banded matrix $d_{j,j}^m = {(d_{j,m-1}, d_{j,m})}^{-1/2}$, $d_{j, j-1}^m =  
-{(d_{j,m-1}, d_{j-1,m})}^{-1/2}$.

\section{Sobolev norm stability of splines}

Let $q_k := \inf_j d_{j,k}$. The Sobolev norm in $H^s = W^s_2$ is defined from the Fourier transform as
\begin{equation}
  {\| f \|}_{H^s} = {\left(\frac{1}{2\pi} \int_{\mathbb{R}} {(1+|w|^2)}^s |\hat{f}(w)|^2 dw \right)}^{1/2}
\end{equation}

In proof of lemma $2.1$, where does $x$ come from when bounding integral over $[\xi_j, \xi_{j+1}-x]$? From 
Cauchy-Schwarz!

\begin{lem}{2.1}
  Let $q_m > 0$. Then for any $v \in l_2(\mathbb{Z})$ and the corresponding spline
  \begin{equation*}
    f = \sigma_v  = \sum_{j \in \mathbb{Z}} v_j \sqrt{d_{j,m}} B_{j,m},
  \end{equation*}
  we have 
  \begin{equation}
    {\| f(.+x) - f\|}_{L_2}^2 \leq \text{const}_m \frac{|x|}{q_m} {\|v\|}_{l_2}^2 \quad \text{for} \quad |x| < q_m
  \end{equation}
\end{lem}

\begin{lem}{2.2}
  Let $q_m > 0$. Then for any $v \in l_2(\mathbb{Z})$ and the corresponding spline
  \begin{equation*}
    f = \sigma_v  = \sum_{j \in \mathbb{Z}} v_j \sqrt{d_{j,m}} B_{j,m},
  \end{equation*}
  we have
  \begin{equation}
    \int \int_{\mathbb{R}^2} \frac{|f(x)-f(y)|^2}{{|x-y|}^{1+2t}} dx dy \leq const_{m, q_m} \frac{1}{t(1-2t)} 
    {\|v\|}_{l_2}^2
  \end{equation}
\end{lem}

\begin{proof} Change of variable $x$ becomes $x+y$ followed by splitting of the integral on $x$ according to absolute 
value relatively to $q$. Application of lemma 2.1 completes the proof.
\end{proof}

\begin{thm}{1}
  Let $\mu \in \{1, \ldots, m\}$ with $q_{\mu} > 0$ and let $0 \leq t < \frac{1}{2}$. Then for any $v \in l_2$ and the 
  corresponding spline $f = \sigma_v$, we have
  \begin{equation}
    D_m \|v\|_{l_2} \leq {\left\| \sum_{j \in \mathbb{Z}} v_j \sqrt{d_{j,m}} B_{j,m} \right\|}_{H^{m-\mu+t}} \leq 
    \text{const}_{m,t,q_{\mu}}\|v\|_{l_2}
  \end{equation}
\end{thm}

\begin{proof}
  Only the upper bound has to be verified as for any $s \geq 0$,
  \begin{align*}
    {\|f\|}_{H^s}^2 &= \frac{1}{2\pi} \int {(1+|w|^2)}^s |\hat{f}(w)|^2 dw \\
    &\geq \frac{1}{2\pi} \int |\hat{f}(w)|^2 dw \\
    & = {\|f\|}_{L_2}^2
  \end{align*}
\end{proof}

\section{Riesz bases of regularized splines}

$X$ is minimally separated of order $m$ if $q_m > 0$. As before we define functions $B_{i,m}$ as
\begin{equation*}
  B_{i,m}(t) = m! {(-1)}^m h*\nu_{i,m}(t)
\end{equation*}

with $h$ a \textbf{tempered distribution} satisfying for some integer $m$
\begin{enumerate}
  \item $\hat{h} \in \mathcal{C}^0(\mathbb{R}^*)$
  \item $\hat{h} \neq 0$ on $\mathbb{R}^*$
  \item ${(jw)}^m\hat{h}(w) \in L_{\infty}(\mathbb{R})$
  \item ${\left({(jw)}^m\hat{h}(w)\right)}^{-1} \in L_{\infty}(U_0)$ for some $U_0 \in \mathcal{V}(0)$.
\end{enumerate}

In that case let $G(w) = |w|^{2m} |\hat{h}(w)|^2$ and define the matrix $B = {(b_{i,j})}_{(i,j) \in \mathbb{Z}^2}$ by
\begin{equation}
  b_{i,j} = \frac{\sqrt{d_{i,m}d_{j,m}}}{2\pi} \int G(w) \hat{B}_{i,m}(w) \overline{\hat{B}_{j,m}(w)} dw
\end{equation}

Clearly $B$ is Hermitian and allows us to define the quadratic form $\langle By, y \rangle = y^* By$ for any $y \in 
l_2^0$. In fact it is possible to extend to all of $l_2$ given that
\begin{equation*}
  |\langle By, y\rangle| \leq \Gamma {\|y\|}_{l_2}^2
\end{equation*}

with $\Gamma = {\|G\|}_{L_{\infty}}$.

\begin{thm}{2}
  Let $q_m > 0$. Then $\exists \gamma > 0$ such that $\forall y \in l_2$, 
  \begin{equation}
    \gamma {\|y\|}_{l_2}^2 \leq |\langle By, y \rangle|
  \end{equation}
\end{thm}

\begin{proof} The proof relies on the fact that $\essinf_{|w| \leq n} G(w) := \gamma_n > 0$ for any $n > 0$. For that 
  let $\epsilon > 0$, $\epsilon < n$ such that ${|\left({(jw)}^m\hat{h}(w)\right)|}^{-1}$ is bounded above on 
  $[-\epsilon, \epsilon]$ by some $\gamma_{\epsilon} > 0$. Then, forall $w \in [-\epsilon, \epsilon]$,
  \begin{equation*}
    {|w|}^{2m} |\hat{h}(w)|^2 \geq \frac{1}{\gamma_{\epsilon}^2}
  \end{equation*}

  The function $w \to   {|w|}^{2m} |\hat{h}(w)|^2$ is continuous on $[\epsilon, n]$ and does not vanish there. Thefore 
  there exists $\gamma_{+} > 0$ such that  ${|w|}^{2m} |\hat{h}(w)|^2 \geq \gamma_{+}$ on $[\epsilon, n]$. The same 
  holds for some $\gamma_{-}$ on $[-n, -\epsilon]$ and $\gamma_n \geq \min \{\frac{1}{\gamma_{\epsilon}^2}, \gamma_+, 
  \gamma_-\}$.
\end{proof}

This result can be reformulated in terms of generalized splines. For that let $\mu_j = \mu_{j,m} = m! \sqrt{d_{j,m}} 
\nu_{j,m}$ the modified divided difference and $\phi_j = h * \mu_j$ (${(-1)}^m$, disappeared but it is not important?).  
This convolution is well-defined given that $\mu_j$ has compact support. Then,

\begin{thm}{3}
  Suppose $q_m > 0$, assumptions on $h$ are satisfied. Let for $i,j \in \mathbb{Z}$
  \begin{equation*}
    b_{i,j} = {\langle \phi_i, \phi_j \rangle}_{L_2}
  \end{equation*}

  i.e $B$ is the Gramian matrix of $(\phi_i)$. Let
  \begin{equation*}
    V_X := \text{clos}_{L_2} \text{span} \{\phi_i, i \in \mathbb{Z}\}
  \end{equation*}
  
  Then $(\phi_i)$ is a Riesz basis for $V_X$ with
  \begin{equation}
    \gamma {\|a\|}_{l_2}^2 \leq {\left\| \sum_{j \in \mathbb{Z}} a_j \phi_j \right\|}_{L_2}^2 \leq \Gamma 
    {\|a\|}_{l_2}^2
  \end{equation}
\end{thm}


\end{document}


