\documentclass[a4paper, 11pt]{article}

\usepackage[left=1.5cm, right=1.5cm, top=2cm, bottom=2cm]{geometry}

\usepackage[utf8]{inputenc}
\usepackage[T1]{fontenc}
\usepackage[english]{babel}  
\usepackage{lmodern}

\usepackage{amsmath, amsthm, amssymb}
\usepackage{mathtools}
\usepackage{dsfont}

\newtheorem{innercustomgeneric}{\customgenericname}
\providecommand{\customgenericname}{}
\newcommand{\newcustomtheorem}[2]{%
  \newenvironment{#1}[1]
  {%
   \renewcommand\customgenericname{#2}%
   \renewcommand\theinnercustomgeneric{##1}%
   \innercustomgeneric}
  {\endinnercustomgeneric}
}

\newcustomtheorem{thm}{Theorem}
\newcustomtheorem{lem}{Lemma}
\newcustomtheorem{cor}{Corollary}
\newcustomtheorem{deftn}{Definition}
\newcustomtheorem{prop}{Proposition}
\newcustomtheorem{remark}{Remark}
\DeclareMathOperator*{\argmin}{argmin} 
\DeclareMathOperator*{\argmax}{argmax}
\DeclareMathOperator*{\Tr}{Tr}
\DeclareMathOperator*{\sinc}{sinc} 
\DeclareMathOperator*{\supp}{supp} 
\DeclareMathOperator*{\essinf}{ess\ inf}
\DeclareMathOperator*{\esssup}{ess\ sup} 

\begin{document}
\title{Summary paper 14: Cardinal exponential splines: Part I}
\author{Yoann Pradat}
\maketitle

This paper by M. Unser and T. Blu is part of a larger research program that is intent on bridging the gap between 
continuous and discrete signal processing. In that matter, polynomial splines are of high interest although the amount 
of research for that specific issue remains limited as polynomial splines found to be advantageous elsewhere, notably in 
the context of high-quality interpolation. In the paper authors consider a larger class of splines, namely 
\textbf{exponential splines} with a restriction to the case of cardinal splines as these are the most suited for signal 
processing, lending themselves to very efficient digital filtering techniques while using concepts and algorithms 
familiar to other communities as the “signal-and-systems” community. \\

The mathematical concepts involved in the paper include Fourier analysis, generalized functions, Green's function of 
ordinary differential operators and z-transforms. As I am not so familiar with the latter, here is a short topo on the 
subject.

\section{Preliminaries}

\paragraph{Topo on Z-transform} \mbox{} \\

The Z-transform converts a discrete-time signal into a complex frequency-domain representation. Can be viewed 
discrete-time equivalent of (unilateral) Laplace transform $\mathcal{L}{f}(s) = F(s) = \int_{0}^{\infty} f(t)e^{-st}dt$.
The (bilateral) Z-transform of a discrete time signal ${(a[k])}_k$ is

\begin{equation*}
  A(z) = \sum_{k=-\infty}^{\infty} a[k]z^{-k}
\end{equation*}

The region of convergence (ROC) is the set of points where the summation converges i.e
\begin{equation*}
  \text{ROC} = \left\{z: \left| \sum_{k=-\infty}^{\infty} a[k]z^{-k} \right| < \infty \right\}
\end{equation*}

The inverse Z-transform is 

\begin{equation*}
  a[k] = \frac{1}{2\pi j} \int_{C} A(z)z^{k-1}dz
\end{equation*}

with $C$ a counterclockwise path encircling the origin and entirely in the ROC\@. In case the ROC is causal, $C$ must 
enclose all poles of $A(z)$. In case where $C$ is the unit circle, the inverse Z-transform simplifies into the inverse 
discrete-time Fourier transform i.e

\begin{equation*}
  a[k] = \frac{1}{2\pi}\int_{0}^{2\pi} A(e^{jw})e^{jwk} dw
\end{equation*}

\underline{Examples}
\begin{enumerate}
  \item $a[k] = \frac{1}{2^k}$ for $k \in \mathbb{Z}$. Then ROC$=\emptyset$.
  \item $a[k] = \frac{1}{2^k}$ for $k \in \mathbb{Z}_{+}$. Then $\displaystyle \sum_{k=0}^{\infty} a[k]z^{-k} = 
    \frac{1}{1-\frac{1}{2z}}$
    and ROC=$\{z: |z| > \frac{1}{2}\}$. The ROC is causal.
  \item $a[k] = \frac{1}{2^k}$ for $k \in \mathbb{Z}_{-}$. Then $\displaystyle \sum_{k=0}^{\infty} {a[k]}^{-1}z^{k} = 
    \frac{1}{1-\frac{1}{2z}}$
    and ROC=$\{z: |z| < \frac{1}{2}\}$. The ROC is anti-causal.
\end{enumerate}

In case the ROC includes neither $z=0$ nor $|z| = \infty$, we say that the system is mixed-causality. The Z-transform of 
$\delta$ is 1, that of $a[k-k_0]$ is $A(z)z^{-k_0}$. A lot more properties can be found on wiki's page. 

\paragraph{Preliminaries of article} \mbox{} \\

Generic differential operator of order $N$ 

\begin{equation*}
  L{f} = D^N{f} + a_{N-1}D^{N-1}{f} + \cdots + a_0 I{f}
\end{equation*}

$L = L_{\alpha}$ with $\alpha$ the roots of the characteristic polynomial. The “Fourier transform” of the operator 
$L_{\alpha}$ is
\begin{equation*}
  L_{\alpha}(jw) = \prod_{n=1}^N (jw-\alpha_n)
\end{equation*}

Meaning that $\widehat{L_{\alpha}f}(w) = L_{\alpha}(jw) \hat{f}(w)$. Let's try to clarify that. Let $L$ be a continuous 
LSI operator from $\mathcal{S} \to \mathcal{S}'$. From Schwartz kernel theorem we know there exists a (generalized) 
function $h_L \in \mathcal{S}'$ such that

\begin{equation*}
  L{f} = h_L*f \quad \text{meaning} \quad L{f}(t) = \langle f(t-.), h_L \rangle
\end{equation*}

$h_L$ is related to $L$ by $h_L = L{\delta}$. In case $L=D$ for instance, $h_L = D{\delta}$ i.e it is the distribution 
that associates to any test function $\psi$ the scalar $-\psi'(0)$. We can now use the properties of the Fourier 
transform directly on this convolution as follows

\begin{equation*}
  \widehat{L{f}}(w) = \hat{h}_L(w) \hat{f}(w)
\end{equation*}

Are we sure though that $\hat{h}_L(w)$ can always be interpreted as a usual function? In case $L=D$ note for instance 
that $\hat{h}_L$ is the distribution that associates to each $\psi$ the quantity $\int jw \psi(w)dw = -\hat{\psi}'(0)$ 
that is $\hat{h}_L$ can be intepreted as the conventional function $\hat{h}_L(w)= jw$. \\

\textbf{Why is the nullspace of $L_{\alpha}$ of dimension $N$}. For that note the following Cauchy theorem 

\begin{thm}{Cauchy}
  Let $E$ a finite-dimensional vector space on $\mathbb{K}$ real or complex field. The differential 
  system
  \begin{equation*}
    X'(t) = A(t) X(t) + B(t)
  \end{equation*}

  with $A \in \mathcal{C}(I, \mathcal{L}(E)), B \in \mathcal{C}(I, E)$ is such that
  \begin{enumerate}
    \item for any $(t_0, X_0) \in I\times E$, $\exists$! solution to the system that satisfies $X(t_0) = X_0$.
    \item for any $t_0 \in I$, the map $\phi: X \in S_0 \mapsto X(t_0) \in E$ is an isormorphism between 
      $\mathbb{K}$-vector spaces.
  \end{enumerate}
\end{thm}

\begin{align*}
  y \in S &\iff y \ \text{solution to} \ y^{(N)} + a_{N-1} y^{(N-1)} + \cdots + a_0 y =0 \\
  &\iff Y = \begin{bmatrix} y \\ \vdots \\ y^{(N-1)} \end{bmatrix} \ \text{solution to} \ Y' = AY
\end{align*}

with $A = \begin{bmatrix} 0 & 1 & 0 & \cdots & 0 \\
  0 & 0 & 1 & \cdots & 0 \\
  \vdots & \vdots & \ddots & \ddots & \vdots \\
  -a_{N-1} & -a_{N-2} & -a_{N-3} & \cdots & -a_0  \\
\end{bmatrix}$

In link with the formulation in the theorem, $A \in \mathcal{C}(\mathbb{R}, \mathcal{L}(\mathbb{C}^N))$ and therefore 
the space of solutions to $Y'=AY$, $T$, is of dimension $N$. However $S$ and $T$ are isomorphic given the equivalence 
above, hence $\dim{S} = N$.

\begin{deftn}{1}
  An exponential spline with parameter $\alpha$ and knots $-\infty < \cdots < t_{k} < t_{k+1} < \cdots < \infty$ is a 
  function $s(t)$ such that
  \begin{equation*}
    L_{\alpha}\{s(t)\} = \sum_{k} a_k \delta(t-t_k)
  \end{equation*}
\end{deftn}

The space spanned by exponential polynomials is shift-invariance. Specifically, for any shift $\tau$, we have
\begin{equation*}
  {(t-\tau)}^n e^{\alpha(t-\tau)} = \sum_{m=1}^n a_{\tau, m} t^m e^{\alpha t}
\end{equation*}

\section{Cardinal exponential splines}

Exponential $B$-splines are localized, shortest-possible version of Green's functions that generate the exponential 
splines. The first-order $B$-splines with parameter $\alpha$ is 

\begin{equation}
  \beta_{\alpha}(t) = \rho_{\alpha}(t) - e^{\alpha} \rho_{\alpha}(t-1)
\end{equation}

$B$-splines are always well-defined and compactly supported. Changing the sign of $\alpha$ has the effect
\begin{equation*}
  \beta_{-\alpha}(t) = \left( \prod_{i=1}^N e^{-\alpha_i} \right) \beta_{\alpha}(-t+N)
\end{equation*}

In case component of $\alpha$ can be grouped into opposite signs pair, $\beta_{\alpha}$ is symmetric wrt its center 
line. 

\begin{thm}{1}
  The set of functions ${\{\beta_{\alpha}(t-k)\}}_{k \in \mathbb{Z}}$ provides a Riesz-basis of $V_{\alpha}$ iif 
  $\alpha_n - \alpha_m \not\in 2\pi\mathbb{Z}$ for all pairs of distinct, purely imaginary roots.
\end{thm}

There are two important aspects to this theorem: completeness and stability. Formally we can reconstruction the Green 
function by inverting the relation $\beta_{\alpha} = \Delta_{\alpha}\{\rho_{\alpha}\}$ leading to 

\begin{equation*}
  \rho_{\alpha}(t) = \sum_{k=0}^{\infty} p_{\alpha}[k]\beta_{\alpha}(t-k)
\end{equation*}

with $p_{\alpha}[k]$ explicitly computable. In the first-order case extended to the negative domain we have

\begin{equation*}
  e^{\alpha t} = \sum_{k=-\infty}^{\infty} e^{\alpha k}\beta_{\alpha}(t-k)
\end{equation*}

\begin{prop}{2}
  Let $\varphi_{\alpha}$ a function that reproduces exponential polynomials in $\mathcal{N}_{(\alpha, \ldots, \alpha)}$.  
  Then for any function $\varphi$ such that $\int \varphi(t)e^{-\alpha t}dt \neq 0$, $\varphi * \varphi_{\alpha}$ also 
  reproduce these exponential polynomials.
\end{prop}

\begin{proof} (Theorem 1)
  The completeness is a consequence of the reproduction properties discussed above. The stability result is more tricky.  
  The upper Riesz bound is easily obtained for compactly supported propototype as we have
  \begin{align*}
    \sum_k |\beta_{\alpha}(w+2k\pi)|^2 &= \sum_k \langle \beta_{\alpha}, \beta_{\alpha}(.-k) \rangle e^{-jwk} \\
    & \leq \sum_{k \in \mathbb{Z}}  |\langle \beta_{\alpha}, \beta_{\alpha}(.-k) \rangle| \\
    & < \infty
  \end{align*}

  More difficult is to prove the existence of a lower Riesz bound. Since $A_{\alpha}(w)$ is continuous and 
  $2\pi$-periodic, we have a lower Riesz bound if and only if $A_{\alpha}(w) > 0$ on $[0,2\pi]$ (in that particular case 
  our vector space is generated by a single prototype which makes the Gram matrix a scalar) that is $\bigcap_{k \in 
  \mathbb{Z}} Z_{k, \alpha} = \emptyset$. The author says when $\Re(\alpha_n)=0$ there is always one  $Z_{k, \alpha}$ 
  that is empty \textbf{which leads to an empty intersection}??
\end{proof}

\begin{prop}{3}
  The upper Riesz bound can be estimated by 
  \begin{equation*}
    \sup_{w \in \mathbb{R}} |\hat{\beta}_{\alpha}(w)| \leq R_{\alpha} \leq \frac{M_{\alpha}}{\sqrt{\max_{1\leq n \leq N} 
    M_{-|\alpha_n|}}}
  \end{equation*}
\end{prop}

\begin{prop}{4}
  If the roots are such that $\Im{\alpha_n} - w_0 \in (-\pi, \pi)$ for all $1\leq n \leq N$, then we have the following 
  lower Riesz bound estimate
  \begin{equation*}
    r_{\alpha} \geq M_{\alpha} \prod_{n=1}^N \frac{2\cos(\frac{\Im{\alpha_n}-w_0}{2})}{\pi + |\Im{\alpha_n}-w_0|}
  \end{equation*}
\end{prop}

\begin{thm}{2}
  Let $f \in L_2$ be a function such that $D^N f \in L_2$ and let $P_T f$ denote its orthogonal projection onto the 
  spline space $V_{\alpha, T}$ at scale $T$. Then,
  \begin{equation*}
    \|f - P_T f\|_{L_2} = C_N T^N \|L_{\alpha}f \|_{L_2} \ \text{as} \ T \to 0
  \end{equation*}
\end{thm}

\end{document}


