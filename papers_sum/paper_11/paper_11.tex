\documentclass[a4paper, 11pt]{article}

\usepackage[left=1.5cm, right=1.5cm, top=2cm, bottom=2cm]{geometry}

\usepackage[utf8]{inputenc}
\usepackage[T1]{fontenc}
\usepackage[english]{babel}  
\usepackage{lmodern}

\usepackage{amsmath, amsthm, amssymb}
\usepackage{mathtools}

\newtheorem{innercustomgeneric}{\customgenericname}
\providecommand{\customgenericname}{}
\newcommand{\newcustomtheorem}[2]{%
  \newenvironment{#1}[1]
  {%
   \renewcommand\customgenericname{#2}%
   \renewcommand\theinnercustomgeneric{##1}%
   \innercustomgeneric}
  {\endinnercustomgeneric}
}

\newcustomtheorem{thm}{Theorem}
\newcustomtheorem{lem}{Lemma}
\newcustomtheorem{cor}{Corollary}
\newcustomtheorem{deftn}{Definition}
\newcustomtheorem{prop}{Proposition}
\DeclareMathOperator*{\argmin}{argmin} 
\DeclareMathOperator*{\argmax}{argmax}
\DeclareMathOperator*{\Tr}{Tr}

\begin{document}
\title{Summary paper 11: Snakes with an ellipse-reproducing property Delgado et al}
\author{Yoann Pradat}
\maketitle

[3] Reviews of snakes energies and algorithms Jacob, Blu, Unser “Efficients energies and algorithms for parametric 
snakes”. Snakes are popular tools for outlining in all sorts of images. They vary from one another by the type of snake 
used and, in the case on continuous snakes, by the basis generators of the representation. Snakes may be classified into

\begin{enumerate}
  \item discrete snakes that make use of a large number of control points
  \item continuous snakes
  \item implicit snakes where representation of the curve is described at level set of a surface.
\end{enumerate}

Continous snakes are parametrized by only a few points which, along with small support of the basis generators, makes 
computations very fast. The drawback from having only a few anchor points is of course that not every curve can be 
represented, only those in the span of integer shifts of the generators are perfectly reproduced. However good 
approximation can be obtained by simply increasing the number of control points. In this paper authors propose 
generators that have minimal support, an important feature for fast and efficient algorithms and that have the property 
of perfectly reproducing ellipses. \\

\underline{Remark} In the parametric representation of a curve we always consider the space spanned by integer shifts of 
the basis generators as this allows to take advantage of fast and stable algorithms.

\paragraph{Desired properties of the snake} \mbox{} \\

In here we are interested in reproducing closed curves from a $M$-periodic sequence ${\{c[k]\}}_{k \in \mathbb{Z}}$ of 
control points in the following form

\begin{equation}
  r(t) = \sum_{k \in \mathbb{Z}} c[k] \varphi (Mt-k)
\end{equation}

There are some properties we would like $\varphi$ to satisfy in order to have a good snake model. As usual the properties 
are

\begin{enumerate}
  \item Representation should be unique and numerically stable. Verified if $\varphi$ satisfies Riesz-basis property.
  \item Affine invariance i.e representation of affinely transformed curve is the affine transformation of the 
    representation. Verified iif $\varphi$ satisfies partition of unity condition
  \item Well-defined curvature that is $\kappa(x_1, x_2) = \frac{\dot{x_1}\ddot{x_2} - \ddot{x_1}\dot{x_2}}{{(\dot{x_1} 
    + \dot{x_2})}^{\frac{3}{2}}}$ defined everywhere. For this $\varphi$ must be twice-diff with bounded second derivative.
\end{enumerate}

\paragraph{Reproduction of ellipses} \mbox{} \\

Given affine invariance it is enough to reproduce the unit sphere in order to reproduce any ellipse. A parametric snake 
is said to reproduce the unit sphere with $M$ anchor points if there exists two $M$-periodic sequences ${\{c_1[k]\}}_{k 
\in \mathbb{Z}}$, ${\{c_2[k]\}}_{k \in \mathbb{Z}}$ such that

\begin{align}
  \cos (2\pi t) &= \sum_{k \in \mathbb{Z}} c_1[k] \varphi (Mt-k) \\
  \sin (2\pi t) &= \sum_{k \in \mathbb{Z}} c_2[k] \varphi (Mt-k)
\end{align}

Here is now the main result of the paper that explicitly give the expression of the generate $\varphi$ with all 
properties stated above.

\begin{thm}{1}
  The centered generating function with minimal support that has the Riesz-basis, partition of unity, twice 
  differentiable with bounded second derivative and ellipse-reproducing with $M$ anchor points properties is given by
    \begin{equation}
      \varphi(t) = \begin{dcases}
        \frac{\cos (\frac{2\pi |t|}{M}) \cos (\frac{\pi}{M}) - \cos (\frac{2\pi}{M})} {1- \cos (\frac{2\pi}{M})} 
        &\text{if } 0 \leq |t| < \frac{1}{2} \\
        \frac{1- \cos (\frac{2\pi(\frac{3}{2} - |t|)}{M})}{2(1 - \cos (\frac{2\pi}{M}))} &\text{if } \frac{1}{2} \leq 
      |t| < \frac{3}{2} \\
      0 &\text{if } |t| \geq \frac{3}{2}
    \end{dcases}
    \end{equation}
\end{thm}

This is proved using a theorem from paper 10-“Exponential splines and minimal-support bases for curve representation” 
that will be read after this one. The theorem states that every minimal support function that reproduces $e^{\alpha_n 
t}$ for $n=0, \ldots, N-1$, with $\alpha$ such that there it has no pair of purely imaginary numbers separated by a 
distance that is a multiple of $2\pi$, can be written as

\begin{equation*}
  \varphi(t) = \sum_{n=0}^{N-1} \lambda_n \frac{d^n t}{dt^n}\beta_{\alpha}(t-a)
\end{equation*}

with $a$ is an arbitrary shift that correspond to the lower extremity of the support of $\varphi$. In order to satisfy 
partition of unity and reproduce sinusoids we need to choose $\alpha = (\frac{-2j\pi}{M}, 0,\frac{2j\pi}{M})$. For such 
an $\alpha$, $\beta_{\alpha}$ has a discontinuous second derivative. This leads us to choosing $\lambda_1 = \lambda_2 = 
0$. In the end, given the property of exponential B-splines, we get

\begin{equation}
  \lambda_0 = \frac{{(\frac{2\pi}{M})}^2}{2(1-\cos(\frac{2\pi}{M}))}
\end{equation}

\paragraph{Approximation properties of $\varphi$} \mbox{} \\

Noticing that $\varphi$ converges to the quadratic B-spline as $M$ grows to infinity, we can expect $\varphi$ to have 
approximately the same approximation properties as that of a quadractic B-spline. The space spanned by $\varphi$ is not 
shift-invariance in general. Therefore the approximation error using $M$ coefficients is dependent upon a shift in the 
continous parameter $t$ of the function of unit period $s$. The minimum mean-square approximation error for a shift 
function is

\begin{equation}
  \gamma(\tau, M) = \int_0^1 \| s(.-\tau) - r(.) \|^2 = {\| s(.-\tau) - r(.) \|}_{L_2([0,1])}^2 
\end{equation}

where $r$ is the best approximation from the span ${\{\varphi(M.-k)\}}_{k \in \mathbb{Z}}$. Since $\tau$ is usually 
unknown, the measure error is average over all possible shift that is

\begin{equation}
  \eta(M) = {\left(\int_0^1 \gamma(\tau, M) d\tau\right)}^{\frac{1}{2}}
\end{equation}

It is shown in the paper, based on the main result of [30] M. Jacob, T. Blu, and M. Unser, “Sampling of periodic 
signals: A quantitative error analysis” that $\eta(M) = \mathcal{O}(M^{-3})$. 

\paragraph{Explicit coefficients for best ellipse fitting and sinusoids} \mbox{} \\

Since the snake can reproduce curves, it is natural to wonder what is the best ellipse that approximates a curve $r$ 
defined by the $M$-periodic sequence ${\{c[k]\}}_{k \in \mathbb{Z}}$ that is find $r_e$ that minimizes

\begin{equation*}
  \int_0^1 \|r(t) - r_e(t)\|^2 dt
\end{equation*}

Aparte. For a function $f: \mathbb{R} \to \mathbb{C}$ that is $2\pi$-periodic one can represent $f$ in its Fourier 
series as 

\begin{equation*}
  f(t) = \sum_{k \in \mathbb{Z}} \frac{\hat{f}[k]}{2\pi} e^{jkt}
\end{equation*}

with $\hat{f}[k] = \int_{-\pi}^{\pi} f(t)e^{-jkt} dt$. What about a function $g$ that is $\frac{2\pi}{a}$-periodic? Let 
$g(t) = f(at)$. We have

\begin{equation*}
  \hat{f}[k] = a\int_{-\frac{\pi}{a}}^{\frac{\pi}{a}} f(at)e^{-jkat} dt
\end{equation*}

and
\begin{equation*}
  g(t) = \sum_{k \in \mathbb{Z}} \frac{\hat{f}[k]}{2\pi} e^{jkat}
\end{equation*}

For example for $a = 2\pi$ i.e $g$ is 1-periodic
\begin{equation*}
  g(t) = \sum_{k \in \mathbb{Z}} \left(\int_{-\frac{1}{2}}^{\frac{1}{2}} g(u)e^{-2\pi j ku}du\right)  e^{j2\pi kt}
\end{equation*}

The continuous 1-periodic function $r$ can be represented as 

\begin{equation}
  r(t) = \sum_{k \in \mathbb{Z}} R[k] e^{j2\pi kt}
\end{equation}

with $R[k] = \left(\int_{0}^{1} g(u)e^{-2\pi j ku}du\right)$. From classical result in harmonic analysis, the best 
approximating ellispe $r_e$ is the first-order truncation of the series above i.e $R[0], R[1], R[-1]$. \\

Finally we derive explicit formulas for the coefficient of the sinusoids of unit period with $M$ anchor points. Recall 
the exponential-reproducing property of exponential B-spline

\begin{equation}
  e^{\alpha t} = \sum_{k \in \mathbb{Z}} e^{\alpha k} \beta_{\alpha}(t-k)
\end{equation}


Taking $\alpha = \frac{2j\pi}{M}$ and convolving both sides above with $\beta_{(0, \frac{-2j\pi}{M})}$ allow us to find 
the coefficients for representing $\cos (2\pi (t+\frac{3}{2}))$ and the equivalent $\sin$ function in the span of
${\{\varphi(M.-k)\}}$

\begin{align}
  c_1[k] &= \frac{2(1-\cos (\frac{2\pi}{M}))\cos (\frac{\pi(2k+3)}{M})}{\cos (\frac{\pi}{M}) - \cos (\frac{3\pi}{M})} \\
  c_2[k] &= \frac{2(1-\cos (\frac{2\pi}{M}))\sin (\frac{\pi(2k+3)}{M})}{\cos (\frac{\pi}{M}) - \cos (\frac{3\pi}{M})}
\end{align}

\end{document}




