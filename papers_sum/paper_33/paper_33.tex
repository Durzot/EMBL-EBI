\documentclass[a4paper, 11pt]{article}

\usepackage[left=1.5cm, right=1.5cm, top=2cm, bottom=2cm]{geometry}

\usepackage[utf8]{inputenc}
\usepackage[T1]{fontenc}
\usepackage[english]{babel}  
\usepackage{lmodern}

\usepackage{amsmath, amsthm, amssymb}
\usepackage{mathtools}
\usepackage{dsfont}
\usepackage{bm}

\newtheorem{innercustomgeneric}{\customgenericname}
\providecommand{\customgenericname}{}
\newcommand{\newcustomtheorem}[2]{%
  \newenvironment{#1}[1]
  {%
   \renewcommand\customgenericname{#2}%
   \renewcommand\theinnercustomgeneric{##1}%
   \innercustomgeneric}
  {\endinnercustomgeneric}
}

\newcustomtheorem{thm}{Theorem}
\newcustomtheorem{lem}{Lemma}
\newcustomtheorem{cor}{Corollary}
\newcustomtheorem{deftn}{Definition}
\newcustomtheorem{prop}{Proposition}
\newcustomtheorem{remark}{Remark}
\DeclareMathOperator*{\argmin}{argmin} 
\DeclareMathOperator*{\argmax}{argmax}
\DeclareMathOperator*{\Tr}{Tr}
\DeclareMathOperator*{\sinc}{sinc} 
\DeclareMathOperator*{\supp}{supp}
\DeclareMathOperator*{\essinf}{ess\ inf}
\DeclareMathOperator*{\esssup}{ess\ sup} 

\begin{document}
\title{Summary paper 33: Wavelets of multiplicity $r$}
\author{Yoann Pradat}
\maketitle

\section{Introduction}

This paper by T. Goodman applies the general theory that derive necessary and sufficient conditions for translates of 
some functions $\phi_1, \ldots, \phi_r$, $\psi_1, \ldots, \psi_r$ to form a Riesz basis for $V_1$ in order to construct 
spline wavelets with multiple knots. \\

\underline{Notations}
\begin{itemize}
  \item ${l^2(\mathbb{Z})}^r = \{(s_1, \ldots, s_r) | s_j \in l^2(\mathbb{Z}), j=0, \ldots, r-1\}$
\end{itemize}

\begin{deftn}{1}
  A \emph{multiresolution approximation} of multiplicity $r$ is a sequence of closed subspaces ${(V_m)}_{m \in 
  \mathbb{Z}}$ of $L^2$ such that
  \begin{enumerate}
    \item $V_m \subset V_{m+1}$
    \item $\bigcup_{m \in \mathbb{Z}} V_m$ is dense in $L^2$ and $\bigcap_{m \in \mathbb{Z}} V_m = \{0\}$
    \item $f \in V_m \implies D_2f \in V_{m+1}$ with $D_a f(x) = f(ax)$
    \item $f \in V_m \implies T_{2^{-m}n}f \in V_m$ for all $n \in \mathbb{Z}$ with $T_{\tau}f(x) = f(x-\tau)$
    \item $\exists$ isomorphism $\mathcal{I}: V_0\to{l^2(\mathbb{Z})}^r$ which commutes with the action of $\mathbb{Z}$
      i.e $\mathcal{I}T_k = t_k \mathcal{I}$ with $t_k$ the translation by $k$ on sequences of ${l^2(\mathbb{Z})}^r$.
  \end{enumerate}
\end{deftn}

In his famous (among many) article\cite{Mallat}, Mallat has given a general construction of wavelets $\phi \in V_0$ and 
$\psi \in W_0$ ($V_1 = V_0 \bigoplus W_0$) such that ${\{T_n \phi\}}_n, {\{T_n \psi \}}_n$ are orthonormal bases of 
$V_0$ and $W_0$ respectively.  Furthermore
\begin{align*}
  \phi_{m,n} = \sqrt{2^m} \phi(2^m x-n) \quad (m,n) \in \mathbb{Z}^2 \\
  \psi_{m,n} = \sqrt{2^m} \psi(2^m x-n) \quad (m,n) \in \mathbb{Z}^2 
\end{align*}

are orthonormal basis for $V_m$ and $W_m$ respectively ($V_{m+1} = V_m \bigoplus W_m$). Goodman refers to another of his 
works \emph{Wavelets in wandering spaces} where he extended Mallat's results to multiresolution approximation of 
multiplicity $r$. \\

Cardinal $B$-splines generate a large class of simple multiresolution approximations. This was studied extensively by 
Chu et Wang in \emph{A cardinal spline approach to wavelets} and \emph{A general framework of compactly supported 
splines and wavelets}. The present papers has following contributions
\begin{enumerate}
  \item New general duality principle
  \item Cardinal spline wavelets with multiple knots. They generate nonorthonormal Riesz spaces for the $V_m$. 
  \item Considers special case $n=r-1$, $n=2r-1$ with $n$ the degree of spline functions
\end{enumerate}

\section{Wavelets for multiresolution approximations of multiplicity $r$}

Note that $\{ f \in L^1, \hat{f} \in L^1\} \subsetneq L^1 \cap L^2$. Indeed if $\hat{f} \in L^1$, $\|f\|_{\infty} \leq 
\frac{{\|\hat{f}\|}_{L^1}}{{(2\pi)}^d}$ and thus $f \in L^2$. However $\mathds{1}_{{[-1,1]}^d}$ is in $L^1 \cap L^2$ but 
not in the first set.  Also if $f \in L^1$ is such that $\hat{f} \in L^2$ then $f \in L^2$. \\ 

Fourier transform of $f \in L^2(\mathbb{R})$ is said to be regular if $\hat{f}$ is continuous and $\hat{f}(w) = 
\mathcal{O}(|w|^{-1})$ as $|w| \to \infty$. This makes $\hat{f}$ a function in $L^2$. In fact, $\mathcal{F}$ is a 
continuous automorphism of $L^2$ and therefore the simple fact of $f$ being in $L^2$ makes its Fourier transform an 
element of $L^2$.  Is the author's condition stronger? \\

Let $\phi = {(\phi_j)}_{j=1}^r \in L^2$ such that $\sum_{n \in \mathbb{Z}} s(n) T_n \phi_j \in L^2$ for any $(s(n)) \in 
l^2$ and similarly let $\psi = {(\psi_j)}_{j=1}^r \in L^2$ such that $\sum_{n \in \mathbb{Z}} s(n) T_n \psi_j \in L^2$ 
for any $(s(n)) \in l^2$.  $\phi_j$ and $\psi_j$ are \textbf{assumed to be regular}. \\

\underline{Notations}
\begin{itemize}
  \item $V_0 = \text{span} \overline{\{ T_n \phi_j, n \in \mathbb{Z}, 1 \leq j \leq r\}}$
  \item $V_1 = \{D_2 f, f\in V_0\}$, \textbf{assume} $V_0 \subset V_1$
  \item \textbf{Assume} $\psi_j \in V_1$, $j=1, \ldots, r$.
  \item $V_1 = V_0 \bigoplus W_0$
  \item $\tilde{L}^2_{r\times r}(0, 2\pi)$ $r\times r$ matrices with entries $2\pi$-periodic and square-integrable.
  \item $\tilde{\mathcal{C}}^2_{r\times r}(0, 2\pi)$ $r\times r$ matrices with entries $2\pi$-periodic and continuous.
\end{itemize}

As $\phi_j, \psi_j$ are elements of $V_1$ while $\{T_n D_2 \phi_j\}$ spans $V_1$, there exist matrices $P_n$ and $Q_n$ 
such that

\begin{align*}
  \phi(x) &= 2 \sum_{n \in \mathbb{Z}} P_n \phi(2x-n) \\
  \psi(x) &= 2 \sum_{n \in \mathbb{Z}} Q_n \phi(2x-n) \\
\end{align*}

Goodman then introduces $r\times r$ Gram matrices for $\phi$ and $\psi$ as

\begin{align*}
  \Phi_{jk}(u) &= \sum_{n\in \mathbb{Z}} \hat{\phi}_j(u+2n\pi) \overline{\hat{\phi}_k(u+2n\pi)} \\
  \Psi_{jk}(u) &= \sum_{n\in \mathbb{Z}} \hat{\psi}_j(u+2n\pi) \overline{\hat{\psi}_k(u+2n\pi)}
\end{align*}

and the “dual-Gram” matrix

\begin{equation*}
  \Omega_{jk}(u) = \sum_{n\in \mathbb{Z}} \hat{\phi}_j(u+2n\pi) \overline{\hat{\psi}_k(u+2n\pi)}
\end{equation*}

$\Phi, \Psi$ are Hermitian matrices and so is $M = \begin{pmatrix} \Phi & \Omega \\ \Omega^* & \Psi \end{pmatrix}$.

\begin{prop}{2.1}
  The set $\{T_n\phi_j\}$ is a Riesz basis iif $\Phi$ is invertible, $\{T_n\psi_j\}$ is a Riesz-basis iif $\Psi$ is 
  invertible and $\{T_n\phi_j, T_n\psi_j\}$ is a Riesz basis iif $M$ is invertible.
\end{prop}

The author claims that $\Phi, \Psi, \Omega$ are invertible iif the eigenvalues are bounded away from zero. First this is 
not precise. It is invertibility everywhere or a.e? Same question for the boundedness away from 0? It is clear that 
latter implies the former, i.e eigenvalues bounded away from zero everywhere (resp a.e) implies invertibility everywhere 
(a.e). The converse implication is hard though. Invertibility everywhere (resp a.e) leads to eigenvalues strictly 
positive everywhere (resp a.e) from which we cannot deduce boundedness away from 0 everywhere (resp a.e) in general. In 
case the entries of the matrices are continuous functions of $u$, $\lambda_j(u)$ are also continuous and also
$2\pi$-periodic. Then strict positivity everywhere leads to boundedness away from 0 everywhere (continuous function on 
compact reaches its bounds) but the same does not apply to a.e case.  \\

The article Goodman refers to is actually just saying in the case where $\phi_j, \psi_j$ are regular, $\Phi, \Psi$ are 
invertible everywhere iif $\exists 0 < A \leq B$ such that $A \leq \lambda_j \leq B$ everywhere (equiv to a.e as 
eigenvalues are continuous here) which is clear. \\

Regarding lemma 2.1, we have
\begin{align*}
  \Phi(2u) &= \sum_{n \in \mathbb{Z}} \hat{\phi}(2u+2n\pi) {\hat{\phi}(2u+2n\pi)}^* \\
  &= \sum_{n \in \mathbb{Z}} P(u+n\pi) \hat{\phi}(u+n\pi) {\hat{\phi}(u+n\pi)}^* {P(u+n\pi)}^* \\
  &= P(u) \Phi(u) {P(u)}^* + P(u+\pi) \Phi(u+\pi) {P(u+\pi)}^*
\end{align*}

A similar expression holds for $\Psi(2u)$ with $P$ replaced  by $Q$ and for $\Omega(2u)$ with $P$ on the right only 
replaced by $Q$.

\begin{thm}{2.1}
  The following are equivalent
  \begin{enumerate}
    \item $\{T_n \phi_j, T_n \psi_j\}$ forms a Riesz basis for $V_1$
    \item $\begin{bmatrix} P(u) & P(u+\pi) \\ Q(u) & Q(u+\pi) \end{bmatrix}$ and $\Phi(u)$ are invertible.  
  \end{enumerate}
  If this holds, for any sequences of matrices $r\times r$ $(A_n), (B_n), (C_n)$ with FT $A(u), B(u), C(u)$ in 
  $\tilde{L}^2_{r\times r}(0, 2\pi)$, we have
  \begin{align}
    &\sum_{n \in \mathbb{Z}} C_n \phi(2x-n) = \sum_{n \in \mathbb{Z}} A_n \phi(x-n) + B_n \psi(x-n) \\
    & \iff \frac{1}{2} \begin{bmatrix} C(u) & C(u+2\pi) \end{bmatrix} = \begin{bmatrix} A(2u) & B(2u) 
    \end{bmatrix}\begin{bmatrix} P(u) & P(u+\pi) \\ Q(u) & Q(u+\pi) \end{bmatrix}
    \end{align}
  If the first equivalence holds, $\Phi(u)$ and $\Psi(u)$ are positive definite and have inverses in 
  $\tilde{\mathcal{C}}_{r\times r}(0, 2\pi)$.  Functions $\psi_j$ belongs to $W_0$ if and only if
  \begin{equation}
     P(u) \Phi(u) {Q(u)}^* + P(u+\pi) \Phi(u+\pi) {Q(u+\pi)}^* = 0
  \end{equation}
\end{thm}

We \textbf{now assume} that $\{T_n \phi_j, T_n \psi_j\}$ forms a Riesz basis for $V_1$. The author choose sequences 
$(G_n)$, $(H_n)$ so that for any integer $l$

\begin{equation}
  \phi(2x-l) = \sum_{n \in \mathbb{Z}} G_{2n-l} \phi(x-n) + \sum_{n \in \mathbb{Z}} H_{2n-l} \psi(x-n)
\end{equation}

I understand now where $2n$ subscript comes from. Even rank elements of $(G_n), (H_n)$ $(G^0, H^0)$, are used for 
$\phi(2x)$, odd rank $(G^1, H^1)$ are used for $\phi(2x-1)$ i.e

\begin{align}
  \phi(2x) &= \sum_{n \in \mathbb{Z}} G_{2n} \phi_n(x) + H_{2n} \psi_n(x) \label{G0} \\
  \phi(2x-1) &= \sum_{n \in \mathbb{Z}} G_{2n-1} \phi_n(x) + H_{2n-1} \psi_n(x) \label{G1}
\end{align}

Note that
\begin{align*}
  G(u) &= \sum_{n \in \mathbb{Z}} G_n e^{-iun} \\
  G(u) + G(u+\pi) &= 2\sum_{n \in \mathbb{Z}} G_{2n} e^{-2iun} \\
  &= 2G^0(2u) \\
  G(u)-G(u+\pi) &= 2\sum_{n \in \mathbb{Z}} G_{2n-1} e^{-2iun+iu} \\
  &= 2G^1(2u)e^{iu}
\end{align*}

and the same holds for $H$. Taking the Fourier transform of equations (\ref{G0}), (\ref{G1}) leads to
\begin{align}
  \frac{1}{2} \hat{\phi}(u) &= G^0(2u)P(u)\hat{\phi}(u) + H^0(2u)Q(u)\hat{\phi}(u) \\
  \frac{1}{2} \hat{\phi}(u) e^{-iu} &= G^1(2u)P(u)\hat{\phi}(u) + H^1(2u)Q(u)\hat{\phi}(u)
\end{align}

\begin{thm}{2.2}
  If $\{T_n \phi_j\}$, $\{T_n \psi_j\}$ are Riesz basis of $V_0$ and $W_0$ respectively, then $G$ and $H$ are given  by 
  \begin{align*}
    G(u) &= \Phi(u){P(u)}^*{\Phi(2u)}^{-1} \\
    H(u) &= \Phi(u){Q(u)}^*{\Psi(2u)}^{-1}
  \end{align*}
\end{thm}

\begin{proof}
  Goodman claims that equations of equivalence of theorem 2.1 holds but \textbf{why}? Because $V_1 = V_0 \bigoplus W_0$.
  The rest of the proof is simply assembling previous equations.
\end{proof}

\textbf{Assume} equivalence of theorem 2.1 holds. By the way \textbf{is it true that}

\begin{equation*}
  \{T_n \phi_j, T_n \psi_j\} \ \text{Riesz  basis for $V_1$} \ \iff \{T_n \phi_j\}, \{T_n \psi_j\} \ \text{Riesz  basis 
  for $V_0, W_0$ respectively?}
\end{equation*}

Let $\tilde{\phi}, \tilde{\psi}$ in $V_1\times\cdots\times V_1$ defined in the Fourier domain by
\begin{equation*}
  \hat{\tilde{\phi}} = \Phi^{-1} \hat{\phi}, \quad \hat{\tilde{\psi}} = \Psi^{-1} \hat{\psi}
\end{equation*}

Then $\langle \phi_j, T_n \tilde{\phi}_k\rangle = \delta_{j,k} \delta_{n,0}$ that is $\phi$ and $\tilde{\phi}$ are dual 
basis. Similarly $\tilde{\psi}$ is dual to $\psi$.  

\begin{thm}{2.3} If $\{T_n \phi_j\}$, $\{T_n \psi_j\}$ are Riesz basis of $V_0$ and $W_0$ respectively, dual functions 
  are such that
  \begin{align*}
    \tilde{\phi}(x) &= 2\sum_{n \in \mathbb{Z}} G_n^* \tilde{\phi}(2x-n) \\
    \tilde{\psi}(x) &= 2\sum_{n \in \mathbb{Z}} H_n^* \tilde{\psi}(2x-n)
  \end{align*}
  and
  \begin{equation}
    \tilde{\phi}(2x-l) = \sum_{n \in \mathbb{Z}} P_{2n-l}^* \tilde{\phi}(x-n) + \sum_{n \in \mathbb{Z}} Q_{2n-l}^* 
    \tilde{\psi}(x-n)
  \end{equation}
\end{thm}

\section{Spline wavelets of multiplicity $r$}

\underline{Notations}
\begin{itemize}
  \itemsep0em
  \item $\zeta_{n,r}(S)$ space of spline functions of degree $n$ on $\mathbb{R}$ with knots multiplicity $r$ on set $S$.  
    Note that $\zeta_{n, r}(\mathbb{Z}) = \$_{n+1, \mathbb{Z}_r}$ in De Boor's notation.
  \item $V_0 = \zeta_{n, r}(\mathbb{Z}) \cap L_2$, $V_1 = \zeta_{n, r}(\frac{1}{2}\mathbb{Z}) \cap L_2$.
  \item $W_0$ orthogonal complement to $V_0$ in $V_1$ i.e $V_1 = V_0 \bigoplus W_0$.
  \item ${(t_i)}_{i \in \mathbb{Z}} = \mathbb{Z}_r$ with $t_i = j$ for $jr \leq i \leq (j+1)r-1$.
  \item $N_i$ $i=0, \ldots, r-1$, B-spline in $\zeta_{n, r}(\mathbb{Z})$ with support in $[t_i, t_{i+n+1}]$ and knots at 
    $t_i, \ldots, t_{i+n+1}$ normalized so that $\sum_i N_i^n = 1$. This is exactly De Boor's B-splines i.e
    \begin{equation*}
      N_i^n(t) = (t_{i+n+1}-t_i) [t_i,  \ldots, t_{i+n+1}] {(\cdot - t)}_+^{n} \ \text{or eq.} \ N_i^n(t) = B_{i, n+1, 
      \mathbb{Z}_r}
    \end{equation*}
    From the work of De Boor (see $\cite{DeBoor}$ chapter XI), we know that ${\{N_i^n\}}_{i \in \mathbb{Z}}$ is a 
    Riesz-basis for the infinite norm as for any bounded sequence ${(c_i)}_{i \in \mathbb{Z}}$ we have
    \begin{equation*}
      D_n^{-1} \|c\|_{\infty} \leq \|\sum_{i \in \mathbb{Z}} c_i N_i^n \|_{\infty} \leq \|c\|_{\infty}
    \end{equation*}
    \textbf{Does this also hold for $l^2, L^2$ norms?} Be aware that in infinite dimension $\|\|_{\infty}$ and 
    $\|\|_{l^2}$ are not equivalent.
  \item $U = \{f \in \zeta_{2n+1,r}(\frac{1}{2}\mathbb{Z}), f^{(i)}_{|\mathbb{Z}} = 0, i=0, \ldots, r-1\}$. \textbf{Is 
    this a subset of $L_2$?} Following Schoenberg Hermite interpolation theorems, it is the case provided results extend 
    to half-integer knots. More specifically, let ${(y_{\nu})}_{\nu \in \frac{1}{2}\mathbb{Z}}, \ldots, 
    {(y^{(r-1)}_{\nu})}_{\nu \in \frac{1}{2}\mathbb{Z}}$ sequences on the half integers. Let ${(z^{(j)}_{\nu} = 
    y^{(j)}_{\frac{\nu}{2}})}_{\nu \in \mathbb{Z}}$ sequences on the integers. Provided $z^{(j)} \in l_p$ for $j=0,
    \ldots, r-1$, $\exists! L \in \zeta_{2m-1, r}(\mathbb{Z}) \cap \mathcal{L}_{p,r}$, $m \geq r$ such that $L$ 
    interpolates $z^{(j)}$ and it is given by
    \begin{equation*}
      \forall x, \quad L(x) = \sum_{\nu \in \mathbb{Z}} \sum_{j=0}^{r-1} z^{(j)}_{\nu} L_{j}(x-\nu)
    \end{equation*}

    Note now that $D_2L \in \zeta_{2m-1, r}(\frac{1}{2}\mathbb{Z})$ and so do $D_2L_j$, $j=0, \ldots, r-1$.  
    Consequently, 
    
    \begin{align*}
      D_2L(x) &= \sum_{\nu \in \mathbb{Z}} \sum_{j=0}^{r-1} z^{(j)}_{\nu} L_{j}(2x-\nu) \\
      &= \sum_{\nu \in \frac{1}{2}\mathbb{Z}} \sum_{j=0}^{r-1} z^{(j)}_{2\nu} L_{j}(2x-2\nu) \\
      &= \sum_{\nu \in \frac{1}{2}\mathbb{Z}} \sum_{j=0}^{r-1} y^{(j)}_{\nu} D_2L_{j}(x-\nu) \\
    \end{align*}

    The $r$ null sequences ${(0)}_{i \in \frac{1}{2}\mathbb{Z}}$ being obviously in $l^2$, there exists a unique $L \in 
    \zeta_{2m-1, r}(\frac{1}{2}\mathbb{Z}) \cap \mathcal{L}_{2,r}$ with $m \geq r$ that has vanishing $(r-1)$ first 
    derivatives on $\frac{1}{2}\mathbb{Z}$. $U$ \textbf{is a subset of $L_2$} (and $L_p$ for any $p \geq 1$) provided 
    $n+1 \geq r$ which is part of the assumptions.
\end{itemize}

\begin{thm}{3.1}
  For each $i \in \mathbb{Z}$, $\exists! \psi_i \in W_0$ with support on $[t_i,\ldots, t_{i+2n+2-r}]$ and integer knots 
  $t_i, \ldots, t_{i+2n+2-r}$. $\psi_i$ does not have smaller support nor a smaller set of knots.
\end{thm}

For that Goodman first constructs functions $\Psi_i$ in $U$.

\begin{lem}{3.1}
  For each $f \in W_0$ with $\supp f \subset [a,b]$, $\exists! g \in U$ with $\supp g \subset [a,b]$ such that 
  $g^{(n+1)} = f$. Conversely, if $g \in U$ has support in $[a,b]$ then $g^{(n+1)}$ is in $W_0$.
\end{lem}

\begin{proof} Goodman says the proof is easy using integration by parts but $\textbf{I don't see that}$.
\end{proof}

\begin{thm}{3.2}
  For each $i \in \mathbb{Z}$, $\exists! \Psi_i \in U$ with support on $[t_i,\ldots, t_{i+2n+2-r}]$ and integer knots 
  $t_i, \ldots, t_{i+2n+2-r}$. $\Psi_i$ does not have smaller support nor a smaller set of knots.
\end{thm}

Normalization is chosen so that $\Psi_{i+r} = \Psi_{i}(\cdot - 1), i \in \mathbb{Z}$ and $\psi_i = \Psi_i^{(n+1)}$ gives 
theorem 3.1.

\section{Linear combinations of wavelets}

\begin{thm}{4.1}
  The sequence ${(\Psi_i)}_{i \in \mathbb{Z}}$ is locally linearly independent on any interval.
\end{thm}

\begin{cor}{4.1}
  Any function $f \in U$ can be written uniquely in the form
  \begin{equation*}
    f = \sum_{i \in \mathbb{Z}} c_i \Psi_i
  \end{equation*}
  for some constants $c_i$. Moreover if support of $\Psi_i$ overlaps $(j, j+1)$ then
  \begin{equation*}
    |c_i| \leq K \| f_{|[j, j+1]} \|_{\infty}
  \end{equation*}
  with $K$ independent of $i,j$ and $f$.
\end{cor}

\begin{proof}
  Goodman says that it is easily seen that $\zeta = \zeta_{2n+1,r}{(\frac{1}{2}\mathbb{Z})}_{|[j,j+M]}$ has dimension 
  $2n+2+r(2M-1)$. I personally don't see this easily but following Curry-Schoenberg (see~\cite{DeBoor}, p97) theorem one 
  can notice that
  \begin{equation*}
    \zeta_{2n+1,r}{(\frac{1}{2}\mathbb{Z})}_{|[j,j+M]} = {\Pi_{<2n+2, \frac{1}{2}\mathbb{Z}, \bm{\nu}}}_{|[j,j+M]}
  \end{equation*}
  with $\nu_j = 2n+2-r$ for all $j \in \mathbb{Z}$. Now we have
  \begin{equation*}
    {\Pi_{<2n+2, \frac{1}{2}\mathbb{Z}, \bm{\nu}}}_{|[j,j+M]} = {\Pi_{<2n+2, \frac{1}{2}\mathbb{Z}\cap[j,j+M], 
    \bm{\nu}}}_{|[j,j+M]}
  \end{equation*}
  The latter is the space of piecewe polynomials of order $2n+2$ on the finite knot sequence 
  $\frac{1}{2}\mathbb{Z}\cap[j,j+M]$ of $2M+1$ elements each with multiplicity $r$. From (see~\cite{DeBoor}, p84), a
  basis of $2n+2 + r(2M-1)$ elements can be built for this space which is thus of dimension $2n+2 + r(2M-1)$. \\

  The dimension of $\zeta_0 = U_{|[j,j+M]} = \{f_{|[j,j+M]} \in \zeta_{|[j,j+M]} | f^{(j)}_{|\mathbb{Z}\cap[j,j+M]} = 0, 
  j=0,\ldots,r-1\}$ is $r(M+1)$ less than the dimension of $\zeta$ that is $2n+2+r(M-2)$. Functions $\Psi_i$ have 
  support overlapping $[j,j+M]$ for $(j+2)r - 2n - 2 \leq i \leq (j+M)r-1$ that is $2n+2+r(M-2)$ of them. Functions 
  ${\Psi_i}_{|[j,j+M]}$ for such $i$ are all in $\zeta_0$ and are linearly independent which makes 
  ${\{{\Psi_i}_{|[j,j+M]}\}}_{(j+2)r - 2n - 2 \leq i \leq (j+M)r-1}$ a basis for $\zeta_0$. \\

  Then \textbf{comes the magic}. Goodman deduces from the fact that ${\{{\Psi_i}_{|[j,j+1]}\}}_{(j+2)r - 2n - 2 \leq i 
  \leq (j+1)r-1}$ is a basis for $\zeta_0$ and from the fact that norms on finite dimension spaces are equivalent to get 
  that $\exists K > 0$ such that
  \begin{equation*}
    \forall f \in U, \quad \max_{(j+2)r - 2n - 2 \leq i \leq (j+1)r-1} |c_i| \leq K \|f_{|[j,j+1]}\|_{\infty}
  \end{equation*}
\end{proof}

\begin{thm}{4.2}
  The sequence ${(\psi_i)}_{i=-\infty}^{\infty}$ is locally linearly independent on any interval $(j, j+M)$ for any $j$ 
  and $M$ such that $r(M+1) \geq n+1$.
\end{thm}

\begin{cor}{4.2}
  For integers j, M with $M \geq 1$,
  \begin{equation}
    {W_0}_{|[j,j+M]} = \left\{f \in \zeta_{n,r}(\frac{1}{2}\mathbb{Z}) | \int_{-\infty}^{\infty} fg = 0, g \in 
    \zeta_{n,r}(\mathbb{Z}), \supp g \subset [j, j+M] \right\}
  \end{equation}
\end{cor}

\begin{cor}{4.3}
  Any function in $W_0$ can be uniquely written $\sum_{i \in \mathbb{Z}} c_i \psi_i$ for some constants $c_i$.
\end{cor}

\begin{cor}{4.4}
  The functions ${\{\psi_i\}}_{i=-\infty}^{\infty}$ form a Riesz basis for $W_0$.
\end{cor}

\begin{proof} Let $M$ such that $r(M+1) \geq n+1$. ${\psi_i}_{|[j,j+M]}$ for $l=(j+2)r-2n-2 \leq i \leq L=(j+M)r-1$ form 
  a basis for ${W_0}_{|[j,j+M]}$. Goodman \textbf{claims then there exists constants $A,B$ independent of $j$ such that 
  for any $f = \sum_{i \in \mathbb{Z}} c_i \psi_i$ in $W_0$ we have}
  \begin{equation*}
    A \int_{j}^{j+M} f^2 \leq \sum_{i=l}^L c_i^2 \leq B \int_{j}^{j+M} f^2
  \end{equation*}
  \textbf{Where does that come from}???
\end{proof}

\section{Construction of the wavelets}

$\Psi_i$ with support in $[0,T] = [t_i, t_{i+2n+2-r}]$ is an element of $\zeta_{2n+1,r}(\frac{1}{2}\mathbb{Z})$ and can 
be written
\begin{equation*}
  \Psi_i(x) = \sum_{j=0}^{(T-1)r} c_j N_{i+j}^{2n+1}(2x)
\end{equation*}

$(T-1)r+1$ coefficients $c_j$ are determined by the conditions $(T-1)r$ conditions $\Psi_i^{(j)}(k) = 0$ for $j=0, 
\ldots, r-1$, $k = 1, \ldots T-1$ and by a normalisation condition. \\

Consider the case $r = n+1$. Note then that
\begin{equation*}
  t_i = 0, \quad t_{i+2n+2-r} = t_{i+r} = 1, \quad \text{for} \ i=0, \ldots, r-1
\end{equation*}

i.e $T=1$. Then 

\begin{equation}
  \Psi_i(x) = N_i^{2n+1}(2x) \quad i=0, \ldots, n
\end{equation}

In De Boor's notation we have $N_j^{2n+1}(t) = B_{j, 2n+2, \mathbb{Z}_{n+1}}(t) = (t_{j+2n+2} - t_j)[t_j, \ldots, 
t_{j+2n+2}]{(\cdot - t)}_+^{2n+1}$. Be aware that the recurrence relationship of De Boor's B-splines does not transpose 
directly to the $N_j^n$ as varying $n$ implies varying the sequences of knots whereas in De Boor's relation the sequence 
of knots is assumed to be fixed. \\

Remember that $\psi$ is related to $\Psi$ by $\psi_i = \Psi_i^{(n+1)}$ and also $\Psi_{i+r} = \Psi_i(\cdot - 1)$.\\

\underline{Case $n=0, r=1$} \mbox{} \\
Then
\begin{align*}
  \Psi_0(x) &= N_0^1(2x) = \begin{dcases} 2x & 0 \leq x < \frac{1}{2} \\ 2-2x & \frac{1}{2}\leq x < 1 \end{dcases} \\
  \psi_0(x) &= \begin{dcases} 2 & 0 \leq x < \frac{1}{2} \\ -2 & \frac{1}{2} \leq x < 1 \end{dcases}
\end{align*}
 
However applying Goodman's relation (5.4) leads to $\psi_0(x) = N_0^0(2x) - N_0^0(2x-1) = \begin{dcases} 1 & 0 \leq x < 
\frac{1}{2} \\ -1 & \frac{1}{2} \leq x < 1 \end{dcases}$. A factor $2$ is missing.

\underline{Case $n=1, r=2$}

In that case we have $\Psi_j(x) = N_j^3(2x) = B_{j, 4, \mathbb{Z}_2}(2x)$, $j=0,1$. Let's compute the latter. Note $I_j 
= [j, j+1]$

\begin{align*}
  B_{2j, 1} &= 0 & B_{2j+1,1}(t) &= \begin{dcases} 1 & I_j \end{dcases} \\
  B_{2j, 2}  &= \begin{dcases} j+1-t & I_j \end{dcases} & B_{2j+1, 2}  &= \begin{dcases} t-j & I_j \end{dcases} \\
  B_{2j, 3} & = \begin{dcases} 2(j+1-t)(t-j) & I_j \end{dcases} & B_{2j+1, 3}  &= \begin{dcases} {(t-j)}^2 & I_j \\ 
  {(j+2-t)}^2 & I_{j+1} \end{dcases} \\
  B_{2j, 4} & = \begin{dcases} \frac{1}{2}(5(j+1-t)+1){(t-j)}^2 & I_j \\ \frac{1}{2}{(j+2-t)}^3 & I_{j+1} \end{dcases} & 
  B_{2j+1,4}  &= \begin{dcases} \frac{1}{2}{(t-j)}^3 & I_j \\ \frac{1}{2}(5(t-j-1)+1){(j+2-t)}^2 & I_{j+1} \end{dcases}
\end{align*}

Consequently,
\begin{align*}
  \Psi_0(x) &= \begin{dcases} 4(3-5x)x^2 & 0 \leq x < \frac{1}{2} \\ 4{(1-x)}^3 & \frac{1}{2}\leq x < 1 \end{dcases} \\
  \Psi_1(x) &= \begin{dcases} 4x^3 & 0 \leq x < \frac{1}{2} \\ 4(5x-2){(1-x)}^2 & \frac{1}{2}\leq x < 1 \end{dcases} \\
  \psi_0(x) &= \begin{dcases} 24(1-5x) & 0 \leq x < \frac{1}{2} \\ 24(1-x) & \frac{1}{2}\leq x < 1 \end{dcases} \\
  \psi_1(x) &= \begin{dcases} 24x & 0 \leq x < \frac{1}{2} \\ 24(5x-4) & \frac{1}{2}\leq x < 1 \end{dcases}
\end{align*}


Again relation (5.4) leads to $a_{0,0} = 2, a_{0,1} = -3, a_{1,0}=0, a_{1,1} = -1$ and 

\begin{align*}
  \psi_0(x) &= 2N_0^1(2x) - 3N_1^1(2x) - N_0^1(2x-1) = \begin{dcases} 2 - 10x & 0 \leq x < \frac{1}{2} \\ 2 - 2x & 
  \frac{1}{2} \leq x < 1 \end{dcases} \\
  \psi_1(x) &= -N_1^1(2x) + 3N_0^1(2x-1) - 2N_1^1(2x-1) = \begin{dcases} 2x & 0 \leq x < \frac{1}{2} \\ -10x + 8 & 
  \frac{1}{2} \leq x < 1 \end{dcases} \\
\end{align*}

A factor 12 or -12 is missing.

\section*{}
\nocite{*}
\bibliographystyle{unsrt}
\bibliography{xmpl}
\addcontentsline{toc}{section}{References}

\end{document}


