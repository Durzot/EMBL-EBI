\documentclass[a4paper, 11pt]{article}

\usepackage[left=1.5cm, right=1.5cm, top=2cm, bottom=2cm]{geometry}

\usepackage[utf8]{inputenc}
\usepackage[T1]{fontenc}
\usepackage[english]{babel}  
\usepackage{lmodern}

\usepackage{amsmath, amsthm, amssymb}
\usepackage{mathtools}
\usepackage{dsfont}
\usepackage{bm}

\newtheorem{innercustomgeneric}{\customgenericname}
\providecommand{\customgenericname}{}
\newcommand{\newcustomtheorem}[2]{%
  \newenvironment{#1}[1]
  {%
   \renewcommand\customgenericname{#2}%
   \renewcommand\theinnercustomgeneric{##1}%
   \innercustomgeneric}
  {\endinnercustomgeneric}
}

\newcustomtheorem{thm}{Theorem}
\newcustomtheorem{lem}{Lemma}
\newcustomtheorem{cor}{Corollary}
\newcustomtheorem{deftn}{Definition}
\newcustomtheorem{prop}{Proposition}
\newcustomtheorem{remark}{Remark}
\DeclareMathOperator*{\argmin}{argmin} 
\DeclareMathOperator*{\argmax}{argmax}
\DeclareMathOperator*{\Tr}{Tr}
\DeclareMathOperator*{\sinc}{sinc} 
\DeclareMathOperator*{\supp}{supp}
\DeclareMathOperator*{\essinf}{ess\ inf}
\DeclareMathOperator*{\esssup}{ess\ sup} 

\usepackage[backend=bibtex, citestyle=authoryear, sorting=nyt]{biblatex} % Use the bibtex backend 
\usepackage[autostyle=true]{csquotes} % Required to generate language-dependent quotes in the bibliography
\usepackage{filecontents}

\begin{filecontents}{xmpl.bib}
@article{Sch73,
  shorthand = {Sch73},
  author = {Schoenberg, I.J.},
  title = {Cardinal Spline Interpolation},
  year = {1973},
  journal = {CBMS-NSF Regional conference series in applied mathematics},
}

@article{LS76,
  shorthand = {LS76},
  author = {Lee, S.L. and Sharma, A.},
  title = {Cardinal Lacunary Interpolation by g-splines. I. The characteristic polynomials.},
  year = {1976},
  journal = {Journal of Approximation theory},
  volume = {16},
  pages = {85-96},
}
\end{filecontents}

\addbibresource{xmpl.bib} % The filename of the bibliography

\begin{document}
\title{Summary paper 44: Exponential Hermite-Euler splines}
\author{Yoann Pradat}
\maketitle

This paper by S.Lee extends Schoenberg's exponential splines to the C.H.I.P, defining so-called exponential 
Euler-Hermite splines. 

\section{Introduction}

\underline{Notations}
\begin{enumerate}
  \item $n, r$ positive integers with $n \geq 2r-1$
  \item $\mathcal{S}_{n,r}$ cardinal splines of degree $n$, multiplicity $r$
  \item $\displaystyle \Phi_n(x;t) = \sum_{k=-\infty}^{\infty} t^k Q_{n+1}(x-k)$ exponential Euler spline degree $n$ 
    base $t$.
  \item $\displaystyle \Pi_n(t) = n! \sum_{j=1}^n Q_n(j)t^{j-1}$ Euler-Frobenius polynomial. It has $n-1$ simples zeros
    \begin{equation*}
      \lambda_{n-1} < \cdots < \lambda_1 < 0
    \end{equation*}
  \item $\displaystyle \Pi_{n,r}(t) = {(-1)}^{m(r-1)}|\Delta_{r,d}(t)|$ Euler-Frobenius polynomial for multiplicity $r$.  
    It has $2m-2r$ simples zeros
    \begin{equation*}
      \lambda_{2m-2r} < \cdots < \lambda_1 < 0
    \end{equation*}
  \item $A_n(x;t) = \frac{n!}{{(1-t^{-1})}^n} \Phi_n(x;t), \ 0\leq x\leq 1$ exponential Euler polynomial degree $n$.
    \begin{align*}
      \frac{t-1}{t-e^z}e^{xz} &= \sum_{n=0}^{\infty} \frac{A_n(x;t)}{n!}z^n \\
      A_n(x;t) &= x^n + a_1(t) \binom{n}{1} + \cdots + a_n(t) \\
      a_n(t) &= \frac{\Pi_n(t)}{{(t-1)}^n} 
    \end{align*}
  \item $\Phi_n(0;t) = \frac{{(1-t^{-1})}^n}{n!} A_n(0;t) = \frac{\Pi_n(t)}{n!}$
  \item $S_n(x;\lambda) = \frac{\Phi_n(x;\lambda)}{\Phi_n(0;\lambda)}$ for $\lambda \not\in\{\lambda_1,\ldots, 
    \lambda_{n-1}\}$ exponential Euler spline.
  \item $S_j(x) = \Phi_n(x, \lambda_j)$ for $j=1, \ldots, n-1$, eigensplines. 
    \begin{equation*}
      \forall k \in \mathbb{Z}, S_j(k) = 0 
    \end{equation*}
\end{enumerate}

As $S_n(x+1, \lambda) = \lambda S_n(x;\lambda)$ and $S_n(0;\lambda) = 1$, $S_n(\cdot, \lambda)$ interpolates 
$\lambda^{\cdot}$ at integers. As an extensions, Lee proposes to define $S_{n,r}(\cdot, \lambda) \in \mathcal{S}_{n,r}$ 
that interpolates $\lambda^{\cdot}$ up to $r-1$ at integers i.e $S_{n,r}^{(\rho)}(k) = \lambda^k {(\log 
\lambda)}^{\rho}$.

\section{The polynomial $A_{n,r,s}(x;\lambda)$}

Let $s=0, \ldots, r-1$. Define
\begin{equation}
  A_{n,r,s}(x;\lambda) = \begin{vmatrix}
    \frac{A_n(0;\lambda)}{n!} & \hdots & \frac{A_{n-s+1}}{(n-s+1)!} & \frac{A_n(x;\lambda)}{n!} & \hdots & 
    \frac{A_{n-r+1}(0;\lambda)}{(n-r+1)!} \\
    \vdots & & & \vdots & & \vdots \\
    \frac{A_{n-r+1}(0;\lambda)}{(n-r+1)!} & \hdots & \frac{A_{n-r-s+2}}{(n-r-s+2)!} & \frac{A_n(x;\lambda)}{(n-r+1)!} & 
    \hdots & \frac{A_{n-2r+2}(0;\lambda)}{(n-2r+2)!} \\
  \end{vmatrix}
\end{equation}

As $\frac{A_n'(x;\lambda)}{n!} = \frac{A_{n-1}(x;\lambda)}{(n-1)!}$, one has that 
\begin{equation*}
  A^{(s)}_{n,r,s}(0; \lambda) = H_r(\frac{A_n(0;\lambda)}{n!})
\end{equation*}
with $H_r$ Hankel determinant of order $r$.

From (\cite{LS76}, Theorem 4), the following holds
\begin{equation*}
  H_r(\frac{\Pi_n(\lambda)}{n!}) = {(-1)}^{[\frac{r}{2}]} C_{n,r} \Pi_{n,r}(\lambda)
\end{equation*}

\section{Exponential Euler-Hermite splines}

The functions $S_{n,r,s}(x;\lambda)$ defined by 
\begin{align*}
  S_{n,r,s}(x;\lambda) &= \frac{A_{n,r,s}(x; \lambda)}{H_r(\frac{A_n(0;\lambda)}{n!})}, \quad 0 \leq x \leq 1 \\
  S_{n,r,s}(x+1;\lambda) &= \lambda S_{n,r,s}(x;\lambda), \quad \forall x
\end{align*}

is such that
\begin{align*}
  S_{n,r,s}^{(\rho)}(k;\lambda)  &= 0, \quad \rho=0, \ldots, r-1, \rho \neq s \\
  S_{n,r,s}^{(s)}(k, \lambda) &= \lambda^k
\end{align*}

so that $S_{n,r,s} \in \mathcal{S}_{n,r}^{(s)}$. 

\begin{thm}{2.1}
  The spline function 
  \begin{equation*}
    S_{n,r}(x;\lambda) = \sum_{s=0}^{r-1} {(\log \lambda)}^s S_{n,r,s}(x;\lambda)
  \end{equation*}
  interpolates $\lambda^{\cdot}$ up to $r-1$ at integers and belongs to $S_{n,r}$.
\end{thm}

Here is the representation of exponential Euler-Hermite splines in terms of B-splines.
\begin{thm}{3.1}
  The exponential Euler-Hermite spline $S_{2m-1, r, s}(x;\lambda)$ is represented in the Hermite B-splines basis as
  \begin{equation*}
    S_{2m-1,r,s}(x;\lambda) = \frac{1}{\Pi_{2m-1,r}(\lambda)} \sum_{k=-\infty}^{\infty} \lambda^k N_s(x+m-r-k)
  \end{equation*}
\end{thm}


% Include all documents in the bibliography
\nocite{*}
\printbibliography%

\end{document}


