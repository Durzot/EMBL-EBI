\documentclass[a4paper, 11pt]{article}

\usepackage[left=1.5cm, right=1.5cm, top=2cm, bottom=2cm]{geometry}

\usepackage[utf8]{inputenc}
\usepackage[T1]{fontenc}
\usepackage[english]{babel}  
\usepackage{lmodern}

\usepackage{amsmath, amsthm, amssymb}
\usepackage{mathtools}

\newtheorem{innercustomgeneric}{\customgenericname}
\providecommand{\customgenericname}{}
\newcommand{\newcustomtheorem}[2]{%
  \newenvironment{#1}[1]
  {%
   \renewcommand\customgenericname{#2}%
   \renewcommand\theinnercustomgeneric{##1}%
   \innercustomgeneric}
  {\endinnercustomgeneric}
}

\newcustomtheorem{thm}{Theorem}
\newcustomtheorem{lem}{Lemma}
\newcustomtheorem{cor}{Corollary}
\DeclareMathOperator*{\argmin}{argmin} 
\DeclareMathOperator*{\argmax}{argmax} 

\begin{document}
\title{Summary paper 2: Hermite Snakes with Control of Tangents Uhlmann}
\author{Yoann Pradat}
\maketitle

In the problem of finding a parametric contour in a continuous fashion, Hermite snakes are more efficient than other 
active contour snake-based techniques. In order to be a good, a model needs to be invariant to affine transformations 
and to have good approximation properties. \\

In this paper, an interpolation scheme is proposed with 2 classes of basis functions compared to 1 class of basis 
functions in other schemes. As we will see, the second type of basis functions allows for sharp corners and tips without 
increasing the number of control points and allows to introduce directional energies that improves the contouring 
results. \\

[18] gives a good review of parametric snakes, check it out when possible. \\

The contour is represented in 2D by $r(t) = (x(t), y(t))$ and $\Phi = (\phi_1, \phi_2)$ are the 2 basis functions we 
will consider. Both open and closed curves can be modeled as follows

\begin{enumerate}
  \item{\underline{Open curve}} For an open curve specified by $M$ anchor points, let $r[k] = r(\frac{k}{M-1}), r'[k] = 
    r'(\frac{k}{M-1})$ then
    \begin{equation}
      \forall t \in [0, 1] \quad r(t) = \sum_{k=0}^{M-1} r[k]\phi_1((M-1)t - k) + r'[k]\phi_2((M-1)t - k)
    \end{equation}

  \item{\underline{Closed curve}} For an closed curve specified by $M$ anchor points, let $r[k] = r(\frac{k}{M}), r'[k] 
    = r'(\frac{k}{M})$ then
    \begin{equation}
      \forall t \in [0, 1] \quad r(t) = \sum_{k=0}^{M-1} r[k]\phi_{1, per}(Mt - k) + r'[k]\phi_{2, per}(Mt - k)
    \end{equation}
\end{enumerate}

The basis functions chosen are the splines $\phi_1 = L_{4, 2, 0}$ and $\phi_2 = L_{4, 2, 1}$ as per defined in the 
Schoenberg paper. Recall the expression of these functions
\begin{equation}
    \phi_1(t) = \begin{dcases}
                1-3|t|^2 + 2|t|^3 &\text{for } 0 \leq |t| \leq 1\\
                0 &\text{for } 1 < |t|\\
            \end{dcases} \quad
    \hfill
    \phi_2(t) = \begin{dcases}
               t(|t|^2-2|t|+1) &\text{for } 0 \leq |t| \leq 1\\
                0 &\text{for } 1 < |t|\\
              \end{dcases}
\end{equation}

These functions belong to $S^0_{4,2} \cap \mathcal{L}_{1,2}$. They are thus splines of degree 3 with knots at integers 
of order 2 and are in $\mathcal{C}^1$ ($2m-r-1 = 1$). Note that they satisfy
\begin{equation*}
  \forall k \in \mathbb{Z} \quad \phi_1(k) = \delta_k \quad \phi_1^{(1)}(k) = 0 \quad \phi_2(k) = 0 \quad 
  \phi_2^{(1)}(k) = \delta_k
\end{equation*}

It is precisely because these functions are in $\mathcal{C}^1$ and not $\mathcal{C}^2$ that sharp corners may arise! \\

The span of functions we will represent with our Hermite interpolation scheme is
\begin{equation*}
  \left\{s(t) = \sum_{k \in \mathbb{Z}} (s[k], s'[k]) \Phi(t-k) \ | \ s[k], s'[k] \in l_2(\mathbb{Z}) \right\}
\end{equation*}

Note that $s$ above are in $\mathcal{L}_{2,2}$ i.e $s$ and $s^{(1)}$ are in $\mathcal{L}_2(\mathbb{R})$. In the Fourier 
domain the basis functions are expressed by
\begin{align}
  \hat{\phi}_1(w) &= \int_{\mathbb{R}} \phi_1(t) e^{-iwt} dt = \frac{-12}{w^4}(w\sin(w) + 2\cos(w) - 2) \\
  \hat{\phi}_2(w) &= \int_{\mathbb{R}} \phi_2(t) e^{-iwt} dt = \frac{-4i}{w^4}(w\cos(w) - 3\sin(w) + 2w) \\
\end{align}

that is $\hat{\Phi}(w) = \hat{R}(w) \hat{\rho}(w)$ or $\hat{\rho}(w) = \hat{S}(w) \hat{\Phi}(w)$ with $\hat{S}$ the 
inverse of $\hat{R}$. Decomposing we have
\begin{align}
  \hat{\rho_1} &= \hat{s_1} \hat{\phi_1} + \hat{s_2} \hat{\phi_2} \\
  \rho_1 &= s_1 * \phi_1 + s_2 * \phi_2 \\
\end{align}

Note that the convolution are mixed i.e $s_1 : \mathbb{Z} \to \mathbb{C}$ while $\phi_1 : \mathbb{R} \to \mathbb{C}$.  
This is simply defined as $s_1 * \phi_1 (t) = \sum_{k \in \mathbb{Z}} s_1[k] \phi_1(t-k)$ and the property of Fourier 
transform over a convolution product holds. Note also that $\hat{s}_1 : \mathbb{R} \to \mathbb{C}$ is $2\pi$-periodic 
and that 
\begin{align*}
  \hat{s}_1(w) &= \sum_{k \in \mathbb{Z}} s_1[k] e^{-iwk} \\
  s_1[k] &= \frac{1}{2\pi}\int_{-\pi}^{\pi} \hat{s}_1(w) e^{iwk} dw
\end{align*}

In the end we can correctly write $\rho = S * \Phi$ with $\rho_1$ and $\rho_2$ being the Green's functions of operators 
$D^4$ and $D^3$ respectively. As the cubic splines are generated by ${(\rho_1(.-k))}_{k \in \mathbb{Z}}$ and the 
quadratic are generated by ${(\rho_2(.-k))}_{k \in \mathbb{Z}}$ \underline{cubic and quadratic splines are included in 
cubic Hermite splines}. 

In the same fashion, cubic B-splines $\beta^3_+ = \Delta^4_+ \rho_{D^4}$ and quadratic B-splines $\beta^2_+ = \Delta^3_+ 
\rho_{D^3}$ are readily expressed as finite combinations of $\phi_1$ and $\phi_2$ which yields that all polynomials of 
degree 3 are in the span of cubic Hermite splines. \\

Uniqueness of the parametric curve defined by its anchor points and control tangents and numerical stability of the 
interpolation process are ensured by the Riesz-basis condition for $\Phi$ i.e there exist $0 \leq A, B < \infty$ such 
that
\begin{equation}
  A \|a\|_{l_2} \leq \| \sum_{k \in \mathbb{Z}} {a[k]}^T \Phi(.-k)\|_{L_2} \leq B \|a\|_{l_2}
\end{equation}

for all $a[k] = (s[k], s'[k])$ with $s[k], s'[k] \in {l_2}(\mathbb{Z})$. This is proved in the Fourier domain. 

\paragraph{Approximation error} \mbox{} \\

As cubic splines are also cubic Hermite splines, the approximation error it as least as good as that of cubic splines.  
For the latter see theorem 1 from [27] for some asymptotic approximation order. 

\begin{thm}{1}
  Let $s[k], s'[k] \in l_2(\mathbb{Z})$. Among all functions that satisfy $\forall k \ f(k) = s[k], f'(k) = s'[k]$ and 
  $f, f', f'' \in L_2$, the one that minimizes $\|f''\|_{L_2}$ is given by
  \begin{equation}
    \forall t \quad f(t) = \sum_{k \in \mathbb{Z}} s[k] \phi_1(t-k) + s'[k] \phi_2(t-k)
  \end{equation}
\end{thm}

$\|f''\|_{L_2}$ is a good approximation of the curvature of the contour. Cubic Hermite splines appear thus as 
interpolating schemes that minimize the curvature. It therefore holds the potential for eliminating an explicit internal 
energy term. \\ 

Let's relate now the Hermite basis functions to the Bernstein polynomials which are themselves basis functions for 
Bézier curves. For $t \in \verb+[+n, n+1\verb+)+$, we have
\begin{equation}
  \forall 0 \leq u < 1 \quad s(n+u) = s[n] \phi_1(u) + s'[n] \phi_2(u) +  s[n+1] \phi_1(u-1) + s'[n]+1 \phi_2(u-1)
\end{equation}

Since $s(n+.)$ is a cubic polynomial we may express it using the Bernstein polynomials
\begin{equation}
  b_{i, 3}(u) = \binom{3}{i} u^i {(1-u)}^{3-i}
\end{equation}

i.e 

\begin{equation}
  \forall 0 \leq u < 1 \quad s(n+u) = p_0 b_{0,3}(u) + \cdots + p_3 b_{3,3}(u) 
\end{equation}

with control polygon
\begin{align*}
  p_0 &= r[n], &p_1= r[n] + \frac{1}{3}r'[n] \\
  p_2 &= r[n+1] - \frac{1}{3}r'[n+1],  &p_3= r[n+1]\\
\end{align*}

The Hermite snake can therefore be easily converted to a Bézier curve while the converse is not true.

\paragraph{Applicative aspects} \mbox{} \\

Suppose we consider open curve with 3 anchor points. If you choose $\forall i=0, 1,2 \ x'(i) = y'(i) = 0$ there is sharp 
corner at knot 1. If additionally you set $(x'(1), y'(1)) \neq 0$, the curve at the knot is smooth. Also, if you set 
$i=0, 2 \ (x'(i), y'(i)) \neq 0$, $(x'(1), y'(1))=0$ the curve has a roundish corner in 1, something you cannot create 
with other active contour snakes. \\

Given a snake curve $r:[0,1] \to S$, we define a directional energy term that we will optimize as
\begin{equation}
    E_{\text{directional}} = - \frac{1}{L} \int_{S} \left| <\theta, \frac{r'}{\|r'\|}>\right| \rho(r)dr
\end{equation}

where $L$ is the length of the snake and $\theta$ and $\rho$ are orientation and amplitude information. \\

\underline{Choice of $\theta, \rho$}
\begin{enumerate}
  \item{Image gradient}
    \begin{align}
      \theta(X) &= \arctan\left(\frac{\frac{\partial f}{\partial y}(X)}{\frac{\partial f}{\partial x}(X)} \right) \\
      \rho(X) &= \sqrt{{\frac{\partial f}{\partial x}(X)}^2 + {\frac{\partial f}{\partial y}(X)}^2}
    \end{align}
  \item{Steerable filter}
    \begin{align}
      \theta(X) &= \argmin_{v} (f * h(R_v.))(X) \\
      \rho(X) &= (f * h(R_{\theta}.))(X)
    \end{align}
\end{enumerate}

with the detection template $\displaystyle h(X) = \sum_{k=0}^N \sum_{i=0}^k a_{k,i} \frac{\partial^{k-i}}{\partial 
x^{k-i}} \frac{\partial^{i}}{\partial ^{i}} g(X)$ and $g$ is a Gaussian window. The features detected by $h$ can be 
modified by acting on $N$, odd $N$ yields edge detectors while even $N$ detects ridges.  

\end{document}




