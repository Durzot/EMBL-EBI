\documentclass[a4paper, 11pt]{article}

\usepackage[left=1.5cm, right=1.5cm, top=2cm, bottom=2cm]{geometry}

\usepackage[utf8]{inputenc}
\usepackage[T1]{fontenc}
\usepackage[english]{babel}  
\usepackage{lmodern}

\usepackage{amsmath, amsthm, amssymb}
\usepackage{mathtools}

\newtheorem{innercustomgeneric}{\customgenericname}
\providecommand{\customgenericname}{}
\newcommand{\newcustomtheorem}[2]{%
  \newenvironment{#1}[1]
  {%
   \renewcommand\customgenericname{#2}%
   \renewcommand\theinnercustomgeneric{##1}%
   \innercustomgeneric}
  {\endinnercustomgeneric}
}

\newcustomtheorem{thm}{Theorem}
\newcustomtheorem{lem}{Lemma}
\newcustomtheorem{cor}{Corollary}
\newcustomtheorem{deftn}{Definition}
\newcustomtheorem{prop}{Proposition}
\newcustomtheorem{remark}{Remark}
\DeclareMathOperator*{\argmin}{argmin} 
\DeclareMathOperator*{\argmax}{argmax}
\DeclareMathOperator*{\Tr}{Tr}
\DeclareMathOperator*{\sinc}{sinc} 

\begin{document}
\title{Summary paper 25: Approximation errors for quasi-interpolators and multi-wavelets expansions}
\author{Yoann Pradat}
\maketitle

This papers is concerned with providing approximation properties of general polynomial preserving operators that 
approximate a function into some subspace of $L_2$. The approximation error is estimated as a function of the scaling 
parameter $T$. The main result of this paper is theorem 1 that decomposed the error into an explicit term and a residual 
term that can be bounded by a given order of $T$. \\

For the analysis, authors chose a very broad class of linear approximation operators by imposing weak conditions on the 
synthesis and analysis functions. The paper treats the case of multi-generators $q > 1$ and therefore includes Hermite 
polynomial interpolation. 

\paragraph{Fourier transform remarks} \mbox{} \\

As later calculations use Poisson-formula, let's prove and give all its variants. For that let $f \in L_1(\mathbb{R})$.  
The Fourier transform is, in frequency and pulsation notation,

\begin{equation*}
  \hat{f}(w) = \int_{\mathbb{R}} f(x) e^{-jwx}dx \qquad \tilde{f}(\nu) = \int_{\mathbb{R}} f(x) e^{-2j\pi \nu x}dx
\end{equation*}

Note that both transformations are linked by $\tilde{f}(\nu) = \hat{f}(2\pi \nu)$. Given that relation, one can derive 
properties on one transformation using that of the other transformation. For instance, we have

\begin{align*}
  \hat{\tilde{f}}(\nu) &= \widehat{\hat{f}(2\pi \nu)} = \frac{1}{2\pi} \hat{\hat{f}}(\frac{\nu}{2\pi}) \\
  \text{therefore}, \quad \tilde{\tilde{f}}(\nu) &= \hat{\tilde{f}}(2\pi\nu) = \frac{1}{2\pi} \hat{\hat{f}}(\nu) \\
  \text{leading to}, \quad \bar{\tilde{\tilde{f}}}(\nu) &= f(\nu)
\end{align*}


The Poisson summation formula is as follows
\begin{lem}{Poisson}
  \begin{enumerate}
    \item Suppose $f \in L_1$ and $\hat{f} \in L_1$, then
      \begin{equation*}
      \sum_{k \in \mathbb{Z}} \hat{f}(w+2k\pi) = \sum_{k \in \mathbb{Z}} f(k) e^{-jwk} \qquad i.e \qquad \sum_{k \in 
      \mathbb{Z}} \hat{f}(\nu+k) = \sum_{k \in \mathbb{Z}} f(k) e^{-2j\pi \nu k}
      \end{equation*}
    \item Suppose $f \in L_1$ and $\hat{f} \in L_1$, then
      \begin{equation*}
        \sum_{k \in \mathbb{Z}} 2\pi f(-w+2k\pi) = \sum_{k \in \mathbb{Z}} \hat{f}(k) e^{-jwk} \qquad i.e \qquad \sum_{k 
        \in \mathbb{Z}} f(-\nu+k) = \sum_{k \in \mathbb{Z}} \tilde{f}(k) e^{-2j\pi \nu k}
      \end{equation*}
  \end{enumerate}
\end{lem}

\begin{remark}{1} Defining the Fourier transform with a reversed sign in the exponential inside the integral leads to 
  following adapted formulas of 2\@. in the lemma above
  \begin{equation*}
    \sum_{k \in \mathbb{Z}} 2\pi f(w+2k\pi) = \sum_{k \in \mathbb{Z}} \hat{f}(k) e^{jwk} \qquad i.e \qquad \sum_{k \in 
    \mathbb{Z}} f(\nu+k) = \sum_{k \in \mathbb{Z}} \tilde{f}(k) e^{2j\pi \nu k}
  \end{equation*}

  The second one is exactly the version given in the article at the end of page 3.
\end{remark}

In accordance with the notation adopted in the article, Fourier transform is defined with a reversed sign. In pulsation 
and frequency notations, Parseval's formula is 

\begin{lem}{Parseval}
  \begin{equation*}
    \forall f, g \in L_2, \quad \int_{\mathbb{R}} f\bar{g} = 2\pi \int_{\mathbb{R}} \hat{f} \bar{\hat{g}} = 
    \int_{\mathbb{R}} \tilde{f} \bar{\tilde{g}}
  \end{equation*}
\end{lem}

For any integer $r$, $\hat{f^{(r)}}(w) = {(-jw)}^r \hat{f}(w)$ or equivalently $\tilde{f^{(r)}}(\nu) = {(-2j\pi \nu)}^r 
\tilde{f}(\nu)$ and therefore ${\|\hat{f^{(r)}}\|}_{L_2}^2 = {\|{(-jw)}^r\hat{f}\|}_{L_2}^2 = {\|{(-2j\pi 
\nu)}^r\tilde{f}\|}_{L_2}^2$. Taking $r$ not integer allows us to define \textbf{fractional derivatives}. Now we define 
the Sobolev space $W^r_2$ as the collection of functions satisfying $\int {(1+\nu^2)}^r |\tilde{f}(\nu)|^2 d\nu < 
\infty$. For such a function, $\int {(1+\nu^2)}^s |\tilde{f}(\nu)|^2 d\nu < \infty$ also holds true for any $0 \leq s 
\leq r$ which means that all fractional derivatives between 0 and $r$ have a finite $L_2$ norm. \\ 


\paragraph{Approximation by multi-wavelets} \mbox{} \\

The problem of describing a continuous signal by a discrete sequence of numbers was first solved in the case of 
bandlimited functions, giving rise to the well-known Shannon's sampling theorem. In that case, the interpolated signal 
is in the space generated by sinc basis that is $V_T = \text{span}{\{ \sinc(\frac{.}{T}-k) \}}_{k \in \mathbb{Z}}$ 
restricted to $L_2$. This space has the property of general shift-invariance and general scale invariance that is for 
any $f \in V_T$, $f(.+\tau)$ and $f(a^{-1}.)$ are also in $V_T$ for any real $\tau$ and $a$. According to some papers it 
is more useful and robust to consider spaces where the shift-invariance is limited to integer translations and where 
scale-invariance is limited to integer exponents of some integer $a_0 \leq 2$. \\

Authors consider the multiple generator case, namely $q$ of them, and the subspace they generate
\begin{equation*}
  V_T = \text{span}{\{ \varphi_i(\frac{.}{T}-kq) \}}_{i=1,\dots, q; k \in \mathbb{Z}} \cap L_2
\end{equation*}

to be $q$-shift-invariant and $a_0$-scale-invariant. As mathematics are the same for single or multiple generators, 
authors do not distinguish by using the notation $\varphi$ for the q-vector ${(\varphi_1, \dots, \varphi_q)}^t$. \\

It is possible to orthormalize the generators through matrix filtering by the $q\times q$ matrix $\hat{A}(\nu)$ defined 
by

\begin{equation*}
  {A(\nu)}_{ij}  = \frac{1}{q} \sum_{k \in \mathbb{Z}} 
  \hat{\varphi_i}(\frac{\nu+k}{q})\overline{\hat{\varphi_j}(\frac{\nu+k}{q})} = \sum_{k \in \mathbb{Z}} \langle \varphi_i, 
  \varphi_j(.-kq) \rangle e^{-2j\pi \nu k}
\end{equation*}

\begin{remark}{2}
  \begin{enumerate}
    \item To note that the above is true, notice that $\langle \varphi_i, \varphi_j(.-kq) \rangle = \varphi_i * \check{\varphi_j}
      (kq)$. Using Parseval's theorem and basic properties of Fourier transform give the equality.
    \item The authors don't name but this is exactly the Gram matrix. As a reminder, we are interested in the Gram 
    matrix in the case of Hermite interpolation as proving that it's minimum and maximum eigenvalues are essentially 
    bounded by strictly positive constants would prove that the Hermite generators are a Riesz-Schauder basis.
  \end{enumerate}
\end{remark}

The Fourier transform of the \textbf{orthonormal} generators and their associated analysis functions are given by 
\begin{align}
  \hat{\phi}(\nu)  &= \sqrt{A(\nu)}^{-1} \hat{\varphi}(\nu) \\
  \hat{\tilde{\phi}}(\nu)  &= \sqrt{A(\nu)} \hat{\tilde{\varphi}}(\nu)
\end{align}

The approximation operator authors study takes the form
\begin{equation}
  \mathcal{I}_T(f) = \sum_{k} \mathcal{S}_T (f) \varphi_k(\frac{t}{T})
\end{equation}

with $\mathcal{S}_T (f) = \int f(\tau) \tilde{\varphi}_k(\frac{\tau}{T}) d\frac{\tau}{T}$ the linear operator that 
associates to the function its sequence of coefficients in the representation by $\varphi_k$. \\

In here the synthesis functions $\{\varphi_k\}$ and the sampling functions $\{\tilde{\varphi_k}\}$ are independent 
parameters.  This explains the notion of quasi-biorthonormality introduced by the authors

\begin{deftn}{1}
    Synthesis functions $\{\varphi_k\}$ and sampling functions $\{\tilde{\varphi_k}\}$ are quasi-biorthormal if
    \begin{itemize}
      \item Functions $\varphi_k$ are of order $L$.
      \item Distributions $\tilde{\varphi}_k$ satisfy
        \begin{equation*}
          \forall s=0,\dots, L-1, \forall k, \quad \int x^s \tilde{\varphi}_k(x)dx = \lambda_k^{(s)}
        \end{equation*}
    \end{itemize}
\end{deftn}

All the results that are soon to be given are dependent on the following hypotheses

\begin{enumerate}
  \item $\{\varphi_k\}$ are in $L_2$ and form a Riesz-basis.
  \item $\mathcal{M} = \{ \{\lambda_k\} | \sum_k \lambda_k \varphi_k = 0 \}$ is finite dimensional.
  \item $\varphi_k$ are of order $L$ and $\forall k=0, \dots, q-1, \forall l=0,\dots, L$, $\int |x-k|^l 
    |\varphi_k(x)|dx$ is finite.
  \item $\tilde{\varphi}_k$ have a bounded Fourier transform.
  \item $f \in W^r_2$ (as this implies $\mathcal{S}_T(f)$ is in $l_2$)
\end{enumerate}

\end{document}
