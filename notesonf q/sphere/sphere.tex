\documentclass[a4paper, 11pt]{article}

\usepackage[left=1.5cm, right=1.5cm, top=2cm, bottom=2cm]{geometry}

\usepackage[utf8]{inputenc}
\usepackage[T1]{fontenc}
\usepackage[english]{babel}  
\usepackage{lmodern}

\usepackage{amsmath, amsthm, amssymb}
\usepackage{mathtools}
\usepackage{booktabs}
\usepackage{stmaryrd}

\usepackage{tikz}
\newcommand{\boxalign}[2][0.97\textwidth]
{\par\noindent\tikzstyle{mybox} = [draw=black,inner sep=6pt]
  \begin{center}\begin{tikzpicture}
   \node [mybox] (box){%
    \begin{minipage}{#1}{\vspace{-5mm}#2}\end{minipage}
   };
  \end{tikzpicture}
  \end{center}}

\newtheorem{innercustomgeneric}{\customgenericname}
\providecommand{\customgenericname}{}
\newcommand{\newcustomtheorem}[2]{%
  \newenvironment{#1}[1]
  {%
   \renewcommand\customgenericname{#2}%
   \renewcommand\theinnercustomgeneric{##1}%
   \innercustomgeneric}
  {\endinnercustomgeneric}
}

\newcustomtheorem{thm}{Theorem}
\newcustomtheorem{lem}{Lemma}
\newcustomtheorem{cor}{Corollary}
\newcustomtheorem{deftn}{Definition}
\newcustomtheorem{prop}{Proposition}
\DeclareMathOperator*{\argmin}{argmin} 
\DeclareMathOperator*{\argmax}{argmax} 

\begin{document}
\title{Parametrisation of the sphere}
\author{Yoann Pradat}
\maketitle

\section{Cardinal Hermite exponential splines}

\subsection{The parametric model}

Conti et al's paper \underline{Ellipse-preserving interpolation and subdivision scheme} introduces two basis functions 
from the space $\mathcal{E}_4 = <1, x, e^{-iw_1x}, e^{iw_1x}>$ where $w = \frac{2\pi}{M}$ to reproduce closed curves 
with $M$ control points. The corresponding parametric representation is 

\begin{equation}
  r(t) = \sum_{k \in \mathbb{Z}} r(k) \phi_{1,w}(t-k) + r'(k) \phi_{2, w}(t-k)
\end{equation}

with $r$ and $r'$ assumed to be $M$-periodic. \\

The basis functions are \textbf{cycloidal splines} (Exponential splines? Exponential B-splines?) given by

\begin{equation}
  \phi_{1, w}(x) =
  \begin{dcases}
    g_{1, w}(x) &\text{for } x \geq 0 \\
    g_{1, w}(-x) &\text{for } x < 0\\
  \end{dcases} \quad
  \hfill
  \phi_{2, w}(x) =
  \begin{dcases}
    g_{2, w}(x) &\text{for } x \geq 0 \\
    -g_{2, w}(-x) &\text{for } x < 0\\
  \end{dcases} \quad
\end{equation}


The resulting parametric model has the following properties
\begin{enumerate}
  \item Unique and stable representation (${\{{\bf \phi_w(.-k)} = (\phi_{1,w}(.-k), \phi_{2,w}(.-k))\}}_{k}$ Riesz 
    basis)
  \item Affine invariance (partition unity condition on  $\phi_1$)
  \item Perfectly reproduce sinusoids of period $M$
  \item Exact interpolation of points and first derivative
  \item Support of $\phi_1, \phi_2$ is $[-1, 1]$
  \item Hermite interpolation property of order 1
  \item $C^1$-continuous
\end{enumerate}


\subsection{The unit sphere with scaling factors $w_1$, $w_2$}

The usual continuous representation of the sphere is given by

\begin{equation}
  \sigma(u,v) = \left(\cos(2\pi u)\sin(\pi v), \sin(2\pi u)\sin(\pi v), \cos(\pi v)\right) \quad (u, v) \in {[0,1]}^2
\end{equation}

Suppose we have $M_1$ control points on latitudes, $M_2$ control points on meridians. The control points are then 
${c[k,l]}_{k \in [0, \ldots, M_1-1], l \in [0, \ldots, M_2-1]}$. Let $w_1 = \frac{2\pi}{M_1}, w_2 = \frac{\pi}{M_2}$.  

From the paper we have (also holds for $\sin$ functions)

\begin{align*}
  \forall u \in [0, M_1] \quad \cos(w_1u) &= \sum_{k \in \mathbb{Z}} \cos (w_1k) \phi_{1, w_1}(u-k) - w_1 \sin (w_1k) 
  \phi_{2, w_1} (u-k) \\
  \forall v \in [0, 2M_2] \quad \cos(w_2v) &= \sum_{l \in \mathbb{Z}} \cos (w_2l) \phi_{1, w_2}(v-l) - w_2 \sin (w_2l) 
\phi_{2, w_2} (v-l) \end{align*}

Normalizing the the continuous parameters leads to

\begin{align*}
  \forall u \in [0, 1] \quad \cos(2\pi u) &= \sum_{k \in \mathbb{Z}} \cos (w_1k) \phi_{1, w_1}(M_1u-k) - w_1 \sin (w_1k) 
  \phi_{2, w_1} (M_1u-k) \\
  \forall v \in [0, 2] \quad \cos(\pi v) &= \sum_{l \in \mathbb{Z}} \cos (w_2l) \phi_{1, w_2}(M_2v-l) - w_2 \sin (w_2l)
\phi_{2, w_2} (M_2v-l)
\end{align*}

Be aware that in the first representations above $\{\cos (w_1k), -w_1\sin(w_1k)\}$ is $(M_1, M_1)$-periodic i.e we need 
point and first derivative values at $M_1$ control points for a full representation. However in the second 
representation $\{\cos (w_2l), -w_2\sin(w_2l)\}$ are $(2M_2, 2M_2)$-periodic i.e we need point and first derivative 
values at $2M_2$ control points for a full representation. 

\subsection{Representation of the sphere}

For all $(u, v) \in {[0,1]}^2$

\begin{align*}
  \sigma(u,v) &= \sum_{(k,l) \in \mathbb{Z}^2} c_1[k,l] \phi_{1, w_1}(M_1u-k)\phi_{1, w_2}(M_2v-l) \\
  &+ \sum_{(k,l) \in \mathbb{Z}^2} c_2[k,l] \phi_{1, w_1}(M_1u-k)\phi_{2, w_2}(M_2v-l) \\
  &+ \sum_{(k,l) \in \mathbb{Z}^2} c_3[k,l] \phi_{2, w_1}(M_1u-k)\phi_{1, w_2}(M_2v-l) \\
  &+ \sum_{(k,l) \in \mathbb{Z}^2} c_4[k,l] \phi_{2, w_1}(M_1u-k)\phi_{2, w_2}(M_2v-l) \\
\end{align*}

Or equivalently for all $(u, v) \in {[0,1]}^2$
\begin{align*}
  \sigma(u,v) &= \sum_{k=0}^{M_1} \sum_{l=0}^{M_2} c_1[k,l] \phi_{1, w_1}(M_1u-k)\phi_{1, w_2}(M_2v-l) \\
  &+ \sum_{k=0}^{M_1} \sum_{l=0}^{M_2} c_2[k,l] \phi_{1, w_1}(M_1u-k)\phi_{2, w_2}(M_2v-l) \\
  &+ \sum_{k=0}^{M_1} \sum_{l=0}^{M_2} c_3[k,l] \phi_{2, w_1}(M_1u-k)\phi_{1, w_2}(M_2v-l) \\
  &+ \sum_{k=0}^{M_1} \sum_{l=0}^{M_2} c_4[k,l] \phi_{2, w_1}(M_1u-k)\phi_{2, w_2}(M_2v-l) \\
\end{align*}

Or equivalently for all $(u, v) \in {[0,1]}^2$
\begin{align*}
  \sigma(u,v) &= \sum_{k=0}^{M_1-1} \sum_{l=0}^{2M_2-1} c_1[k,l] \phi_{1, w_1, per}(M_1u-k)\phi_{1, w_2, per}(M_2v-l) \\
  &+ \sum_{k=0}^{M_1-1} \sum_{l=0}^{2M_2-1} c_2[k,l] \phi_{1, w_1, per}(M_1u-k)\phi_{2, w_2, per}(M_2v-l) \\
  &+ \sum_{k=0}^{M_1-1} \sum_{l=0}^{2M_2-1} c_3[k,l] \phi_{2, w_1, per}(M_1u-k)\phi_{1, w_2, per}(M_2v-l) \\
  &+ \sum_{k=0}^{M_1-1} \sum_{l=0}^{2M_2-1} c_4[k,l] \phi_{2, w_1, per}(M_1u-k)\phi_{2, w_2, per}(M_2v-l) \\
\end{align*}

Or equivalently for all $(u, v) \in {[0,1]}^2$
\boxalign[0.6\textwidth]{\begin{align*}
  \sigma(u,v) &= \sum_{k=0}^{M_1-1} \sum_{l=0}^{M_2} c_1[k,l]\phi_{1, w_1, per}(M_1u-k)\phi_{1, w_2}(M_2v-l) \\
  &+ \sum_{k=0}^{M_1-1} \sum_{l=0}^{M_2} c_2[k,l] \phi_{1, w_1, per}(M_1u-k)\phi_{2, w_2}(M_2v-l) \\
  &+ \sum_{k=0}^{M_1-1} \sum_{l=0}^{M_2} c_3[k,l] \phi_{2, w_1, per}(M_1u-k)\phi_{1, w_2}(M_2v-l) \\
  &+ \sum_{k=0}^{M_1-1} \sum_{l=0}^{M_2} c_4[k,l] \phi_{2, w_1, per}(M_1u-k)\phi_{2, w_2}(M_2v-l) \\
\end{align*}}


\boxalign[0.6\textwidth]{\small {\begin{align*}
  c_1[k,l]=\begin{bmatrix} \cos(w_1k)\sin(w_2l) \\ \sin(w_1k)\sin(w_2l) \\ \cos(w_2l) \end{bmatrix} &= 
  \sigma(\frac{k}{M_1},\frac{l}{M_2}) & c_2[k,l]=\begin{bmatrix} w_2\cos(w_1k)\cos(w_2l) \\ w_2\sin(w_1k)\cos(w_2l) \\ 
  -w_2\sin(w_2l) \end{bmatrix} &= \frac{1}{M_2}\frac{\partial \sigma}{\partial v}(\frac{k}{M_1}, \frac{l}{M_2}) \\
  c_3[k,l]=\begin{bmatrix} -w_1\sin(w_1k)\sin(w_2l) \\ w_1\cos(w_1k)\sin(w_2l) \\ 0 \end{bmatrix} &= \frac{1}{M_1} 
  \frac{\partial \sigma}{\partial u}(\frac{k}{M_1}, \frac{l}{M_2}) &
  c_4[k,l]=\begin{bmatrix} -w_1w_2\sin(w_1k)\cos(w_2l) \\ w_1w_2\cos(w_1k)\cos(w_2l) \\ 0 \end{bmatrix} &= \frac{1}{M_1 
  M_2} \frac{\partial^2 \sigma}{\partial u \partial v}(\frac{k}{M_1}, \frac{l}{M_2})
\end{align*}}%
}

\begin{align*}
  \phi_{1,w_1,per}(.) &= \sum_{k \in \mathbb{Z}} \phi_{1,w_1}(.-M_1k) & \phi_{1,w_2,per}(.) &= \sum_{k \in \mathbb{Z}} 
  \phi_{1,w_2}(.-2M_2k) \\
  \phi_{2,w_1,per}(.) &= \sum_{k \in \mathbb{Z}} \phi_{2,w_1}(.-M_1k) & \phi_{2,w_2,per}(.) &= \sum_{k \in \mathbb{Z}} 
  \phi_{2,w_2}(.-2M_2k) \\
\end{align*}

\section{Exponential B-splines in 3D}

\subsection{The parametric model}

Delgado et al's paper \underline{Spline-based deforming ellipsoids for 3D bioimage segmentation} derive an exponential 
B-splines-based model that allow to reproduce ellipsoids. The model can well approximate blobs and perfectly spheres and 
ellipsoids. The corresponding parametric representation is 

\begin{equation}
  \sigma(u, v) = \sum_{(i,j) \in \mathbb{Z}^2} c[i,j] \phi_1(\frac{u}{T_1}-i) \phi_2(\frac{v}{T_2}-j)
\end{equation}

with $T_1, T_2 > 0$ sampling steps for each parametric dimension and ${\{c[i,j]\}}_{(i,j) \in \mathbb{Z}^2}$ are the 3D 
control points.

The basis functions, reproducing unit period sinusoids with $M$ coefficients, are exponential B-splines given by

\begin{equation}
  \varphi_{M}(.) = \sum_{k=0}^3 {(-1)}^k h_{M}[k] \varsigma_M(. + \frac{3}{2} - k)
\end{equation}
  
where $\displaystyle \varsigma_M(.) = \frac{1}{4} sgn(.) \frac{\sin^2(\frac{\pi}{M}.)}{\sin^2(\frac{\pi}{M})}$ and $h_M 
= [1, 1+2\cos(\frac{2\pi}{M}), 1+2\cos(\frac{2\pi}{M}), 1]$. \\ 

Suppose we have $M_1$ control points on latitudes, $M_2$ control points on meridians. The resulting parametric model has 
the following properties
\begin{enumerate}
  \item Unique and stable representation (sufficient is ${\{\phi_1(.-k)\}}_{k}, {\{\phi_2(.-k)\}}_{k}$ Riesz basis)
  \item Affine invariance (partition unity condition on  $\phi_1$, $\phi_2$)
  \item Well-defined Gaussian curvature. $\phi_1$, $\phi_2$ are twice differentiable with bounded second derivative
  \item Perfectly reproduce ellipsoids
  \item Support of $\phi_1=\varphi_{M_1}, \phi_2=\varphi_{2M_2}$ is $[-\frac{3}{2}, \frac{3}{2}]$
\end{enumerate}

\subsection{Conditions for representing the unit sphere}

The parametric representation of a closed surface with sphere-like topology, $M_1$ control points on latitudes and $M_2$ 
control points on meridians is

\begin{equation}
  \forall (u, v) \in {[0,1]}^2 \quad \sigma(u,v) = \sum_{k=0}^{M_1-1} \sum_{l=-1}^{M_2+1} c[k,l] \phi_{1} 
  (M_1u-k)\phi_2(M_2v-l)
\end{equation}

Unlike before, continuity of points and tangents at poles is not guaranteed. Conditions are

\begin{align}
  \forall k=0, \ldots, M_1-1 \quad {\bf c_N} &= c[k,1] \phi_2(-1) + c[k,0]\phi_2(0) + c[k,-1] \phi_2(1) \\
  {\bf c_S} &= c[k,M_2+1] \phi_2(-1) + c[k,M_2]\phi_2(0) + c[k,M_2-1] \phi_2(1) \\ 
  {\bf T_{1,N}}\cos (2\pi u) + {\bf T_{2,N}}\sin (2\pi u) &=  M_2 \sum_{k=0}^{M_1-1} \sum_{l=-1}^{M_2+1} c[k,l] \phi_{1} 
  (M_1u-k)\phi_2'(-l) \\
  {\bf T_{1,S}}\cos (2\pi u) + {\bf T_{2,S}}\sin (2\pi u) &=  M_2 \sum_{k=0}^{M_1-1} \sum_{l=-1}^{M_2+1} c[k,l] \phi_{1} 
  (M_1u-k)\phi_2'(M_2-l) \\
\end{align}


Incorporating such conditions in the model, a parametric splines-based surface with a sphere-like topology, $C^1$ 
continuity and ellipsoid-reproducing capabilities (all positions and orientations) is given by

\begin{equation}
    \forall (u, v) \in {[0,1]}^2 \quad \sigma(u,v) = \sum_{k=0}^{M_1-1} \sum_{l=-1}^{M_2+1} c[k,l] \phi_{1, per} 
    (M_1u-k)\phi_2(M_2v-l)
\end{equation}
  
where ${\{c[i,j]\}}_{i \in [0, \ldots, M_1-1], j \in [1, \ldots, M_2-1]}, {\bf T_{1, N}}, {\bf T_{2, N}}, {\bf T_{1, S}} 
, {\bf T_{1, S}}, {\bf c_{N}}, {\bf c_{S}}$ are free parameters that is $M_1(M_2-1) + 6$ control points. \\

$c[k,-1], c[k, 0], c[k, M_2], c[k, M_2+1]$ are constrained by the values of the free parameters. \\

\subsection{Representation of the sphere}

The unit sphere is thus represented by

\begin{equation}
    \boxed{\forall (u, v) \in {[0,1]}^2 \quad \sigma(u,v) = \sum_{k=0}^{M_1-1} \sum_{l=-1}^{M_2+1} c[k,l] \phi_{1, per} 
    (M_1u-k)\phi_2(M_2v-l)}
\end{equation}


With coefficients are given by

\begin{equation}
  \boxed{c[k, l] = \begin{bmatrix} c_{M_1}[k]s_{2M_2}[l] \\ s_{M_1}[k]s_{2M_2}[l] \\ c_{2M_2}[l] \end{bmatrix} = 
    \begin{bmatrix} \frac{2(1-\cos(\frac{2\pi}{M_1}))}{\cos (\frac{\pi}{M_1})-\cos (\frac{3\pi}{M_1})}  
      \frac{2(1-\cos(\frac{\pi}{M_2}))}{\cos (\frac{\pi}{2M_2})-\cos (\frac{3\pi}{2M_2})} \cos (\frac{2\pi k}{M_1}) \sin 
      (\frac{\pi l}{M_2}) \\ \frac{2(1-\cos(\frac{2\pi}{M_1}))}{\cos (\frac{\pi}{M_1})-\cos (\frac{3\pi}{M_1})}  
      \frac{2(1-\cos(\frac{\pi}{M_2}))}{\cos (\frac{\pi}{2M_2})-\cos (\frac{3\pi}{2M_2})} \sin (\frac{2\pi k}{M_1}) \sin 
      (\frac{\pi l}{M_2}) \\
    \frac{2(1-\cos(\frac{\pi}{M_2}))}{\cos (\frac{\pi}{2M_2})-\cos (\frac{3\pi}{2M_2})} \cos (\frac{\pi l}{M_2}) 
\end{bmatrix}}
\end{equation}

and

\begin{align*}
  c_M[k] = \frac{2(1-\cos(\frac{2\pi}{M}))}{\cos (\frac{\pi}{M})-\cos (\frac{3\pi}{M})} \cos (\frac{2\pi k}{M}) \\
  s_M[k] = \frac{2(1-\cos(\frac{2\pi}{M}))}{\cos (\frac{\pi}{M})-\cos (\frac{3\pi}{M})} \sin (\frac{2\pi k}{M}) 
\end{align*}

These coefficients satisfy the constraints with 

\begin{align*}
{\bf c_N} &= {[0 \ 0 \ 1]}^T & {\bf c_N} &= {[0 \ 0 \ -1]}^T &  {\bf T_{1, N}} &= {[\pi \ 0 \ 0]}^T \\
{\bf T_{2,N}} &= {[0 \ \pi \ 0]}^T & {\bf T_{1, S}} &= {[-\pi \ 0 \ 0]}^T & {\bf T_{2,S}} &= {[0 \ -\pi \ 0]}^T
\end{align*}

\clearpage

\section{Compactly-supported smooth interpolators for shape modeling}

\subsection{The parametric model}

Schmitter et al's paper \underline{Compactly-supported smooth interpolators for shape modeling with varying resolution} 
propose a continuous representation of curves and surfaces with the help of generators that have the advantages of both 
continuous and discrete schemes. The generator is expressed as a linear combination of half integer shifts of 
exponential B-spline of vector $\alpha \in \mathbb{C}^n$  i.e

\begin{equation}
  \phi_{\lambda, \alpha}(t) = \sum_{k \in \mathbb{Z}} \lambda[k] \beta_{\alpha}(t-\frac{k}{2})
\end{equation}

$\beta_{\alpha}$ has support $[-\frac{n}{2}, \frac{n}{2}]$. In what follows we choose to have $\lambda[k]=0$ for $k 
\not\in \llbracket-n+2, n-2\rrbracket$ and $\lambda[-k]=\lambda[k]$. There are therefore $(n-1)$ unknowns $\lambda[0], 
\ldots, \lambda[n-2]$. We also impose that elements in $\alpha$ are 0 or come in complex conjugate pairs and that no 
pair of purely imaginary elements of $\alpha$ is separated by integer multiple of $2j\pi$ (for Riesz basis property). \\

This function is interpolatory if and only if $\phi_{\lambda, \alpha}(0)=1$ and $\phi_{\lambda, 
\alpha}(1)=\cdots=\phi_{\lambda, \alpha}(n-2)=0$. This defines a system of $n-1$ equations with $n-1$ unknowns. The 
system has a solution if the matrix defined by $k,l = 0, \ldots, n-2$

\begin{equation}
  {[A_{\alpha}]}_{k+1, l+1} = \begin{dcases}
    \beta_{\alpha}(k) &\text{if } l=0 \\
    \beta_{\alpha}(k-\frac{l}{2}) + \beta_{\alpha}(k+\frac{l}{2})  &\text{else }      
  \end{dcases}
\end{equation}

is invertible. In that case $\lambda = A_{\alpha}^{-1}(1, 0, \ldots, 0)$ and we define $\phi_{\alpha} = \phi_{\lambda, 
\alpha}$. Tensor-product surfaces are represented with the help of two generators in the form

\begin{equation}
  \sigma(u, v) = \sum_{k \in \mathbb{Z}}\sum_{l \in \mathbb{Z}} \sigma[k,l] \phi_{\alpha_1}(u-k) \phi_{\alpha_2}(v-l)
\end{equation}

The resulting interpolation scheme has the following properties
\begin{enumerate}
  \item Unique and stable representation ($\alpha_m - \alpha_n \not\in 2j\pi\mathbb{Z}$ Riesz basis)
  \item Affine invariance ($0 \in \alpha_1, 0 \in \alpha_2$)
  \item Perfectly reproduce ellipsoids (conditions on $\alpha$)
  \item $\phi_{\alpha}$ is interpolatory
  \item $\phi_{\alpha}$ is smooth i.e at least $\mathcal{C}^1$
  \item Can reproduce the nullspace $\mathcal{N}_{\alpha}$
  \item Can reproduce shapes at various resolutions
  \item $\phi_{\alpha}$ is compactly supported on $[-n+1, n-1]$
\end{enumerate}

\subsection{Conditions for representing the unit sphere}

Let $M_1$ be the number of control points in $u$ and $M_2$ the number of control points in $v$. For $\phi_{\alpha_1}$ to 
be able to reproduce $\cos (\frac{2\pi.}{M_1}), \sin (\frac{2\pi.}{M_1})$ we need to have $(\frac{-2i\pi}{M_1}, 
\frac{2i\pi}{M_1}) \in \alpha_1$. Adding affine invariance condition, $\phi_{(0, \frac{-2i\pi}{M_1}, 
\frac{2i\pi}{M_1})}$ can reproduce constants and $M_1$-periodic sinusoids with $M_1$ control points as follows

\begin{equation}
  \cos (\frac{2\pi.}{M_1}) = \sum_{k \in \mathbb{Z}} \cos (\frac{2\pi k}{M_1}) \phi_{\alpha_1}(.-k)
\end{equation}

\clearpage

Similarly $\phi_{(0, \frac{-i\pi}{M_2}, \frac{i\pi}{M_2})}$ can reproduce constants and $2M_2$-periodic sinusoids with 
$2M_2$ control points as follows

\begin{equation}
  \cos (\frac{\pi.}{M_2}) = \sum_{k \in \mathbb{Z}} \cos (\frac{\pi k}{M_2}) \phi_{\alpha_2}(.-k)
\end{equation}

Generators $\phi_{\alpha_1}, \phi_{\alpha_2}$ both have support of size 4 ($n=3$) so that they are given by

\begin{align*}
  \phi_{\alpha_1}(t) &= \lambda_1[0]\beta_{\alpha_1}(t) + \lambda_1[1](\beta_{\alpha_1}(t-1/2) + 
  \beta_{\alpha_1}(t+1/2)) \\
  \phi_{\alpha_2}(t) &= \lambda_2[0]\beta_{\alpha_2}(t) + \lambda_2[1](\beta_{\alpha_2}(t-1/2) + 
  \beta_{\alpha_2}(t+1/2)) 
\end{align*}

In order to find $\lambda_1[0], \lambda_1[1]$ one has to solve $\phi_{\alpha_1}(0) = 1, \phi_{\alpha_1}(1) = 0$. \\ 

\underline{Aparte on tempered distributions} \\

Green function of operator $L_{\alpha}: \mathcal{S}(\mathbb{R}) \to \mathcal{S}'(\mathbb{R})$ is an element 
$\rho_{\alpha}$ of $\mathcal{S}(\mathbb{R})$ that satisfies $L\{\rho_{\alpha}\} = \delta$ where $\delta$ is the Dirac 
tempered distribution. There is a unique such function (to be proved?) that also satisfies $\forall t < 0 \ 
\rho_{\alpha}(t) < 0$. The tempered distribution $T_{\rho_{\alpha}} : \phi \mapsto \int_{0}^{\infty} e^{\alpha t} 
\phi(t) dt$ is such that the associated element of $\mathcal{S}(\mathbb{R})$ (bijection $\mathcal{S}(\mathbb{R}) \to 
\mathcal{S}'(\mathbb{R})$? probably not true), $\rho_{\alpha}$ satisfies $L_{\alpha}\{\rho_{\alpha}\} = \delta$. \\

Consequently $\beta^+_{\alpha}$ is an element of Schwartz space $\mathcal{S}(\mathbb{R})$. \textbf{Is that right}? With 
abuse of notation we write $\beta^+_{\alpha}(t) = e^{\alpha t}\chi_{[0,1]}(t)$. Using distribution then we would have

\begin{equation*}
  {\beta^+_{\alpha}}'(t) = \delta(t) + \alpha e^{\alpha t}\chi_{[0,1]}(t)
\end{equation*}

The equality is to be taken in the distribution sense. 

\subsection{Representation of the sphere}

The unit sphere is thus represented by

\begin{equation}
  \forall (u,v) \in {[0,1]}^2 \quad \sigma(u,v) = \sum_{(k,l) \in \mathbb{Z}^2} c[k,l] \phi_{\alpha_1}(M_1u-k) 
  \phi_{\alpha_2}(M_2v-l)
\end{equation}

Or equivalently
\begin{equation}
  \forall (u,v) \in {[0,1]}^2 \quad \sigma(u,v) = \sum_{k=0}^{M_1-1} \sum_{l \in \mathbb{Z}} c[k,l] \phi_{\alpha_1, 
  per}(M_1u-k) \phi_{\alpha_2}(M_2v-l)
\end{equation}

Or equivalently
\begin{equation}
  \boxed{\forall (u,v) \in {[0,1]}^2 \quad \sigma(u,v) = \sum_{k=0}^{M_1-1} \sum_{l=-1}^{M_2+1} c[k,l] \phi_{\alpha_1, 
  per}(M_1u-k) \phi_{\alpha_2}(M_2v-l)}
\end{equation}

Denoting $w_1 = \frac{2\pi}{M_1}, w_2 = \frac{\pi}{M_2}$, the coefficients are given by

\begin{equation}
  \boxed{c[k,l] =\begin{bmatrix} \cos(w_1k)\sin(w_2l) \\ \sin(w_1k)\sin(w_2l) \\ \cos(w_2l) \end{bmatrix} = 
\sigma(\frac{k}{M_1}, \frac{l}{M_2})} \end{equation}

\section{Smooth shapes with spherical topology}

\subsection{The parametric model}

In 2017 Schmitter et al's paper \underline{Smooth shapes with spherical topology} derive a parametric model very similar 
to that presented in Spline-based deforming ellipsoids for 3D bioimage segmentation. In user interactive applications 
one usually wants a curve/shape reproducing model to have some or all following properties: 1.intuitive manipulation, 
2.stable deformation, 3.shape deformation as optimization process requiring fast evaluation of surface and volume 
integrals, 4.smooth representation. It is usually impossible to find a model optimal w.r.t to all these requirements.  
In practice a compromise is made with existing models based on polygon meshes, subdivision or NURBS\@. \\

Parametric shapes are built as linear combinations of integers shifts of a generator function $\varphi$ that is to say

\begin{equation}
  r(t) = \sum_{k \in \mathbb{Z}} c[k]\varphi(t-k)
\end{equation}

$\varphi$ is piecewise exponential. It is the smoothed version of third order exponential B-spline that is $\varphi = 
\beta*\psi$ with $\psi$ an appropriate smoothing kernel. The model can be extended to tensor-product surfaces $\sigma(u, 
v)$ as previously done in previous representations.  

\begin{equation*}
  \sigma(u,v) = \sum_{(k,l) \in \mathbb{Z}^2} c[k,l] \varphi_{1} (u-k)\varphi_2(v-l)
\end{equation*}

Authors define

\begin{align*}
  \phi_1(t) &= \varphi_{M_1, per}(t) = \sum_{n \in \mathbb{Z}} \varphi_{M_1}(t-M_1n) & \phi_2(t) &= \varphi_{2M_2}(t) \\
  \forall k \in \mathbb{Z} \quad \phi_{1,k}(t) &= \phi_1(M_1t-k) & \phi_{2,k}(t) &= \phi_2(M_2t-k)
\end{align*}

The resulting interpolation scheme has the following properties
\begin{enumerate}
  \item Unique and stable representation (${\{\varphi_M(.-k)\}}_{k}$ Riesz basis)
  \item Affine invariance (partition unity condition on  $\varphi_M$)
  \item Well-defined Gaussian curvature. $\varphi_M$ is twice differentiable with bounded second derivative
  \item Perfectly reproduce ellipsoids for $M \geq 3$
  \item $\varphi_M$ is interpolatory
  \item Support of $\varphi_M$ is in $[-2, 2]$
\end{enumerate}

\subsection{Conditions for representing the unit sphere}

The parametric representation of a closed surface with sphere-like topology, $M_1$ control points on latitudes and $M_2$ 
control points on meridians is

\begin{equation}
  \forall (u, v) \in {[0,1]}^2 \quad \sigma(u,v) = \sum_{k=0}^{M_1-1} \sum_{l=-1}^{M_2+1} c[k,l] 
\phi_{1,k}(u)\phi_{2,k}(v) 
\end{equation}

As for the model from article 7, continuity of points and tangents at poles is not guaranteed.The exact same conditions 
are used leading to

\begin{align}
  \forall k=0, \ldots, M_1-1 \quad {\bf c_N} &= c[k,0] \\
  {\bf c_S} &= c[k, M_2] \\
  {\bf T_{1,N}}\cos (2\pi u) + {\bf T_{2,N}}\sin (2\pi u) &= M_2 \sum_{k=0}^{M_1-1} \sum_{l=-1}^{M_2+1} c[k,l] 
  \varphi_{M_1} (M_1u-k)\varphi_{2M_2}'(-l) \\
  {\bf T_{1,S}}\cos (2\pi u) + {\bf T_{2,S}}\sin (2\pi u) &= M_2 \sum_{k=0}^{M_1-1} \sum_{l=-1}^{M_2+1} c[k,l] 
  \varphi_{M_1} (M_1u-k)\varphi_{2M_2}'(M_2-l)\\
\end{align}


By incorporating conditions in the model to ensure continuity of the surface and of the tangent plane at poles we obtain 
constraints on $c[k, -1], c[k,0], c[k,M_2], c[k, M_2+1]$. Other values $c[k,l]$ are free parameters as well as
${\bf c_N, c_S, T_{1, N}, T_{2, N}, T_{1, S}, T_{2, S}}$ describing the poles. \\

$\varphi_{M_1}$ can reproduce $\cos (\frac{2\pi.}{M_1})$ with $M_1$ control points

\begin{align*}
  \cos (\frac{2\pi u}{M_1}) &= \sum_{k \in \mathbb{Z}} \cos (\frac{2 \pi k}{M_1}) \varphi_{M_1}(u-k) \\
  \cos (2\pi u) &= \sum_{k=0}^{M_1-1} \cos (\frac{2 \pi k}{M_1}) \phi_{1, k}(u) \\
\end{align*}

In a similar fashion $\varphi_{2M_2}$ can reproduce $\cos (\frac{\pi.}{M_2})$ with $2M_2$ control points i.e

\begin{align*}
  \cos (\frac{\pi v}{M_2}) &= \sum_{k \in \mathbb{Z}} \cos (\frac{\pi k}{M_2}) \varphi_{2M_2}(v-k) \\
  \cos (\pi v) &= \sum_{k \in \mathbb{Z}} \cos (\frac{\pi k}{M_2}) \phi_{2, k}(v) \\
\end{align*}

\subsection{Representation of the sphere}

Given the usual representation of the unit sphere, it can be represented in our scheme as follows

\begin{equation}
    \sigma(u, v) = \sum_{k=0}^{M_1-1} \sum_{l \in \mathbb{Z}} c[k,l] \phi_{1,k}(u)\phi_{2, k}(v)
\end{equation}

or using the fact $\varphi_{2M_2}$ has support of size 4

\begin{equation}
  \boxed{\forall (u,v) \in {[0,1]}^2 \quad \sigma(u, v) = \sum_{k=0}^{M_1-1} \sum_{l=-1}^{M_2+1} c[k,l] 
  \phi_{1,k}(u)\phi_{2, k}(v)}
\end{equation}

Denoting $w_1 = \frac{2\pi}{M_1}$, $w_2 = \frac{\pi}{M_2}$, the coefficients are given by

\begin{equation}
  \boxed{c[k,l]=\begin{bmatrix} \cos(w_1k)\sin(w_2l) \\ \sin(w_1k)\sin(w_2l) \\ \cos(w_2l) 
  \end{bmatrix}=\sigma(\frac{k}{M_1}, \frac{l}{M_2})}
\end{equation}

%
%where 
%
%\begin{align*}
%  c_M[k] &= \frac{2(1-\cos (\frac{2\pi}{M}))}{\cos (\frac{\pi}{M})  - \cos (\frac{3\pi}{M})} \cos (\frac{2\pi k}{M}) \\
%
%\end{align*}
%
%\section{Comparison of the different representations}
%
%\begin{center}
%  \begin{tabular}{p{2cm}|p{2cm}|p{2cm}|p{2cm}|p{2cm}|p{1.5cm}|p{1.25cm}|p{1.25cm}|p{1.25cm}}
%    & Basis functions & Prop basis functions &  Unique \& Stable & Affine invariance & Repro. \ ellipsoids & Interp. \  
%    points & Interp.  1st deriv. & Interp.  2nd deriv   \\ \toprule
%    Exponential Hermite splines (1st-order) & $\phi_{1, w_1}, \phi_{1,w_2}$ $\phi_{2, w_1}, \phi_{2,w_2}$ & Piecewise 
%    polynomial exp., $\mathcal{C}^1$ continuous & Yes (Riesz) $\phi_{1, w_i}, \phi_{2, w_i}$ & Yes. Part. \ unity  
%    $\phi_{1, w_1}, \phi_{1, w_2}$ & Yes  & Yes & Yes & No \\ \bottomrule
%  \end{tabular}
%\end{center}

\end{document}

