\chapter{The Hermite interpolation problem}\label{chapter2}

From a historical perspective, the interpolation problem was intially stated as finding an interpolation function at 
equally-spaced locations, as described in Chapter~\ref{chapter1}. Later on, as the interest in interpolation grew among 
mathematicians, a number of more general formulas were indentified to cover fairly arbitrary point configurations.  
However, configurations with repeated interpolation point locations remained largely unexplored until the general 
problem formulated by Hermite in 1877~\cite{hermite_1877}. There, Hermite sets the task of finding a polynomial of 
degree $n-1$ that satisfies a total of $n$ interpolating conditions in the form of consecutive derivatives at distinct 
locations.  Theorem~\ref{thm:unique_pol} states that there exists a unique such polynomial, and we discussed how the 
interpolating polynomial can be written in \emph{Newton form}, using the extended divided difference operator. However,
the result does not extend to the problem of interpolating at an infinite number of locations, even when only values and 
not derivatives are to be reproduced, as the degree of the interpolating polynomial would then be infinite. Another 
approach to the problem, as previously mentioned at the end of Section~\ref{sec:splines}, is to break it down to an 
infinite number of finite interpolation problems. In that setting, the focus is restricted to a subset of the 
conditions, defining a \emph{piece} of the interpolant, before joining pieces in a smooth fashion.  This is the essence 
of splines, which are fundamental in our formulation of solutions to the so-called \emph{cardinal Hermite interpolation 
problem}.

\section{Schoenberg's theorems}

We here introduce the fundamental results of Schoenberg on the general problem of interpolating a function and a certain 
number of consecutive derivatives on the integer grid. Schoenberg dedicated a good part of his life to his work on 
splines, starting from his landmark paper~\cite{schoenberg_contributions_1946}, and continued with a series of papers 
among 
which~\cite{schoenberg_cardinal_1969},~\cite{schoenberg_cardinal_1972a},~\cite{schoenberg_cardinal_1972b},~\cite{lipow_cardinal_1973} 
and~\cite{schoenberg_cardinal_1973} are of particular relevance to our work. This section is highly inspired by his work 
to which we will refer a lot.  An elegant compilation Schoenberg's work on splines can be found 
in~\cite{schoenberg_cardinal_1973:1}, which contains 10 lectures, each presenting one specific aspect of splines while 
laying the foundations for the lectures that follows. We start by stating the notations that we will use throughout this 
chapter.

\begin{itemize}
  \itemsep0em
  \item $r, m, n \in \mathbb{N}^*$, $r \geq m$;
  \item $\mathbb{Z}_r = {\{ t_j=k | kr \leq j < k(r+1) \}}_{j \in \mathbb{Z}}$ is the set of integers repeated $r$ 
    times;
  \item $\mathscr{S}_{n}$ are the Schoenberg's cardinal splines of \emph{order} $n$    
    (Definition~\ref{def:card-splines});
  \item $\mathscr{S}_{n, \mathbb{Z}_r}$ are the De Boor's splines of order $n$ with knots on $\mathbb{Z}_r$ 
    (Definition~\ref{def:gen-splines});
  \item $\bm{y}^{(0)}, \ldots, \bm{y}^{(r-1)}$ are sequences of real numbers.
\end{itemize}

\subsection{The cardinal Hermite interpolation problem}

\begin{deftn}[C.H.I.P, {\cite[(10)-(12)]{lipow_cardinal_1973}}]
  The $r$ sequences of real numbers $\bm{y}^{(0)}, \ldots, \bm{y}^{(r-1)}$ being prescribed, the cardinal Hermite 
  interpolation problem (C.H.I.P) for the vector space $\mathcal{V}$, ($\bm{y}^{(0)}, \ldots, \bm{y}^{(r-1)}, 
  \mathcal{V}$), is the problem of finding a function $S \in \mathcal{V}$
  that agrees with the sequences $\bm{y}^{(0)}, \ldots, \bm{y}^{(r-1)}$ in the sense that
  \begin{equation}\label{def:chip}
    \forall \rho \in \llbracket0,r-1\rrbracket, \quad \forall k \in \mathbb{Z}, \quad S^{(\rho)}(k) = y^{(\rho)}_k
  \end{equation}
\end{deftn}
\noindent In non-mathematical words, a C.H.I.P aims at interpolating an unknown function for which we only have samples 
of its values and possibly its derivatives on a uniformly-spaced grid. Such questions arise in signal processing and 
image nalysis, where the uniform grid is the array detector of a camera and the samples are the pixels, all of which 
form an image. An image is thus only a discretized version of an underlying continuous reality that we would like to 
approximate from the pixels. One such continuous reality could, for instance, be the surface of a cell, which, when 
successfully modelled, can increase our understanding of its mechanical and biological properties.\\

In Chapter~\ref{chapter1}, the space of cardinal splines of order $n$, $\mathscr{S}_n$, was defined as those functions 
that are polynomials of order $n$ on all intervals $(k, k+1)$ for $k \in \mathbb{Z}$ and that belong to the class 
$\mathcal{C}^{n-2}$. It is possible and relevant for the Hermite interpolation problem to consider similar functions 
with less degrees of continuity at the joining points.
\begin{deftn}[{\cite[Lecture 5]{schoenberg_cardinal_1973-1}}]
  The set $\mathscr{S}_{n, r}$ of cardinal splines of \emph{order} $n$ and \emph{multiplicity} $r$ denotes all functions 
  $S$ such that
  \begin{enumerate}
    \item $S \in \Pi_{<n}$ on $(k, k+1)$ for $k \in \mathbb{Z}$;
    \item $S \in \mathcal{C}^{n-r-1}$.
  \end{enumerate}
  At times, it is convenient to consider the splines halfway between the integers, that is,
  \begin{equation*}
    \mathscr{S}^*_{n,r} = \{S | S(\cdot + \frac{1}{2}) \in \mathscr{S}_{n,r}\}.
  \end{equation*}
\end{deftn}

\begin{remark}
  This new set of splines connects to the sets defined in Definition~\ref{def:ppol-cont},~\ref{def:gen-splines}
  \begin{equation*}
    \mathscr{S}_{n,r} = \mathscr{S}_{n, \mathbb{Z}_r} = \Pi_{<n, \mathbb{Z}, \bm{\nu}},
  \end{equation*}
  where $\bm{\nu}$ is the constant sequence with value $n-r$. Also, splines of order $n$ for the particular case of 
  multiplicity $r=1$ are cardinal splines as in Definition~\ref{def:card-splines}, i.e,
  \begin{equation*}
    \mathscr{S}_n = \mathscr{S}_{n,1}.
  \end{equation*}
\end{remark}
\noindent Solutions for the C.H.I.P~\ref{def:chip} are readily obtained in the form of functions in $S_{n,r}$ or $S_{n, 
r}^*$ as expressed in the following lemma.
\begin{lem}\label{lemma:manifold}
  \begin{enumerate}
    \item The Hermite interpolation problem~\ref{def:chip} with $\mathcal{V}=\mathscr{S}_{n,r}$ has infinitely many 
      solutions that form a linear manifold of dimension $n-2r$.
    \item The Hermite interpolation problem~\ref{def:chip} with $\mathcal{V}=\mathscr{S}_{n,r}^*$ has infinitely many 
      solutions that form a linear manifold of dimension $n-r$.
  \end{enumerate}
\end{lem}

The proof is given in Appendix~\ref{chapter:annexA}, extending Schoenberg's proof of (\cite[Lemma 1.1, Lecture 
4]{schoenberg_cardinal_1973-1}) for cardinal interpolation, which is none other than cardinal Hermite interpolation for 
$r=1$. An immediate corollary of this lemma for the sequences to be interpolated that vanish identically is obtained as 
follows.

\begin{lem}\label{lemma:linear-space}
  Define 
  \begin{align}\label{def:ringsplines}
    \mathring{\mathscr{S}}_{n, r} &= \{S \in \mathscr{S}_{n,r}| S^{(\rho)}(k) = 0, \rho=0, \ldots, r-1, k \in 
    \mathbb{Z}\}; \\
    \mathring{\mathscr{S}}_{n,r}^* &= \{S \in \mathscr{S}_{n,r}^*| S^{(\rho)}(k) = 0, \rho=0, \ldots, r-1, k \in 
    \mathbb{Z}\}.
  \end{align}
  Then, $\mathring{\mathscr{S}}_{n, r},  \mathring{\mathscr{S}}_{n, r}^*$ are linear spaces with
  \begin{equation}
    \dim \mathring{\mathscr{S}}_{n,r} = n-2r, \quad \dim \mathring{\mathscr{S}}_{n,r}^* = n-r.
  \end{equation}
\end{lem}

\begin{proof}
  The proof is immediate using Lemma~\ref{lemma:manifold}. Indeed, $\mathring{\mathscr{S}}_{n,r}$ is exactly the set of 
  solutions to the C.H.I.P ($0, \ldots, 0, \mathscr{S}_{n,r}$), which is not only a linear manifold but also a linear 
  space as it contains the trivial spline. Furthermore, it has dimension $n-2r$. A similar reasoning applies to 
  $\mathring{\mathscr{S}}_{n,r}^*$.
\end{proof}

\subsection{Spline interpolant to sequences of power growth}

Let $\gamma \geq 0$ be a nonnegative real number and let
\begin{align}
  \mathcal{Y}^{\gamma} &= \{\bm{y} \in \mathbb{R}^{\mathbb{Z}} | y_k = \mathcal{O}_{|k| \to \infty}({|k|}^{\gamma})\},\\
  \mathscr{S}_{n, r}^{\gamma} &= \{S \in \mathscr{S}_{n,r} | S(t) = \mathcal{O}_{|t| \to \infty}({|t|}^{\gamma})\},
\end{align}
be respectively the spaces of power growth sequences and power growth splines with power $\gamma$.  As mentioned by in 
after (\cite[(2.1)]{schoenberg_cardinal_1973-1}), application of Markov's theorem on bounds shows that all derivatives 
of $S \in \mathscr{S}_{n,r}^{\gamma}$ satisfy the same decay condition. This is something Schoenberg most probably 
noticed after publishing (\cite{lipow_cardinal_1973}) as, in the latter, he defines $\mathscr{S}_{n, r}^{\gamma}$ as the set of splines 
whose derivatives up to $r-1$ are of power growth $\gamma$. \\

From now on we assume that $n=2m$ is an even number with $m \geq r$. This is the choice made by Lipow and Schoenberg n 
(\cite{lipow_cardinal_1973}) in order to avoid rewriting known results with slightly different notations.  Indeed, all 
results for even values of $n$ are easily extended to the case of odd values of $n$ using the spline space $S_{n,r}^*$ 
and all subsequently defined sets as we did in Lemma~\ref{lemma:manifold}.  This lemma also shows that the 
C.H.I.P~\ref{def:chip} has an infinite number of solutions in the form of splines. However, if we consider functions in 
the set $\mathscr{S}_{2m,r}^{\gamma}$ of splines with power growth $\gamma$, the set of solutions to ($\bm{y^{(0)}}, 
\ldots, \bm{y^{(r-1)}}, \mathscr{S}_{2m,r}^{\gamma}$) reduces to a unique element provided that the sequences satisfy 
the same power growth. This is the topic of the following theorem, which is the main result and is central to the theory 
of Hermite interpolation

\begin{thm}[{\cite[Theorems 1,4]{lipow_cardinal_1973}}]\label{thm:LH-gamma}
  The C.H.I.P ($\bm{y}^{(0)}, \ldots, \bm{y}^{(r-1)}, \mathscr{S}_{2m,r}^{\gamma})$ has a unique solution if and only if 
  $\bm{y}^{(0)}, \ldots, \bm{y}^{(r-1)}$ are in  $\mathcal{Y}^{\gamma}$. In that case, the solution is explicitly given 
  by the Lagrange-Hermite expansion
  \begin{equation}\label{eq:LH}
    S(t) = \sum_{k=-\infty}^{\infty} \sum_{s=0}^{r-1} y_k^{(s)} L_s(t-k)
  \end{equation}
  where $L_0, \ldots, L_{r-1}$ are the fundamental splines as defined in Definition~\ref{def:fundamental-r}. This 
  expansion converges absolutely and locally uniformly.
\end{thm}
\noindent The full proof is provided in~\cite{lipow_cardinal_1973} and~\cite[Lecture 5]{schoenberg_cardinal_1973-1}. We 
hereafter give a sketch of the proof and refer the reader to~\cite{lipow_cardinal_1973} for details. 
  
\begin{proof}[Proof sketch]
  \textbf{(Unicity)} We start by noticing that the difference of two solutions $S$ belongs to 
  $\mathring{\mathscr{S}}_{2m, r}^{\gamma}$.  From Lemma~\ref{lemma:linear-space}, this set is a linear space of 
  dimension $2m-2r$. The $2m-2r$ ``eigensplines'' $\{S_1, \ldots, S_{2m-2r}\}$ (Definition~\ref{def:eigsplines-r}) form 
  a basis of this linear space (\cite[Lemma 3, Lecture 5]{schoenberg_cardinal_1973-1}). As a consequence, there exist 
  coefficients $c_1, \ldots, c_{2m-2r}$ such that
  \begin{equation}\label{eq:diff-expansion}
    S = \sum_{j=1}^{2m-2r} c_j S_{j}.
  \end{equation}
  Eigensplines behaving towards infinity as~\cite[(5.16), (5.17)]{lipow_cardinal_1973}
  \begin{align}
    0 &< \overline{\lim}_{x \to -\infty} \frac{|S_{j}(x)|}{|\lambda_{j}|^x} < \infty \quad j=1, \ldots, m-r, \\
    0 &< \overline{\lim}_{x \to \infty} \frac{|S_{j}(x)|}{|\lambda_{j}|^x} < \infty \quad j=m-r+1, \ldots, 2m-2r,
  \end{align}
  and $S(t) = \mathcal{O}_{|t|\to\infty}(|t|^{\gamma})$ having power growth $\gamma$ at infinity, all the coefficients 
  must vanish and so does $S$.  \\

  \textbf{(Existence)} An explicit solution is constructed using an expansion in terms of \emph{fundamental} splines 
  ${L_s := L_{2m,r,s}}$ for $s \in \llbracket0,r-1\rrbracket$ (Definition~\ref{def:fundamental-r}). They are defined as 
  bounded functions such that $L_s$ and $s$ have the same parity and
  \begin{equation*}
    L_s(t) = \begin{dcases} P_s(t) & \text{if} \ 0 \leq t \leq 1, \\
    \sum_{j=1}^{m-r} c_{j,s} S_j(t) & \text{if} \ t \geq 1, \end{dcases}
  \end{equation*}
  with
  \begin{equation*}\scriptstyle
    P_s(t) = \begin{dcases}\scriptstyle \frac{1}{s!}t^s + a_{1,s}t^r + a_{2,s}t^{r+2} + \cdots + a_{m-r+1,s}t^{2m-r} + 
      a_{m-r+2,s} t^{2m-r+1} + \cdots + a_{m,s} t^{2m-r} & \text{if $2|(r-s)$} \\
      \scriptstyle\frac{1}{s!}t^s + a_{1,s}t^{r+1} + a_{2,s}t^{r+3} + \cdots + a_{m-r,s}t^{2m-r-1} + a_{m-r+1,s} 
      t^{2m-r} + \cdots + a_{m,s} t^{2m-r} & \text{otherwise},
   \end{dcases}
  \end{equation*}
  where the $2m-r$ unknowns $a_{1,s}, \ldots, a_{m,s}, c_{1,s}, \ldots, c_{m-r,s}$ defining $L_s$ are obtained as the 
  unique solution of the linear system of $2m-r$ equations
  \begin{equation*}
    \forall \rho \in \llbracket0,2m-r-1\rrbracket, \qquad P_s^{(\rho)}(1) =\sum_{j=1}^{m-r} c_{j,s} S_j^{(\rho)}(1).
  \end{equation*}
  The solution is unique because the associated homogeneous system (removing the $\frac{1}{s!}t^s$) is non-singular.  
  Indeed, if it were to be singular, there would exist a non trivial bounded spline in $\mathscr{S}_{2m,r}^{0}$
  that vanishes with all its derivatives up to order $r-1$. However, from the proof of unicity we know that there can be 
  at most one such spline. The trivial spline being one of of them leads to a contradiction.\\

  A solution to the C.H.I.P ($\bm{y}^{(0)}, \ldots, \bm{y}^{(r-1)}, \mathscr{S}_{2m,r}^{\gamma})$ is then given by
  \begin{equation*}
    S(t) = \sum_{k=-\infty}^{\infty} \sum_{s=0}^{r-1} y_k^{(s)} L_s(t-k),
  \end{equation*}
  since, by construction, the \emph{fundamental} splines $L_s$ are in $\mathscr{S}_{2m,r}^0 \subset 
  \mathscr{S}_{2m,r}^{\gamma}$ and satisfy
  \begin{equation}\label{eq:fund-prop}
    \forall \rho \in \llbracket0,r-1\rrbracket, \qquad \forall k \in \mathbb{Z}, \quad L_s^{(\rho)}(k) = 
    \delta_{k}\delta_{s-\rho}.
  \end{equation}
  This completes the proof.
\end{proof}

These results have been extended to the cases of sequences in $l^p$ for $1 \leq p \leq \infty$, where the interpolating 
spline is a function in the set 
\begin{equation*}
  \mathcal{L}^p_r = \{F:\mathbb{R} \to \mathbb{C}| F^{(\rho)} \in \mathcal{L}^p(\mathbb{R}, \mathbb{C}), \rho=0, \ldots, 
  r-1\}.
\end{equation*}

\begin{thm}[{\cite[Theorems 2,4]{lipow_cardinal_1973}}]\label{thm:LH-p} Let $1 \leq p \leq \infty$. The C.H.I.P 
  ($\bm{y}^{(0)}, \ldots, \bm{y}^{(r-1)}, \mathscr{S}_{2m,r}\cap L^p_r)$ has a unique solution if and only if 
  $\bm{y}^{(0)}, \ldots, \bm{y}^{(r-1)}$ are in  $l^p$. In that case, the solution is explicitly given by the 
  Lagrange-Hermite expansion
  \begin{equation}
    S(t) = \sum_{k=-\infty}^{\infty} \sum_{s=0}^{r-1} y_k^{(s)} L_s(t-k).
  \end{equation}
  This expansion converges absolutely and locally uniformly.
\end{thm}

\section{Hermite B-splines}

We refer to as \emph{generators} functions that span a given linear space in the usual algebraic sense.  This set of 
functions can be infinite as we shall see. In some cases, it is itself generated as all integer translates of a finite 
set of functions, to which we shall also refer to as \emph{generators} by extension. In that sense, the set of functions 
arising from the Whittaker-Shannon interpolation formula, expressed as
\begin{equation*}
  SW = \{y(t) = \sum_{k=-\infty}^{\infty} y_k \sinc(t-k) | y_k \in \mathcal{Y}\}
\end{equation*}
with $\mathcal{Y}$ all real or complex sequences satisfying $\displaystyle\sum_{k=-\infty, k\neq0}^{\infty} 
|\frac{y_k}{k}|<\infty$ is generated by the infinite set of functions ${\{\sinc(\cdot-k)\}}_{k \in \mathbb{Z}}$. This 
infinite set of functions is itself generated by integer translates of the single function $\{\sinc\}$. We shall thus 
refer to $\sinc$ as a generator for the space $SW$. In that sense also, the $r$ fundamental splines $\{L_0, \ldots, 
L_{r-1}\}$ are generators for the linear space
\begin{equation}\label{eq:def-V}
  V = \{S(t) = \sum_{k=-\infty}^{\infty} \sum_{s=0}^{r-1} c_{k,s} L_s(t-k) | c_0, \ldots, c_{r-1} \in 
\mathbb{R}^{\mathbb{Z}}\}
\end{equation}
The functions appearing as infinite sums in (\ref{eq:def-V}) may not always be properly defined, in which case
conditions must be set on sequences for the series to converge. From Theorem~\ref{thm:LH-gamma} and 
Theorem~\ref{thm:LH-p}, we know that, if the sequences of coefficients are in $\mathcal{Y}^{\gamma}$ or in $l^p$, then 
the series converges locally uniformly to a function in $S_{2m, r}^{\gamma}$ or $S_{2m, r}\cap L^p_r$ respectively. \\

Many questions naturally arise about the properties of the generators and the properties of the generated functions. The 
next section introduces the Hermite B-splines, closely related to $L_{2m,r,0}, \ldots, L_{2m, r, r-1}$, while an 
extensive study of their properties is postponed until Chapter~\ref{chapter3}.

\subsection{Definition from fundamental splines}

The Hermite B-splines were first introduced in an attempt to extend the B-spline representation existing in the set 
$\mathscr{S}_n$ of splines of order $n$ to the set $\mathscr{S}_{n,r}$ of splines of order $n$ and multiplicity $r$.   
As was the case in the previous section, we continue to only consider even values of $n$ with $n=2m$.

\begin{deftn}[Hermite B-splines,{~\cite[Definition 1]{schoenberg_cardinal_1973}}]\label{def:hbsplines}
  The Hermite B-splines are the $r$ elements $N_{2m,r,0}, \ldots, N_{2m,r,r-1}$ of $\mathscr{S}_{2m,r}$ defined by 
  \begin{equation}\label{eq:def-hbsplines}
    N_{2m,r,s}(t) = \sum_{k=-(m-r)}^{m-r} c_k L_{2m,r,s}(t-k),
  \end{equation}
  with $c_k$ the coefficients of the Euler-Frobenius polynomial of multiplicity $r$ (Definition~\ref{def:EF-r}).
\end{deftn}

\begin{example}\label{ex:N0}
  Consider the case of multiplicity 1, \textit{i.e} $r=1$, in order to see how Hermite B-splines relate to $M_{2m}$, 
  B-splines of $S_{2m}$. As $M_{2m}$ is supported in $(-m,m)$, the sequence ${\bm{y} = {(M_{2m}(k))}_{k \in 
  \mathbb{Z}}}$ has at most $2m-1$ non zero elements and is hence bounded. From Theorem~\ref{thm:LH-gamma}, the 
  Lagrange-Hermite expansion for the bounded sequence $\bm{y}$ defines the unique bounded element in $\mathscr{S}_{2m,1} 
  = \mathscr{S}_{2m}$ that interpolates $\bm{y}$.  As $M_{2m}$ itself is one such element, unicity implies that this 
  Lagrange-Hermite expansion equals $M_{2m}$ everywhere, meaning that \begin{equation}\label{eq:expansion-M2m}
    \forall t \in \mathbb{R}, \qquad M_{2m}(t) = \sum_{k=-(m-1)}^{m-1} M_{2m}(k) L_0(t-k).
  \end{equation}
  From Definition~\ref{def:EF} of the Euler-Frobenius polynomial, we have that
  \begin{equation*}
    \Pi_{2m}(t) = (2m-1)! \sum_{k=-(m-1)}^{m-1} Q_{2m}(k+m) t^{k+m-1}.
  \end{equation*}
  However, from (\ref{eq:def-EF-r}), $(2m-1)!Q_{2m}(k+m)=c_k$. From (\ref{eq:cbspline}), ${Q_{2m}(k+m) = M_{2m}(k)}$.  
  Consequently, multiplying (\ref{eq:expansion-M2m}) by $(2m-1)! $ and replacing $(2m-1)!M_{2m}(k)$ by $c_k$ leads to
  \begin{equation*}
    \forall t \in \mathbb{R}, \quad (2m-1)!M_{2m}(t) = N_{2m,1,0}(t).
  \end{equation*}
\end{example}
The central B-spline $M_{2m}$ of order $2m$ is localized in the sense that it is supported within the compact set 
$[-m,m]$. Therefore, so is $N_{2m,1,0}$. The following lemma extends this observation to all Hermite B-splines of 
general multiplicity $r$.

\begin{lem}[{\cite[Lemma 2]{schoenberg_cardinal_1973}}]\label{lemma:supp-hbsplines}
  The Hermite B-splines $N_{2m,r,0}, \ldots, N_{2m,r,r-1}$ have their support in
  \begin{equation*}
    [-(m-r+1), m-r+1].
  \end{equation*}
  Moreover, for $s \in \llbracket0,r-1\rrbracket$, $N_{2m,r,s}$ has the same parity as $s$.
\end{lem}
The proof can be found in~\cite{schoenberg_cardinal_1973} and is reproduced in~\ref{proof:supp-hbsplines}. 

\subsection{Hermite B-splines form a basis}

The ``B'' in B-splines stands for basis. Therefore, the names of $N_{2m,r,0}, \ldots, N_{2m,r,r-1}$ suggest that they 
form a basis of some spline spaces together with their integer translates. This result is true, and we shall prove that, 
for $s \in \llbracket0,r-1\rrbracket$, $N_{2m,r,s}$ and its translates form a basis for the subspace
\begin{equation}\label{def:subspace-s}
  \mathscr{S}_{2m,r}^{(s)} = \{ S \in \mathscr{S}_{2m,r} | S^{(\rho)}(k) = 0, \rho \in \{0, \ldots, r-1\}\setminus 
  \{s\}, k \in \mathbb{Z}\}.
\end{equation}
Let $s \in \llbracket0,r-1\rrbracket$. Observe first that $N_{2m,r,s} \in \mathscr{S}_{2m,r}^{(s)}$. Indeed, combining 
(\ref{eq:fund-prop}) and (\ref{eq:def-hbsplines}) shows that
\begin{equation*}
  \forall \rho \in \llbracket0,r-1\rrbracket, \forall k \in \mathbb{Z}, \qquad N_s^{(\rho)}(k) = \delta_{s-\rho} 
  \sum_{l=-(m-r)}^{m-r} c_k \delta_{l-k}.
\end{equation*}
Thus, only derivatives of order $s$ of $N_{2m,r,s}$ may be non-zero on the integer grid. The definition 
(\ref{def:subspace-s}) indicates that $\mathscr{S}_{2m,r}^{(s)}$ is invariant by integer shift. Therefore, all integer 
translates of $N_{2m,r,s}$ are also in that subspace. The set $\mathscr{S}_{2m,r}^{(s)}$ being a linear space, any 
function $S$ of the form
\begin{equation}\label{eq:lc-hbsplines}
  S = \sum_{k=-\infty}^{\infty} c_{k} N_s(\cdot-k)
\end{equation}
is also in $\mathscr{S}_{2m,r}^{(s)}$, so that
\begin{equation*}
  \Span \{N_s(\cdot-k) | k \in \mathbb{Z} \} \subseteq \mathscr{S}_{2m,r}^{(s)}.
\end{equation*}
The function given by (\ref{eq:lc-hbsplines}) is well-defined for any sequence of coefficients since, at any real $t$,
the value $N_{2m,r,s}(t-k)$ does not vanish only for a finite number of integers $k$.  \\

It remains to be shown that any spline $S \in \mathscr{S}_{2m,r}^{(s)}$ can also be written as (\ref{eq:lc-hbsplines}) 
for a unique sequence $\bm{c}$. If that is the case, then $\{N_{2m,r,s}(\cdot-k) | k \in \mathbb{Z}\}$ is a basis for 
$\mathscr{S}_{2m,r}^{(s)}$ and the names ``Hermite B-splines'' is justified. In order to be able to prove that, 
Schoenberg and Sharma assume that the polynomial $\Pi_{2m,r}$ is irreducible over the rational field~\cite[Assumption 
1]{schoenberg_cardinal_1973}. This assumption is most probably true but is too complex compared to the arguments used so 
far and is therefore slightly unsatisfying. From there, one can then show that $\{N_{2m,r,s}(\cdot-k) | k \in 
\mathbb{Z}\}$ are locally linearly independent, meaning that that the relation 
\begin{equation*}
  \sum_{k=-(2m-2r+1)}^{0} c_k N_s(t-k) = 0, \quad  -(m-r+1) \leq t \leq -(m-r) 
\end{equation*}
can only hold if all the coefficients vanish as
\begin{equation*}
  c_{-(2m-2r+1)} = \cdots = c_0 = 0
\end{equation*}
It was however show~\cite{lee_b-splines_1975} that this assumption is not needed and the local linear independence always holds.
\begin{lem}[{\cite[Lemma 1]{lee_b-splines_1975}}]\label{lemma:lee73}
  For every $s \in \llbracket0, r-1\rrbracket$, the $2m-2r+2$ polynomials
  \begin{equation*}
    N_{2m,r,s}, N_{2m,r,s}(\cdot+1), \ldots, N_{2m,r,s}(\cdot+2m-2r+1)
  \end{equation*}
  are linearly independent over $\left(-(m-r+1), -(m-r)\right)$.
\end{lem}

The proof consists in establishing that the determinant of the matrix of an homogeneous system of equations is non-zero, 
which is quite technical. We refer to~\cite{lee_b-splines_1975} for details. \\

The following theorem can now be established and concludes our discussion.
\begin{thm}[{\cite[Theorem 3]{schoenberg_cardinal_1973}}]
  Every $S \in \mathscr{S}_{2m,r}^{(s)}$ admits a unique representation of the form
  \begin{equation}\label{eq:expansion-Ns}
    S = \sum_{k=-\infty}^{\infty} a_k N_{2m,r,s}(\cdot-k).
  \end{equation}
\end{thm}

\begin{proof} For $s\in \llbracket0,r-1\rrbracket$, let $N_s := N_{2m,r,s}$.
  \begin{enumerate}
    \item Let $S_0 \in \mathscr{S}_{2m,r}^{(s)}$ be such that $S_0(t) = 0$ for $t < 0$. Let $R_s$ be the forward Hermite 
      B-spline, \textit{i.e}, $R_s(t) = N_s(t-(m-r+1))$, so that $R_s$ is supported in $(0, 2m-2r+2)$. Let $a_0 = 
      \frac{S^{(s)}(1)}{R_s^{(s)}(1)}$. Consider the function
      \begin{equation*}
	S_1 = S_0 - a_0 R_s.
      \end{equation*}
      By construction, the function $S_1$ is in $\mathscr{S}_{2m,r}^{(s)} \subset \mathcal{C}^{2m-r-1}$ and vanish 
      identically on $(-\infty, 0)$.  Expanding it about 0 in its Taylor series shows that the order $2m$ polynomial 
      ${S_1}_{|(0,1)}$ can be written as
      \begin{equation*}
	S_1(t) = \sum_{k=2m-r}^{2m-1} \alpha_{k} t^k, \quad 0 < t < 1
      \end{equation*}
      From the definition (\ref{def:subspace-s}) of $\mathscr{S}_{2m,r}^{(s)}$, $S_1^{(\rho)}(1) = 0$ for $\rho=0, 
      \ldots, r-1, \rho \neq s$ but also $S_1^{(s)}(1) = 0$ by construction of $S_1$. These $r$ vanishing derivatives at 
      $1$ can only hold if the coefficients $\alpha_{k}$ vanish altogether, that is, $S_1$ vanishes on $(0,1)$.  
      Considering $a_1 = \frac{S_1^{(s)}(2)}{R_s^{(s)}(1)}$, and the function
      \begin{equation*}
	S_2 = S_1 - a_1 R_s(\cdot-1),
      \end{equation*}
      we see that $S_2$ vanishes on $(1,2)$ and by extension and $(-\infty, 2)$. Continuing in like manner, we have
      \begin{equation*}
	S_0 = \sum_{k=0}^{\infty} a_k R_s(\cdot-k),
      \end{equation*}
      with the coefficients ${(a_k)}_{k \geq 0}$ uniquely determined. \\
    \item Let $S \in \mathscr{S}_{2m,r}^{(s)}$. As 
      \begin{equation*}
	R_s(t), \ldots, R_s(t+2m-2r+1), \qquad 0 \leq t \leq 1,
      \end{equation*}
      are linearly independent on $(0,1)$ (Lemma~\ref{lemma:lee73}), there exist unique coefficients $a_{-2m+2r-1}, 
      \ldots, a_0$ such that
      \begin{equation*}
	S(t) = \sum_{k=-(2m-2r+1)}^{0} a_k R_s(t-k), \qquad 0 < t < 1,
      \end{equation*}
      Indeed, $S_{|(0,1)}$ is a polynomial of order $2m$ satisfying 
      \begin{equation*}
	S^{(\rho)}(0) = S^{(\rho)}(1) = 0, \qquad \rho=0, \ldots, r-1, \rho \neq s.
      \end{equation*}
      This make a total of $2r-2$ constraints. As a consequence, the set of all polynomial or order $2m$ satisfying 
      these constraints is a linear space of dimension $2m-2r+2$. As the $2m-2r+2$ B-splines $R_s$ are linearly 
      independent and satisfy the constraints, they span that linear space. \\ 

      Now, the function
      \begin{equation*}
	S_0(t) = S(t) - \sum_{k=-(2m-2r+1)}^{0} a_k R_s(t-k)
      \end{equation*}
      vanishes on $[0,1]$. Let,
      \begin{equation*}
	S_1(t) = \begin{dcases} S_0(t) &  \text{if} \ t \geq 1 \\
	  0 & \text{otherwise}
	\end{dcases},
	\quad
	S_2(t) = \begin{dcases} S_0(t) &  \text{if} \ t \leq 0 \\
	  0 & \text{otherwise}
	\end{dcases}.
      \end{equation*}
      Applying the first case shows that 
      \begin{equation*}
	S_1(t) = \sum_{k=1}^{\infty} a_k R_s(t-k), \quad S_2(t) = \sum_{k=-\infty}^{-(2m-2r+2)} a_k R_s(t-k)
      \end{equation*}
      for unique coefficients ${(a_k)}_k$. Putting all pieces together, and defining $b_k = a_{k-(m-r+1)}$ we have that
      \begin{equation*}
	\forall t \in \mathbb{R}, \quad S(t) = \sum_{k=-\infty}^{\infty} b_k N_s(t-k).
      \end{equation*}
  \end{enumerate}
\end{proof}

\section{Applications to bio-image analysis}

The past decade saw a multiplication in the number of images recorded and in the number of acquisition techniques.  As a 
consequence, the need for automatic image analysis tools grew significantly in many research communities. More 
precisely, shape-modelling is of particular interest for many purposes, going from image segmentation to interactive 
modelling of curves and surfaces. In all applications of shape-modelling there is a need for intuitive user interaction, 
which is best achieved when the representation is interpolatory. Domains related to interactive curves or shape modelling 
include computer graphics, biomedical imaging, industrial design, modelling of animated surfaces, etc.  Shape-modelling 
techniques can be categorized between discrete and continuous methods.  Discrete methods include polygonal meshes and 
subdivision schemes, which allow to locally refine shapes, but are poorly suited for theoretical modelling.  
Continuous-domain methods in contrast offer good theoretical properties. They usually consist of Bézier shapes or 
splines-based models. The most used splines-based method is the NURBS model, but it generally cannot be smooth and 
interpolatory at the same time. We refer to as \emph{active contour} a computational tool for detecting and outlining 
objects in digital images.  We hereafter present splines-based active contours models, with applications to biomedical 
imaging or computer graphics, that are practically convenient and theoretically powerful. \\

In an ideal setting, the basis functions generating the active contour are smooth, compactly supported, interpolatory 
and allow to vary the resolution of the constructed shape.  

\subsection{Modelling of 2D contours}\label{ssec:2d}

\subsubsection{Hermite polynomial}

In 2016, V.Uhlmann et al\@. published a paper~\cite{uhlmann_hermite_2016} that presents practical use-cases and good 
practical properties of the Hermite fundamental splines $L_0 := L_{4,2,0}$ and $L_1 := L_{4,2,1}$ for active contour 
modelling.  More precisely, if $r[0], \ldots, r[M-1]$, $r'[0], \ldots, r'[M-1]$ are $M$ measurements of the value and 
first derivative value of a M-periodic closed 2D curve, a good interpolatory model for the curve is the Lagrange-Hermite 
expansion~\ref{eq:LH}, that is,
\begin{align*}
  \forall t \in \mathbb{R}, \qquad r(t) &= \sum_{k=-\infty}^{\infty} r[k] L_0(t-k) + r'[k] L_1(t-k) \\
  &= \sum_{k=0}^{M-1} r[k] L_{0,per}(t-k) + r'[k] L_{1,per}(t-k) \end{align*}
with $L_{0,per}, L_{1,per}$ the M-periodizations of the functions $L_0, L_1$ respectively.  

\begin{remark}
  \begin{enumerate}
    \item In contrast to the expansion~\ref{eq:LH}, the terms in front of the fundamental splines are not scalars but 2D 
      vectors. The formula above is in fact short for the interpolation of each coordinate separately $(r_1(t), r_2(t)) 
      = r(t)$, \textit{i.e},
      \begin{align*}
	\forall t \in \mathbb{R}, \qquad r_1(t) &= \sum_{k=0}^{M-1} r_1[k] L_{0, per}(t-k) + r_1'[k] L_{1,per}(t-k) \\
	\qquad r_2(t) &= \sum_{k=0}^{M-1} r_2[k] L_{0,per}(t-k) + r_2'[k] L_{1,per}(t-k)
      \end{align*}
    \item In the vocabulary of active contour modelling, the $(r[k],r'[k])_{k \in \llbracket0,M-1\rrbracket}$ are called 
      the \emph{control points} at the \emph{knots} $(k)_{k \in \llbracket0,M-1\rrbracket}$ respectively.
    \item The set of all functions that are can be written as linear combinations of ${\{L_{4,2,0}(\cdot-k), 
      L_{4,2,1}(\cdot-k)\}}_{k \in \mathbb{Z}}$ are called \emph{cubic Hermite splines}.
  \end{enumerate}
\end{remark}

The functions $L_0$ and $L_1$ are explicitly given by
\begin{equation}
    L_0(t) = \begin{dcases}
      1-3|t|^2 + 2|t|^3 &\text{for } 0 \leq |t| \leq 1,\\
      0 &\text{for } 1 < |t|,\\
    \end{dcases} \quad
    \hfill
    L_1(t) = \begin{dcases}
      t(|t|^2-2|t|+1) &\text{for } 0 \leq |t| \leq 1,\\
      0 &\text{for } 1 < |t|.\\
    \end{dcases}
\end{equation}
The graphs of these functions are displayed in Figure~\ref{fig:fund-r2-m2}. They are by definition Hermite splines of 
order 4 and multiplicity $2$. On top of that, the functions $L_0$, $L_1$ are: compactly supported with support of size 
2; $\mathcal{C}^1$ continuous; interpolatory with multiplicity 2 (value and derivative value) at integers; capable of 
reproducing cubic and quadratic splines; a partition of unity; a Riesz basis with Riesz bounds $A, B = 
(210^{\frac{-1}{2}}, B)$~\cite{uhlmann_hermite_2016}.

\subsubsection{Hermite exponential}

In 2015, Conti et al\@. presented in~\cite{conti_ellipse-preserving_2015} practical uses of a non-polynomial Hermite 
active contour model. The study in the paper was motivated by the observation that usual spline-based deformable models  
were unable to reproduce shapes as elementary as ellipses. More precisely, a lot of control points are needed to 
satisfyingly approximate ellipses. In contrast, in~\cite{conti_ellipse-preserving_2015}, authors devised a new Hermite 
interpolation scheme that perfectly reproduce ellipses but also retain attractive properties of cubic Hermite splines.  
\\

Let $\varphi_1, \varphi_2$ be the new basis functions. In order to have a Hermite interpolatory scheme, it is enough to 
choose the basis functions $\varphi_1, \varphi_2$ so that
\begin{equation*}
  \forall k \in \mathbb{Z}, \forall \rho \in \{0,1\}, \qquad \varphi_1^{(\rho)}(k) = \delta_{\rho}\delta_k, \quad 
  \varphi_2^{(\rho)}(k) = \delta_{\rho-1}\delta_k.
\end{equation*}

In order to reproduce ellipses from $M$ control points, the space spanned by ${\{\varphi_0(\cdot-k), 
\varphi_1(\cdot-k)\}_{k \in \mathbb{Z}}}$ should include the cosinus and sinus functions at frequency $\frac{1}{M}$, 
\textit{i.e},
\begin{align*}
  \cos (wt) &= \sum_{k=-\infty}^{\infty} \cos (wk) \varphi_1(t-k) - w \sin (wk) \varphi_2(t-k), \\
  \sin (wt) &= \sum_{k=-\infty}^{\infty} \sin (wk) \varphi_1(t-k) + w \cos (wk) \varphi_2(t-k),
\end{align*}
with $w = \frac{2\pi}{M}$.

The following functions were found to satisfy all the requirements,
\begin{align*}
  \varphi_{1, w}(x) &=
  \begin{dcases}
    g_{1,w} = a_1(w) + b_1(w)x + c_1(w) e^{jwx} + d_1(w)e^{-jwx} &\text{for } x \geq 0 \\
    g_{1, w}(-x) &\text{for } x < 0\\
  \end{dcases}, \\
  \varphi_{2, w}(x) &=
  \begin{dcases}
    g_{2, w}(x) =  a_2(w) + b_2(w)x + c_2(w) e^{jwx} + d_2(w)e^{-jwx} &\text{for } x \geq 0 \\
    -g_{2, w}(-x) &\text{for } x < 0\\
  \end{dcases},
\end{align*}
with $a_1(w), \ldots, d_1(w), a_2(w), \ldots, d_2(w)$ as in~\cite[(2), (3), (4)]{conti_ellipse-preserving_2015}.

The graphs of these functions are displayed in Figure~\ref{fig:conti_phi_12}.
\begin{figure}[!h]
  \centering
  \includegraphics[width=\textwidth]{conti_phi_12.png}
  \caption{Graphs of $\varphi_{1,w}, \varphi_{2,w}$ for $w=\frac{2\pi}{3}$}\label{fig:conti_phi_12}
\end{figure}

The functions $\varphi_1, \varphi_2$ are exponential splines (Definition~\ref{def:exponential-spline}). On top of that, 
the functions $\varphi_1$, $\varphi_2$ are: compactly supported with support of size 2; $\mathcal{C}^1$ continuous; 
interpolatory with multiplicity 2 (value and derivative value) at integers; capable of reproducing cubic and quadratic 
exponential splines; a partition of unity; a Riesz basis~\cite{conti_ellipse-preserving_2015}.
 
\subsection{Modelling of 3D, sphere-like surfaces}

In order to represent surfaces, an additional continuous parameter is required. Compared to the Section~\ref{ssec:2d}, 
the 2D contour $r$ depending on the parameter $t$ is replaced by the 3D surface $\sigma$ depending on the parameters 
$(u,v)$. The sphere is an elementary surface that appears very often in biology. As a consequence, we shall require that 
our deformable models in 3D can satisfyingly approximate the sphere and surfaces that have a sphere-like topology.  A 
continuous  parametrization of the sphere is given by
\begin{equation*}
  \forall(u, v) \in {[0,1]}^2, \qquad \sigma(u,v) = \begin{bmatrix} \cos(2\pi u)\sin(\pi v) \\ \sin(2\pi u)\sin(\pi v) 
  \\ \cos(\pi v) \end{bmatrix}
\end{equation*}
The sphere has the good property of having a parametrization that is separable in $(u,v)$ i.e it can be written as the 
Schur product
\begin{equation*}
  \forall(u, v) \in {[0,1]}^2, \qquad \sigma(u,v) = \sigma_1(u) \otimes \sigma_2(v)
\end{equation*}
For such surfaces, the active contours of Section~\ref{ssec:2d} for 2D contours can be extended to 3D surfaces.  

\subsubsection{Hermite exponential}

Suppose that we have measurements ${(r[k])}_{k \in \llbracket0,M-1\rrbracket}, {(r'[k])}_{k \in 
\llbracket0,M-1\rrbracket}$ of a $M$-periodic closed 2D curve or an open 2D curve. Then, from Conti et al's work 
in~\cite{conti_ellipse-preserving_2015}, we know that the exponentials splines $\varphi_{1,w}, \varphi_{2,w}$ can be 
used to interpolate these measurements. The corresponding parametrization of the 2D curve is
\begin{equation*}
  \forall t \in \mathbb{R}, \qquad  r(t) = \sum_{k=0}^{M-1} r[k] \varphi_{1,w,per}(t-k) + r'[k] \varphi_{2,w,per}(t-k),
\end{equation*}
for closed curves, or 
\begin{align*}
  \forall t \in [0, M-1] \in \mathbb{R}, \qquad r(t) &= \sum_{k=-\infty}^{\infty} r[k] \varphi_{1,w}(t-k) + r'[k] 
  \varphi_{2, w}(t-k) \\
  & = \sum_{k=0}^{M-1} r[k] \varphi_{1,w}(t-k) + r'[k] \varphi_{2, w}(t-k),
\end{align*}
for open curves. 

Suppose that we have measurements of a 3D surface at $M_1$ locations on latitudes and $M_2+1$ location on meridians.  
Let's see what measurements we need exactly to represent a sphere-like surface. Let $w_1 = \frac{2\pi}{M_1}, w_2 = 
\frac{\pi}{M_2}$.  We know that, by construction, $\varphi_{1,w}, \varphi_{2,w}$ can exactly reproduce ellipses at at 
pulsation $w$, \textit{i.e},
\begin{align*}
  \forall u \in [0, M_1] \quad \cos(w_1u) &= \sum_{k=-\infty}^{\infty} \cos (w_1k) \varphi_{1, w_1}(u-k) - w_1 \sin (w_1k) 
  \varphi_{2, w_1} (u-k) \\
  \forall v \in [0, 2M_2] \quad \cos(w_2v) &= \sum_{l=-\infty}^{\infty} \cos (w_2l) \varphi_{1, w_2}(v-l) - w_2 \sin (w_2l) 
\varphi_{2, w_2} (v-l) \end{align*}
For implementation purposes, the continuous parameters are normalized to lie on $[0,1]$ so that,
\begin{align*}
  \forall u \in [0, 1] \quad \cos(2\pi u) &= \sum_{k=-\infty}^{\infty} \cos (w_1k) \varphi_{1, w_1}(M_1u-k) - w_1 \sin 
  (w_1k) \varphi_{2, w_1} (M_1u-k) \\
  \forall v \in [0, 2] \quad \cos(\pi v) &= \sum_{l=-\infty}^{\infty} \cos (w_2l) \varphi_{1, w_2}(M_2v-l) - w_2 \sin 
  (w_2l)
\varphi_{2, w_2} (M_2v-l)
\end{align*}

Similar parametrizations hold for the sinus function but are omitted here. As a consequence, the four basis functions 
$\varphi_{1,w_1}, \varphi_{2,w_1}, \varphi_{1,w_2}, \varphi_{2,w_2}$ should allows us to perfectly reproduce a sphere, 
\textit{i.e},
\begin{align*}
  \forall (u, v) \in {[0,1]}^2, \qquad \sigma(u,v) &= \sum_{k=0}^{M_1-1} \sum_{l=0}^{M_2} c_1[k,l]\varphi_{1, w_1, 
  per}(M_1u-k)\varphi_{1, w_2}(M_2v-l) \\
  &+ \sum_{k=0}^{M_1-1} \sum_{l=0}^{M_2} c_2[k,l] \varphi_{1, w_1, per}(M_1u-k)\varphi_{2, w_2}(M_2v-l) \\
  &+ \sum_{k=0}^{M_1-1} \sum_{l=0}^{M_2} c_3[k,l] \varphi_{2, w_1, per}(M_1u-k)\varphi_{1, w_2}(M_2v-l) \\
  &+ \sum_{k=0}^{M_1-1} \sum_{l=0}^{M_2} c_4[k,l] \varphi_{2, w_1, per}(M_1u-k)\varphi_{2, w_2}(M_2v-l) \\
\end{align*}
where,
{\footnotesize
\begin{align*}
  c_1[k,l]=\begin{bmatrix} \cos(w_1k)\sin(w_2l) \\ \sin(w_1k)\sin(w_2l) \\ \cos(w_2l) \end{bmatrix} &= 
  \sigma(\frac{k}{M_1},\frac{l}{M_2}) & c_2[k,l]=\begin{bmatrix} w_2\cos(w_1k)\cos(w_2l) \\ w_2\sin(w_1k)\cos(w_2l) \\ 
  -w_2\sin(w_2l) \end{bmatrix} &= \frac{1}{M_2}\frac{\partial \sigma}{\partial v}(\frac{k}{M_1}, \frac{l}{M_2}) \\
  c_3[k,l]=\begin{bmatrix} -w_1\sin(w_1k)\sin(w_2l) \\ w_1\cos(w_1k)\sin(w_2l) \\ 0 \end{bmatrix} &= \frac{1}{M_1} 
  \frac{\partial \sigma}{\partial u}(\frac{k}{M_1}, \frac{l}{M_2}) &
  c_4[k,l]=\begin{bmatrix} -w_1w_2\sin(w_1k)\cos(w_2l) \\ w_1w_2\cos(w_1k)\cos(w_2l) \\ 0 \end{bmatrix} &= \frac{1}{M_1 
  M_2} \frac{\partial^2 \sigma}{\partial u \partial v}(\frac{k}{M_1}, \frac{l}{M_2}),
\end{align*}
}%
and,
\begin{align*}
    \varphi_{1,w_1,per} &= \sum_{k=-\infty}^{\infty} \varphi_{1,w_1}(\cdot-M_1k), & \varphi_{1,w_2,per} &= 
    \sum_{k=-\infty}^{\infty} \varphi_{1,w_2}(\cdot-2M_2k), \\
  \varphi_{2,w_1,per} &= \sum_{k=-\infty}^{\infty} \varphi_{2,w_1}(\cdot-M_1k), & \varphi_{2,w_2,per} &= 
  \sum_{k=-\infty}^{\infty} \varphi_{2,w_2}(\cdot-2M_2k). \\
\end{align*}



\subsection{The problem of cross-derivatives}
