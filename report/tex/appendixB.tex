\chapter{The splines zoo}\label{chapter:annexB}

This appendix aims at putting in one place all the splines that we have encountered in the litterature in order to catch 
a glimpse of their broad diversity and in order to facilitate comparisons. If relevant, connections will be made between 
the different notations and definitions.  In order to have a presentation as coherent as possible, we will stick to the 
following notation rules.

\begin{enumerate}
  \item $n$ or $m$ related quantities denote the \emph{order} of the spline, not to be confused with the \emph{degree}.  
  \item $r$ denotes the multiplicity of the interpolation or in other words the order up to which derivatives are 
    reproduced.
  \item $j,s$ are used for the running index from $0$ to $r-1$
  \item $i,k,l$ are used for running integers
  \item $j$ is used for the purely imaginary number
  \item Calligraphed letters as $\mathscr{S}$ or $\mathcal{C}$ denote functional sets
  \item Bold lowercase letters as $\bm{c}$ denote sequences or vectors
  \item Bold uppercase letters as $\bm{M}$ denote matrices
\end{enumerate}


\section{Polynomial splines}

\subsection{General B-splines}

\begin{deftn}[\cite{Sch73}, Lecture 1, p2]
  The general B-spline of order $n$ with knots $t_0 < \cdots < t_n$ is
  \begin{equation}\label{def:sch-gen-bspline}
    M(t;t_0, \ldots, t_n) = n[t_0, \ldots, t_n]{(\cdot - t)}_+^{n-1}
  \end{equation}
\end{deftn}

\begin{prop}
  \begin{enumerate}
    \item (Peano's theorem) For any $f \in \mathcal{C}^n$, 
      \begin{equation*}
	[t_0, \ldots, t_n]f = \frac{1}{n!}\int_{t_0}^{t_n} M(t;t_0, \ldots, t_n)f^{(n)}(t)dt
      \end{equation*}
  \end{enumerate}
\end{prop}

\subsection{Cardinal splines}

\begin{deftn}[B-splines equidistant knots,~\cite{Sch73}, Lecture 2]
  The forward $B$-spline of order $n$ is
  \begin{equation*}
    Q_n(t) = n[0, 1, \ldots, n]{(\cdot-t)}_{+}^{n-1}
  \end{equation*}
  and the central $B$-spline of order $n$
  \begin{equation*}
    M_n(t) = n\left[\frac{-n}{2}, \frac{-n}{2}+1, \ldots, \frac{n}{2}\right]{(\cdot-t)}_{+}^{n-1} = Q_n(t+\frac{n}{2})
  \end{equation*}
\end{deftn}

\subsection{Exponential splines}

\begin{deftn}[Exponential spline,~{\cite[(4.15), Lecture 2]{Sch73}}]
  Let $t \neq 0, t \neq 1$. The exponential spline of degree $n$ for the base $t$ is the function defined by
  \begin{equation}
    \Phi_n(x;t) = \sum_{k=-\infty}^{\infty} t^k Q_{n+1}(x-k)
  \end{equation}
  It belongs to the space $\mathscr{S}_{n+1}$.
\end{deftn}

\begin{prop}
  The exponential spline satisfies
  \begin{enumerate}
    \item $\Phi_n(x+1;t) = t\Phi_n(x;t)$, $\forall x,t$
    \item $\Phi_n^{(n)}(x;t) = {(1-t^{-1})}^n$, for $0 < x < 1$
  \end{enumerate}
\end{prop}

\begin{deftn}[Euler-Frobenius polynomial,~{\cite[(1.7), Lecture 3]{Sch73}}]\label{def:EF}
  The $n^{th}$ Euler-Frobenius polynomial is
  \begin{equation}
    \Pi_n(t) = n! \sum_{j=1}^n Q_n(j)t^{j-1}
  \end{equation}
  It belongs to the linear space $\Pi_{<n}$ and has $n-1$ simple zeros in reciprocal pairs
  \begin{equation*}
    \lambda_{n-1} < \cdots < \lambda_1 < 0
  \end{equation*}
\end{deftn}

\begin{deftn}[Exponential Euler polynomial,~{\cite[(1.2), Lecture 3]{Sch73}}]\label{def:EE}
  The exponential Euler polynomial of degree $n$ is
  \begin{equation}
    A_n(x;t) = \frac{n!}{{(1-t^{-1})}^n} \Phi_n(x;t), \ 0\leq x\leq 1
  \end{equation}
\end{deftn}

\begin{prop}\label{prop:EF}
  The exponential Euler polynomial satisfies
  \begin{align*}
    \frac{t-1}{t-e^z}e^{xz} &= \sum_{n=0}^{\infty} \frac{A_n(x;t)}{n!}z^n \\
    A_n(x;t) &= x^n + a_1(t) \binom{n}{1} + \cdots + a_n(t), \quad a_n(t) = \frac{\Pi_n(t)}{{(t-1)}^n} \end{align*}
\end{prop}

\begin{deftn}[Exponential Euler spline,~{\cite[(5.4), Lecture 3]{Sch73}}]
  The exponential Euler spline of degree $n$ for base $\lambda$ is
  \begin{equation}
    S_n(x;\lambda) = \frac{\Phi_n(x;\lambda)}{\Phi_n(0;\lambda)} \ \text{for} \ \lambda \not\in\{\lambda_1,\ldots, 
    \lambda_{n-1}\}
  \end{equation}
  It is in $\mathscr{S}_{n+1}$ and has the property of interpolating $\lambda^x$ at integers.
\end{deftn}

\begin{deftn}[Eigensplines,~{\cite[(5.6), Lecture 3]{Sch73}}]
  The eigenspline for base $\lambda_j$ is
  \begin{equation}
    S_j(x) = \Phi_n(x;\lambda_j)
  \end{equation}
  It is in $\mathscr{S}_{n+1}$ and has the property of vanishing at integers.
\end{deftn}

\subsection{Eigensplines \texorpdfstring{$S_1, \ldots, S_{2(m-r)}$}{Lg}, fundamental splines \texorpdfstring{$L_0, 
\ldots, L_{r-1}$}{Lg}}

Let $\matr{P} = {(\binom{i}{j})}_{(i,j) \in \mathbb{Z}^2}$ be an infinite matrix with binomial coefficients and let 
$P\left(\begin{matrix} i_1 & i_2 & \hdots & i_q \\ j_1 & j_2 & \hdots & j_q \end{matrix}; \lambda \right)$ the 
determinant of the submatrix of $\matr{P}-\lambda\matr{I}$ obtained by deleting all rows and columns except $i_1, 
\ldots, i_q$ and $j_1, \ldots, j_q$ respectively.

\begin{deftn}[Euler-Frobenius polynomial multiplicity $r$,~{\cite[(3.10), Lecture 5]{Sch73}}]\label{def:EF-r}
  The $n^{th}$ Euler-Frobenius of multiplicity $r$ is the polynomial
  \begin{equation}\label{eq:def-EF-r}
    \Pi_{n,r}(\lambda) = P\left(\begin{matrix} r & r+1 & \hdots & n \\ 0 & 1 & \hdots & n-r \end{matrix}; \lambda 
    \right)
    \end{equation}
    It belongs to the linear space $\Pi_{<n-2r+2}$.
\end{deftn}

\begin{remark}
  \begin{enumerate}
    \item The $n^{th}$ Euler-Frobenius polynomial of multiplicity $r$ is the determinant of the following matrix whose 
      coefficients depend on $\lambda$ \begin{equation*}
	\begin{vmatrix} 1 & \binom{r}{1} & \hdots & \binom{r}{r-1} & 1- \lambda & 0  & \hdots &  0 \\
	  1 & \binom{r+1}{1} & \hdots & \binom{r+1}{r-1} & \binom{r+1}{r} & 1- \lambda &  \hdots & 0 \\
	  \vdots & & & & & & & \vdots \\
	  1 & \binom{n-r}{1} & & & \hdots &  & \binom{n-r}{n-r-1} &  1- \lambda  \\
	  1 & \binom{n-r+1}{1} & & & \hdots &  & \binom{n-r+1}{n-r-1} &  \binom{n-r+1}{n-r}  \\
	  \vdots & & & & & & & \vdots \\
	  1 & \binom{n}{1} & & & \hdots & & \binom{n}{n-r-1} & \binom{n}{n-r}  \\
	\end{vmatrix}
      \end{equation*}
    \item The $n^{th}$ Euler Frobenius polynomial of multiplicity $r=1$ is the $n^{th}$ Euler Frobenius from 
      Definition~\ref{def:EF} i.e
      \begin{equation*}
	\Pi_{n,1} = \Pi_n
      \end{equation*}
  \end{enumerate}
\end{remark}

It is assumed from now on that $n=2m-1$ is odd.
\begin{prop}[{\cite[Theorem 4]{Sch73}}]\label{prop:EF-r}
  The $(2m-1)^{th}$ Euler-Frobenius polynomial for complicity $r$ is such that
  \begin{equation}
    \Pi_{2m-1,r}(\lambda) = \sum_{k=0}^{2m-2r}   c_{k-(m-r)}\lambda^k
  \end{equation}
  with
  \begin{equation*}
  c_0 > 0, c_{-k} = c_k, c_{-(m-r)} = c_{m-r} = \pm 1
  \end{equation*}
  It is therefore reciprocal and monic, except for the sign. Moreover, it has $n-2r+1 = 2(m-r)$ real simple zeros of 
  sign ${(-1)}^r$
  \begin{equation*}
    0 < |\lambda_1 | < \cdots < |\lambda_{m-r}| < 1 < |\lambda_{m-r+1}| < \cdots < |\lambda_{2(m-r)}|
  \end{equation*}
  which are reciprocal in pairs i.e
  \begin{equation*}
    \lambda_j \lambda_{2m-2r+1-j} = 1, \quad j=1, \ldots, m-r
  \end{equation*}
\end{prop}

\begin{deftn}[Eigensplines for muliplicity $r$,~\cite{Sch73}, Lecture 5]\label{def:eigsplines-r}
  Let $j \in \llbracket1,2(m-r)\rrbracket$. The $j^{th}$ eigenspline for multiplicity $r$ is the spline defined by
  \begin{equation*}
    \begin{dcases}
      S_j(x) = a_{0,j}x^{2m-1} + \binom{2m-1}{1} a_{1,j} x^{2m-2} + \cdots + \binom{2m-1}{2m-r-1} a_{2m-r-1, j}x^r, & 
      \text{if} \ 0 \leq x \leq 1 \\
      S_j(x+1) = \lambda_j S_j(x) & \text{elsewhere}
    \end{dcases}
  \end{equation*}
  with $a_{0,j}, \ldots, a_{2m-r-1,j}$ the unique solution to the system formed of the homogeneous equations 
  (\ref{eq:def-EF-r}) at $\lambda = \lambda_j$ and the equation $S_j^{(r)}(0) = 1$. 
  
  It belongs to the linear space $\mathring{\mathscr{S}}_{2m,r}$.
\end{deftn}

\underline{Examples}
\begin{enumerate}
  \item Let $r=1$. If $m=1$, $\dim \mathring{\mathscr{S}}_{2m,r} = 0$ and the only eigenspline is the trivial 
      function.
      If $m=2$, $P = S_{|[0,1]}$ takes the form \begin{equation*}
	P = a_0 x^3 + 3 a_1 x^2 + 3a_2 x
      \end{equation*}
      with homogeneous system
      \begin{equation*}
      \begin{array}{lcl}
       a_0 + 3a_1 + 3a_2 & = & 0 \\
       3a_0 + 6a_1 + 3(1-\lambda)a_2 & = & 0 \\
       6a_0 + 6a_1(1-\lambda) & = & 0
      \end{array}
      \end{equation*}
     
      $\lambda$ is chosen so that the matrix of the system in singular in order for the eigenspline not to be trivial.  
      Coefficients are then determined up to constant which is fixed by the following constraint $S^{(1)}(0) = 1$, 
      adding the equation
      \begin{equation*}
       3a_2 = 1
      \end{equation*}
      
      The matrix of the homogeneous system is singular if and only if $\lambda$ is a zero of \begin{equation*}
       \Pi_{2m, r}(\lambda) = 1+4\lambda+\lambda^2
      \end{equation*}
      
      that is $\lambda_1 = -2 + \sqrt{3}, \lambda_2 = -2 - \sqrt{3}$.

      \begin{figure}[!h]
	\centering
	\includegraphics[width=\textwidth]{eigsplines_r1_m2.png}
	\caption{Eigensplines for $r=1$, $m=2$}
      \end{figure}
    \item Let $r=2$. If $m=2$, $\dim \mathring{\mathscr{S}}_{2m,r} = 0$ and the only eigenspline is the trivial 
      function. If $m=3$, $P = S_{|[0,1]}$ takes the form   
      \begin{equation*}
          P = a_0 x^5 + 5 a_1 x^4 + 10a_2 x^3 + 10a_3 x^2
      \end{equation*}
      with homogeneous system
      \begin{equation*}
      \begin{array}{lcl}
       a_0 + 5a_1 + 10a_2 + 10a_3 & = & 0 \\
       5a_0 + 20a_1 + 30a_2 + 20a_3 & = & 0 \\
       20a_0 + 60a_1 + 60a_2 + 20a_3(1-\lambda) & = & 0 \\
       60a_0 + 120a_1 + 60a_2(1-\lambda) & = & 0
      \end{array}
      \end{equation*}
	   
      $\lambda$ is chosen so that the matrix of the system in singular in order for the eigenspline not to be trivial.  
      Coefficients are then determined up to constant which is fixed by the following constraint $S^{(2)}(0) = 1$, 
      adding the equation
      \begin{equation*}
       20a_3 = 1
      \end{equation*}
      
      The matrix of the homogeneous system is singular if and only if f $\lambda$ is a zero of \begin{equation*}
       \Pi_{2m, r}(\lambda) = 1-6\lambda+\lambda^2
      \end{equation*}
      that is $\lambda_1 = 3-2\sqrt{2}, \lambda_2 = 3+2\sqrt{2}$.

      \begin{figure}[!h]
	\centering
	\includegraphics[width=\textwidth]{eigsplines_r2_m3.png}
	\caption{Eigensplines for $r=2$, $m=3$}
      \end{figure}
\end{enumerate}

\begin{prop}[\cite{Sch73}, p49]
  To every $S \in \mathring{\mathscr{S}}_{2m,r}$ corresponds a unique sequence $(c_1, \ldots, c_{2m-2r})$ such that
  \begin{equation}
    S = \sum_{j=1}^{2m-2r} c_j S_{j}
  \end{equation}
\end{prop}

\begin{deftn}[Fundamental splines,{~\cite[(5.1)-(5.5), Lecture 5]{Sch73}}]\label{def:fundamental-r}
  Let $s \in \llbracket 0, r-1\rrbracket$. The ${s+1}^{th}$ \emph{fundamental} splines $L_s := L_{2m,r,s}$ is defined by  
  \begin{equation}
    L_s(t) = 
    \begin{dcases} 
      P_s(t) & \text{if} \ 0 \leq t \leq 1 \\
      \sum_{j=1}^{m-r} c_{j,s} S_j(t) & \text{if} \ t \geq 1 \\
      {(-1)}^s L_s(-t) & \text{if} \ t < 0
    \end{dcases}
  \end{equation}
  with~(\cite[(7.13)]{LS73})
  \begin{equation}
    P_s(t) = \begin{dcases} \frac{1}{s!}t^s + a_{1,s}t^r + a_{2,s}t^{r+2} + \cdots + a_{m-r+1,s}t^{2m-r} \\ \quad +  
      a_{m-r+2,s} t^{2m-r+1} + \cdots + a_{m,s} t^{2m-1} & \text{if $r-s \equiv 0[2]$} \\
      \frac{1}{s!}t^s + a_{1,s}t^{r+1} + a_{2,s}t^{r+3} + \cdots + a_{m-r,s}t^{2m-r-1} \\
      \quad + a_{m-r+1,s} t^{2m-r} + \cdots + a_{m,s} t^{2m-1} & \text{otherwise}
   \end{dcases}
  \end{equation}
  and the $2m-r$ unknowns $a_{1,s}, \ldots, a_{m,s}, c_{1,s}, \ldots, c_{m-r,s}$ defining $L_s$ are obtained as the 
  unique solution to the linear system of $2m-r$ equations
  \begin{equation}
   \forall \rho=0, \ldots, 2m-r-1, \quad P_s^{(\rho)}(1) =\sum_{j=1}^{m-r} c_{j,s} S_j^{(\rho)}(1)
  \end{equation}
\end{deftn}

\underline{Examples}
\begin{enumerate}
  \item If $r=1, m=1$, $L_0$ is even, has compact support in $[-1,1]$ with \begin{equation}
      L_0(t) = \begin{cases} 1 + a_{1,0} t & \text{if} \ 0 \leq t \leq 1 \\
	0 & \text{if} \ t \geq 1
      \end{cases}
    \end{equation}
    and the coefficient $a_{1,0}$ satisfies
    \begin{equation}
     1 + a_{1,0} = 0
    \end{equation}
    that is $L_0$ is the hat function.
    \begin{figure}[!h]
      \centering
      \includegraphics[width=\textwidth]{fundspline_r1_m1.png}
      \caption{Fundamental spline for $r=1$, $m=1$}
    \end{figure}
    
  \item If $r=1, m=2$, $L_0$ is even, infinitely supported with
    \begin{equation*}
      L_0(t) =
      \begin{dcases}
	1 + a_{1,0} t^2 + a_{2,0}t^3 & \text{if} \ 0 \leq t \leq 1  \\
	c_{1,0}S_1(t) & \text{if} \ t \geq 1  \\
      \end{dcases}
    \end{equation*}
    and the coefficients satisfy
    \begin{equation}
      \begin{array}{rcl}
       a_{1,0} + a_{2,0} - c_{1,0} S_1(1) & = & -1  \\
       2a_{1,0} + 3a_{2,0} - c_{1,0} S_1^{(1)}(1) & = & 0 \\
       2a_{1,0} + 6a_{2,0} - c_{1,0} S_1^{(2)}(1) & = & 0
      \end{array}
    \end{equation}
    \begin{figure}[!h]
      \centering
      \includegraphics[width=\textwidth]{fundspline_r1_m2.png}
      \caption{Fundamental spline for $r=1$, $m=2$}
    \end{figure}

  \item If $r=2, m=2$, $L_0, L_1$ are compactly supported in $[-1,1]$
    \begin{align*}
      L_0(t) &= 
      \begin{cases}
	 1 + a_{1,0} t^2 + a_{2,0}t^3 & \text{if} \ 0 \leq t \leq 1 \\
         0 & \text{if} \  t \geq 1 \\
      \end{cases} \\
      L_1(t) &= 
      \begin{cases}
         t + a_{1,1} t^2 + a_{2,1}t^3 & \text{if} \ 0 \leq t \leq 1 \\
         0 & \text{if} \  t \geq 1 \\
      \end{cases} \\
    \end{align*}
    and the coefficients satisfy
    \begin{equation*}
     \begin{array}{rcl}
     a_{1,0} + a_{2,0} & = & -1  \\
     2a_{1,0} + 3a_{2,0} & = & 0 \\
    \end{array}
    \end{equation*}
    and 
    \begin{equation*}
    \begin{array}{rcl}
     a_{1,1} + a_{2,1} & = & -1  \\
     2a_{1,1} + 3a_{2,1} & = & -1 \\
    \end{array}
    \end{equation*}
    \begin{figure}[!h]
      \centering
      \includegraphics[width=\textwidth]{fundspline_r2_m2.png}
      \caption{Fundamental splines for $r=2$, $m=2$}\label{fig:fund-r2-m2}
    \end{figure}

  \item If $r=2, m=3$, $L_0, L_1$ are infinitely supported with
    \begin{align*}
      L_0(t) &= 
     \begin{cases}
        1 + a_{1,0} t^2 + a_{2,0}t^4 + a_{3,0}t^5 & \text{if} \ 0 \leq t \leq 1 \\
	c_{1,0} S_1 & \text{if} \  t \geq 1 \\
      \end{cases} \\
     L_1(t) &= 
     \begin{cases}
        t + a_{1,1} t^3 + a_{2,1}t^4 + a_{3,1}t^5 & \text{if} \ 0 \leq t \leq 1 \\
	c_{1,1} S_1 & \text{if} \  t \geq 1 \\
      \end{cases}
    \end{align*}
    and the coefficients satisfy
    \begin{equation*}
      \begin{array}{rcl}
       a_{1,0} + a_{2,0} + a_{3,0} - c_{1,0} S_1(1) & = & -1  \\
       2a_{1,0} + 4a_{2,0} + 5a_{3,0} - c_{1,0} S_1^{(1)}(1) & = & 0 \\
       2a_{1,0} + 12a_{2,0} + 20a_{3,0} - c_{1,0} S_1^{(2)}(1) & = & 0 \\
       24a_{2,0} + 60a_{3,0} - c_{1,0} S_1^{(3)}(1) & = & 0 
      \end{array}
    \end{equation*}
    and 
    \begin{equation*}
      \begin{array}{rcl}
       a_{1,1} + a_{2,1} + a_{3,1} - c_{1,1} S_1(1) & = & -1  \\
       3a_{1,1} + 4a_{2,1} + 5a_{3,1} - c_{1,1} S_1^{(1)}(1) & = & -1 \\
       6a_{1,1} + 12a_{2,1} + 20a_{3,1} - c_{1,1} S_1^{(2)}(1) & = & 0 \\
       a_{1,1} + 24a_{2,1} + 60a_{3,1} - c_{1,1} S_1^{(3)}(1) & = & 0 
      \end{array}
    \end{equation*}
    \begin{figure}[!h]
      \centering
      \includegraphics[width=\textwidth]{fundspline_r2_m3.png}
      \caption{Fundamental splines for $r=2$, $m=3$}
    \end{figure}
\end{enumerate}


\subsection{Exponential Euler-Hermite splines}

Let $s \in \llbracket0,r-1\rrbracket$. Define
\begin{equation}\label{eq:def-anrs}
  A_{n,r,s}(x;\lambda) = \begin{vmatrix}
    \frac{A_n(0;\lambda)}{n!} & \hdots & \frac{A_{n-s+1}}{(n-s+1)!} & \frac{A_n(x;\lambda)}{n!} & \hdots & 
    \frac{A_{n-r+1}(0;\lambda)}{(n-r+1)!} \\
    \vdots & & & \vdots & & \vdots \\
    \frac{A_{n-r+1}(0;\lambda)}{(n-r+1)!} & \hdots & \frac{A_{n-r-s+2}}{(n-r-s+2)!} & \frac{A_n(x;\lambda)}{(n-r+1)!} & 
    \hdots & \frac{A_{n-2r+2}(0;\lambda)}{(n-2r+2)!} \\
  \end{vmatrix}
\end{equation}

As $\frac{A_n'(x;\lambda)}{n!} = \frac{A_{n-1}(x;\lambda)}{(n-1)!}$, one has that 
\begin{equation}\label{eq:anrs}
  A^{(s)}_{n,r,s}(0; \lambda) = H_r\left(\frac{A_n(0;\lambda)}{n!}\right)
\end{equation}
with  $A_n$ the exponential Euler polynomial of degree $n$ (Definition~\ref{def:EE}) and $H_r$ the Hankel determinant of 
order $r$ (Definition~\ref{def:Hankel}).

\begin{prop}
  \begin{align*}
    A_{n,r,s}^{(\rho)}(1;\lambda) &= \lambda A_{n,r,s}^{(\rho)}(0;\lambda), \quad \rho=r, \ldots, n-r  \\
    A_{n,r,s}^{(\rho)}(1;\lambda) &= A_{n,r,s}^{(\rho)}(0;\lambda) = 0, \quad \rho=0, \ldots, r-1, \rho \neq s \\
    A_{n,r,s}^{(s)}(0;\lambda) &= H_r(\frac{A_n(0;\lambda)}{n!})
  \end{align*}
\end{prop}

\begin{deftn}[Exponential Euler-Hermite spline,{\cite[(2.1)]{Lee76a}}]\label{def:EEH}
  Let $s \in \llbracket0, r-1\rrbracket$ and $\lambda \in \mathbb{R}$ such that $\Pi_{n,r}(\lambda) \neq 0$. The 
  ${s+1}^{th}$ exponential Euler-Hermite spline of order $n+1$ for the base $\lambda$ is the function $S_{n+1,r,s}$ 
  defined by \begin{align}
    S_{n+1,r,s}(x) &= \frac{A_{n,r,s}(x; \lambda)}{A_{n,r,s}^{(s)}(0; \lambda)}, \quad 0 \leq x \leq 1 \\
    S_{n+1,r,s}(x+1) &= \lambda S_{s}(x), \quad \forall x \in \mathbb{R}
  \end{align}
  It belongs to the linear space $\mathscr{S}_{n+1,r}$.
\end{deftn}

\begin{prop}[{\cite[(2.2)]{Lee76a}}]\label{prop:snrs}
  For $k \in \mathbb{Z}$,
  \begin{align*}
    S_{n+1,r,s}^{(\rho)}(k) &= 0, \quad \rho=0, \ldots, r-1, \rho \neq s \\
    S_{n+1,r,s}^{(s)}(k) &= \lambda^k, \quad \rho=0, \ldots, r-1, \rho \neq s
  \end{align*}
  hence $S_{n+1,r,s} \in \mathscr{S}_{n+1,r}^{(s)}$ (\ref{def:subspace-s}).
\end{prop}

\subsection{Hermite B-splines}

\section{Exponential splines}

In the first part of this zoo, a lot of different splines and splines basis were introduced, all of which were piecewise 
polynomials. However, it is not necessary to restrict ourselves to polynomials to fulfill all the good properties of 
splines. Other types of splines, more general, exist and provide additional tools in all areas where polynomials were 
already in use, but also in areas where their impact was less significant. More specifically, polynomial splines didn't 
develop much in continuous-time signal processing for the reason that the most prominent functions in this domain are 
the exponentials, not the polynomials. This observation motivated the study of an enlarged class of splines, namely the 
class of \emph{exponential} splines, that contributed a lot to the unification between continuous and discrete-time 
approaches. The kind of splines that are the most convenient are the \emph{cardinal} ones, which are defined on a 
uniform grid. We shall thefore introduce hereafter the theory of \emph{cardinal exponential splines}, following the 
works of Unser~\cite{unser_cardinal_2005},~\cite{unser_cardinal_2005-1}.

\subsection{Green's function of differential operator}
In order to define exponential splines, it is necessary to define what the \emph{Green's} function of a differential 
operator, $L$, is. Let $n \in \mathbb{N}^*$, $L$ be the generic differential operator of order $n$ \begin{equation*}
  L\{f\} = D^n{f} + a_{n-1}D^{n-1}{f} + \cdots + a_0 I{f}
\end{equation*}
with $a_0, \ldots, a_{n-1} \in \mathbb{C}$. Formally, $L$ is an operator from the Schwartz space $\mathcal{S} := 
\mathcal{S}(\mathbb{R})$ into its topological dual $\mathcal{S}' := \mathcal{S}'(\mathbb{R})$, also known as the space 
of \emph{distributions} or \emph{generalized functions} on $\mathbb{R}$. Let $\langle\cdot,\cdot\rangle: 
\mathcal{S}\times\mathcal{S}' \to \mathbb{C}$ that associates to each pair $(\varphi, f) \in 
\mathcal{S}\times\mathcal{S}'$ the complex number $\langle \varphi, f \rangle := f\{\varphi\}$. Let $T_s$ the 
translation operator by $s$, \textit{i.e}, for any $\varphi \in \mathcal{S}$, $T_s\varphi(t) = \varphi(t-s)$. $L$ is a 
continuous, linear and shift-invariant (LSI) operator, where shift-invariance means that
\begin{equation*}
  \forall \varphi \in \mathcal{S}, \forall s \in \mathbb{R}, \qquad L\{T_s\varphi\} = T_sL\{\varphi\},
\end{equation*}
From Schwartz kernel theorem~\cite[Corollary 3.3]{unser_introduction_2014}, we know there exists a generalized function 
$h \in \mathcal{S}'$, called convolution kernel of $L$, such that
\begin{equation*}
  \forall \varphi \in \mathcal{S}, \qquad L\{\varphi\} = \varphi * h
\end{equation*}

\begin{remark}
  \begin{enumerate}
    \item The convolution $\varphi * f$ is also noted $f * \varphi$, mirroring the commutativity of the convolution 
      product on $\mathcal{S}$.
    \item Formally, the convolution $\varphi * f$ between a test function $\varphi$ and a distribution $f$ is a 
      distribution such that
    \begin{equation*}
      \forall \psi \in \mathcal{S}, \qquad f*\varphi\{\psi\} = f\{\check{\varphi}*\psi\}
    \end{equation*}
    This defines a continuous and linear functional (exercise) on $\mathcal{S}$, that is, a distribution.
    \item The Fourier transform of the distribution $f * \varphi$ is the distribution
      \begin{align*}
	\forall \psi \in \mathcal{S}, \qquad \widehat{f*\varphi}\{\psi\} &= f*\varphi\{\hat{\psi}\}, \\
	&= f\{\check{\varphi}*\hat{\psi}\}, \\
	&= f\{\widehat{\hat{\varphi}\psi}\}, \\
	&= \hat{f}\{\hat{\varphi}\psi\}.
      \end{align*}
  \end{enumerate}
\end{remark}

\begin{example}
  \begin{enumerate}
    \item
      For $L := D$, derivation operator, the distribution $h$ is $\delta'$, \textit{i.e}, the distribution that 
      associates to each test function $\varphi$ the complex number $-\varphi'(0)$. Indeed,
      \begin{align*}
	\forall (\varphi,\psi) \in \mathcal{S}^2, \qquad \delta'*\varphi \{\psi\} &= \delta'\{\check{\varphi}*\psi\}, \\
	&= - \left(\check{\varphi}*\psi\right)'(0), \\
	&= \int_{-\infty}^{\infty} \varphi'(t)\psi(t)dt.
      \end{align*}
      The Fourier transform of $D\{\varphi\}$ is by definition the distribution
      \begin{align*}
	\forall \psi \in \mathcal{S}, \qquad \widehat{D\{\varphi\}}\{\psi\} &= \int \varphi'\hat{\psi}, \\
	&= \int \hat{\varphi'} \psi.
      \end{align*}
    There, we use the notation $\hat{D}(u) = ju$ for the distributional equality 
    \begin{equation*}
      \forall \varphi \in \mathcal{S}, \quad \widehat{\hat{D}\{\varphi\}}(u) = ju\varphi(u).
    \end{equation*}
  \item For $L := D-\alpha I$, the distribution $h$ is $\delta' - \alpha id$. The Fourier transform of $(D-\alpha 
    I)\{\varphi\}$ is the distribution
    \begin{equation*}
      \forall \psi \in \mathcal{S}, \qquad \reallywidehat{(D-\alpha I)\{\varphi\}}\{\psi\} = \int (\hat{\varphi'}-\alpha 
      \varphi)\psi.
    \end{equation*}
    Similarly, $\widehat{D-\alpha I}(u) = ju-\alpha$ denotes the distributional equality
    \begin{equation*}
    \forall \varphi \in \mathcal{S}, \quad \reallywidehat{(D-\alpha I)\{\varphi\}}(u) = (ju-\alpha)\varphi(u).
    \end{equation*}
  \end{enumerate}
\end{example}

The operator $L$ is also characterized by the roots $\bm{\alpha}$ of its characteristic polynomial
\begin{equation*}
  s^n + a_{n-1}s^{n-1} + \cdots + a_0 = \prod_{k=1}^{n} (s-\alpha_k)
\end{equation*}
Accordingly, we shall denote $L_{\bm{\alpha}}$ the operator induced by the vector $\bm{\alpha}$. The “Fourier transform” 
of the operator $L_{\bm{\alpha}}$ is
\begin{equation*}
  \hat{L_{\alpha}}(u) = \prod_{k=1}^n (ju-\alpha_k)
\end{equation*}
Let $\{\alpha_{(m)}\}_{m=1,\ldots, n_d}$ be the $n_d$ distinct components of $\bm{\alpha}$ with multiplicity $n_{(m)}$, 
hence ${\sum_{m=1}^{n_d} n_{(m)} = n}$. Then, the null space of $L_{\bm{\alpha}}$ is the set~\cite{unser_cardinal_2005}
\begin{equation*}
  \mathcal{N}_{\bm{\alpha}} = \Span \{t^{k-1}e^{\alpha_{(m)}t}\}_{m \in \llbracket1, n_d\rrbracket, k \in \llbracket1, 
  n_{(m)}\rrbracket}
\end{equation*}

The Fourier multiplier of $L$, is the distribution $\hat{L}$, such that 
\begin{equation*}
  \forall \varphi \in \mathcal{S}, \qquad L\{\varphi\} = \mathscr{F}^{-1}\{\hat{L}\hat{\varphi}\}
\end{equation*}

\begin{deftn}\label{def:Green}
  Let $L:\mathcal{S} \to \mathcal{S}'$ be an operator with smooth Fourier multiplier that is nowhere zero on 
  $\mathbb{R}$ and decaying slowly (as a polynomial at most) at infinity. The following equivalence holds
  \begin{align*}
    \rho &= \mathscr{F}^{-1}\left\{\frac{1}{\hat{L}}\right\} \\
    &\iff \\
    L \{\rho\} &= \delta \quad \text{and} \ \rho \ \text{is causal.}
  \end{align*}
  A function, $\rho_L$, satisfying these is called the \emph{Green} function of the operator $L$. It is the convolution 
  kernel of the LSI inverse operator $L^{-1}: \mathcal{S}\to\mathcal{S}$ given by
  \begin{equation*}
    \forall \varphi \in \mathcal{S}, \qquad 
    L^{-1}\{\varphi\}=\mathscr{F}^{-1}\left\{\frac{\hat{\varphi}}{\hat{L}}\right\}.
  \end{equation*}
 \end{deftn}

\begin{example}
  The Green function of the first-order operator $L_{\alpha} := D-\alpha I$ is the one-sided (or causal) exponential
  \begin{equation*}
    \rho_{\alpha}(t) = 1_+(t)e^{\alpha t}
  \end{equation*}
  It easily verified that $L_{\alpha}\{\rho_{\alpha}\} = \delta$.
\end{example}

\subsection{Exponential splines and B-splines}

An exponential spline is then defined as
\begin{deftn}[{\cite[Definition 1]{unser_cardinal_2005}}]\label{def:exponential-spline}
  An exponential spline with parameter $\bm{\alpha}$ and knots $(t_{k})_{k}$ is a function $S$ such that
  \begin{equation*}
    L_{\bm{\alpha}}\{S\} = \sum_{k} a_k \delta\{\cdot-t_k\}
  \end{equation*}
  with $a_k$ bounded and $\delta\{\cdot-t_k\}$ the shifted Dirac distribution.
\end{deftn}

From the definition of the Green function, the exponential spline $S$ can be explicitly represented as 
\begin{equation*}
  S(t) = \sum_{k} a_k \rho_{\bm{\alpha}}(t-t_k) + p_{\bm{\alpha}}(t)
\end{equation*}
with $p_{\bm{\alpha}} \in \mathcal{N}_{\bm{\alpha}}$. 

For the reason explained in the preamble to the second part of this zoo, the knots are chosen to be the uniform integer 
grid $\mathbb{Z}$. The associated exponential splines are said to be \emph{cardinal}. In the polynomial splines, 
localized function called $B$-splines were defined, with the property that they span the set of all splines. Similarly, 
one can define exponential B-splines as follows

\begin{deftn}[{\cite[(10), (11)]{unser_cardinal_2005}}]\label{def:exponential-B-spline}
  A cardinal exponential B-spline of first-order with parameter $\alpha$ is the function given by
  \begin{equation*}
    \beta_{\alpha}(t) = \rho_{\alpha}(t) - e^{\alpha} \rho_{\alpha}(t-1)
  \end{equation*}
  High-order B-splines are obtained as successive convolution of first-order B-splines
  \begin{equation*}
    \beta_{\bm{\alpha}} = \beta_{\alpha_1}*\cdots*\beta_{\alpha_n}
  \end{equation*}
\end{deftn}
