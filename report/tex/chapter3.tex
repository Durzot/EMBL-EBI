\chapter{General properties of Hermite B-splines}\label{chapter3}

In Chapter~\ref{chapter1} we presented the first general works on interpolation dating from Newton before laying the 
foundations for the modern theory of splines. In Chapter~\ref{chapter2} we gave general results of existence and unicity 
for solutions to the Hermite interpolation problem, most of them due to I.J Schoenberg. In Theorems~\ref{thm:LH-gamma} 
and~\ref{thm:LH-p}, we saw that the solution is explicitly given by its Lagrange-Hermite expansion which is spanned by 
the fundamental splines $L_0, \ldots, L_{r-1}$ (\ref{def:fundamental-r}). This expansion is on the one hand very 
advantageous because its coefficients are exactly the samples of the function and its derivatives we wish to 
interpolate. On the other hand, it may be numerically impractical for the reason that the fundamental splines are not 
compactly supported in general. This led us to the introduction of Hermite B-splines, the Hermite equivalent to the 
compactly supported B-splines $M_n, Q_n$ we introduced for the cardinal interpolation problem. In the specific case 
where the order of the splines is exactly twice the multiplicity of the interpolation, these Hermite B-splines are just 
rescaled versions of the fundamental splines. We saw that Hermite B-splines are a basis for some specific spline 
subspaces but we yet have to conduct an extensive study of their properties. In this third and last chapter, we wish to 
present the numerous good properties of these Hermite B-splines in order to convince the reader that they are a great 
and effective tool for a lot of practical problems found in image analysis and computer-aided geometrical design. Most 
of the work in this chapter is the result of my own research, which is split between two different aspects. On the one 
hand, it took a lot of time to connect results found by different communities which, while having existed for decades, 
seem to be largely ignored by practitioners interested in splines. This lack of a global picture of Hermite B-splines is 
sometimes confusing and may lead some to publish theorems that are subcases of general results already known in more 
abstract communities. I intend to partially bridge the gap between these communities by presenting general results 
connected to the Hermite splines in a manner that is more familiar to practitioners. On the other hand, largely inspired 
by the very recent paper (\cite{FAUU19}) by J. Fageot et al that formally investigates properties of cubic Hermite 
splines, I dediced to push the investigation of the general properties of Hermite splines. Convinced that all the 
results found at order $2$ and $4$ extend to the general case, I tried to extends the proofs to any order. 

\section{Fourier transforms}

\underline{Notations}
\begin{enumerate}
  \item $r, m \in \mathbb{N}^*$, $m \geq r$
  \item $L_0:=L_{2m,r,0}, \ldots, L_{r-1}=L_{2m,r,r-1}$ are the \emph{fundamental} splines 
    Definition~\ref{def:fundamental-r}.
  \item $N_0:=N_{2m,r,0}, \ldots, N_{r-1}:=N_{2m,r,r-1}$ are the Hermite B-splines Definition~\ref{def:hbsplines}.
  \item $\hat{\phi}$ is the Fourier transform of $\phi \in L^1, L^2$ or Schwartz space $\mathscr{S}$, $u$ denotes the 
    frequency variable, $t$ the temporal variable
\end{enumerate}

\subsection{Lee's general formulas}

The Fourier transform has become such a fundamental tool in signal processing that is hard to find a paper in this 
community where it is not used. Characterizing a function in the frequency domain is very helpful for understanding its 
properties in the time domain. In this perspective, the Fourier transform of Hermite B-splines will prove very useful 
for characterizing their mathematical properties as well as for understanding their connections to other B-splines.  
However, computing the Fourier transform for general $r$ of $L_0, \ldots, L_{r-1}$, or equivalently of $N_0, \ldots, 
N_{r-1}$, is a difficult task for which no explicit formula has yet been reported. A formula in terms of Hankel 
determinant of $r$-dimensional matrices exists though, which was established by S. Lee in his papers 
(\cite{Lee76a},~\cite{Lee76b}) and will be useful for proving that some determinant does not vanish. In the case of 
multiplicity $r=1$ the Fourier transforms of $N_0$ and $L_0$ can be computed as detailed in the following example. 

\begin{example} The Fourier transform of the splines $N_0 := N_{2m, 1, 0}$ and $L_0 := L_{2m,1,0}$ for cardinal 
  interpolation were known to Schoenberg in (\cite[Lecture 2, (1.8)]{Sch73},~\cite[(1.3)]{SS73}) and (\cite[Lecture 10, 
  (1.11)]{Sch73}). More specifically, from Example~\ref{ex:N0}, we know that \begin{equation*}
    N_0 = (2m-1)!M_{2m}
  \end{equation*}
  while the Fourier transform of $M_{2m}$ is given by item 3 of Proposition~\ref{prop:sch-B-splines}. Therefore, 
  \begin{equation}
    \forall u \in \mathbb{R}, \quad \hat{N}_0(u) = (2m-1)! {\left(\frac{2\sin(\frac{u}{2})}{u}\right)}^{2m}
  \end{equation}

  As for $L_0 = L_{2m,1,0}$, note once again that the Lagrange-Hermite expansion is exact for $M_{2m}$ that is 
  \begin{equation*}
    \forall t \in \mathbb{R}, \quad M_{2m}(t) = \sum_{k=-\infty}^{\infty} M_{2m}(k) L_0(t-k)
  \end{equation*}
  The Fourier transform of this mixed convolution is \begin{equation}\label{eq:mixed}
    \hat{M}_{2m}(u) = \hat{M}_{d,2m}(u) \hat{L}_0(u)
  \end{equation}
  where \begin{align*} \hat{M}_{d, 2m}(u) &= \sum_{k=-\infty}^{\infty} M_{2m}(k) e^{-juk} \\
    &= \sum_{k=-\infty}^{\infty} {\left(\frac{2 \sin(\frac{u+2k\pi}{2})}{u+2k\pi}\right)}^{2m}
  \end{align*}
  is the discrete Fourier transform written in two different ways using Poisson summation formula. This last series is 
  the sum of positive terms and therefore can only vanish if the terms of the sum vanish altogether. However, for this 
  to happen $u$ would need to be in $2\pi\mathbb{Z}$ say $u = 2l\pi$ but then the term $\frac{2 
  \sin(\frac{u-2l\pi}{2})}{u-2l\pi}$ is 1 by definition. Therefore we can isolate $\hat{L}_0$ in (\ref{eq:mixed}) and 
  write
  \begin{equation}
    \forall u \in \mathbb{R}, \quad \hat{L}_0(u) = \frac{u^{-2m}}{\displaystyle\sum_{k=-\infty}^{\infty} 
    {(u+2k\pi)}^{-2m}}
  \end{equation}
  after some obvious simplifications.
\end{example}

Inspired by these formulas, Lee computed the Fourier transforms of ${(N_s)}_{s=0}^{r-1}$ and 
${(L_{2m,r,s})}_{s=0}^{r-1}$ for general $r \in \mathbb{N}^*$. To express these we need to introduce

\begin{deftn}[Hankel determinant]\label{def:Hankel}
  The Hankel determinant of order $r$ for the sequence ${(a_n)}_n$, $H_r(a_n)$, is the determinant
  \begin{equation}
      H_r(a_n) = |\matr{H_r(a_n)}|
   \end{equation}
  where
  \begin{equation*}
    \matr{H_r(a_n)} = \begin{bmatrix} a_n & a_{n-1} & \hdots & a_{n-r+1} \\
      a_{n-1} & a_{n-2} & \hdots & a_{n-r} \\
      \vdots & \vdots & \ddots & \vdots \\
      a_{n-r+1} & a_{n-r} & \hdots & a_{n-2r+2}  
    \end{bmatrix}
  \end{equation*}
\end{deftn}

Lee defines the following additional quantities to express the Fourier transforms more compactly.
\begin{enumerate}
  \item $\forall u \in \mathbb{R}\setminus2\pi\mathbb{Z}, \quad \displaystyle \alpha_n(u) = \sum_{k=-\infty}^{\infty} 
    {(u+2k\pi)}^{-n}$
  \item Let $C^{\matr{H_r}(\bm{\alpha_n})}_0, \ldots, C_{r-1}^{\matr{H_r}(\bm{\alpha_n})}$ the columns of 
    $\matr{H_r}(\bm{\alpha_n})$.  Let $C(u)$ the column vector \begin{equation}
      C(u) = \begin{bmatrix} u^{-n} & u^{-n+1} & \hdots & u^{-n+r-1} \end{bmatrix}^T
    \end{equation}
    For $s=0, \ldots, r-1$,  define the matrix
    \begin{equation*}
      \matr{H_{r,s}}(\bm{\alpha}_n) = \begin{bmatrix} C_0^{\matr{H_r}({\bm{\alpha}}_n)} & \ldots & 
	C_{s-1}^{\matr{H_r}({\bm{\alpha}}_n)} & C(u) & C_{s+1}^{\matr{H_r}({\bm{\alpha}}_n)} & \hdots & 
	C_{r-1}^{\matr{H_r}({\bm{\alpha}}_n)}
      \end{bmatrix}
    \end{equation*}
    and the \emph{modified} Hankel determinant
    \begin{equation}\label{eq:modified-Hankel}
    H_{r,s}(\alpha_n(u)) = |\matr{H_{r,s}}({\bm{\alpha}}_n(u))|
  \end{equation}
  \item $K(m,r) = {(-1)}^{m(r+1)} \frac{(2m-1)!(2m-2)!\ldots(2m-r)!}{1!2!\ldots(r-1)!}$
\end{enumerate}

Given these notations, Lee stated and proved the following theorem
\begin{thm}[{\cite[Theorems 1, 2]{Lee76b}}]\label{thm:Lee}
  The Fourier transform of the fundamental splines is
  \begin{equation}\label{eq:Lee1}
    \hat{L}_{2m,r,s}(u) = {(-j)}^{s} \frac{H_{r,s}(\alpha_{2m}(u))}{H_r(\alpha_{2m}(u))}
  \end{equation}
  and that of the Hermite B-splines is
  \begin{equation}\label{eq:Lee2}
    \hat{N}_s(u) = {(-j)}^s K(m,r) {\left(2 \sin \frac{u}{2} \right)}^{2m} H_{r,s}(\alpha_{2m}(u))
  \end{equation}
\end{thm}

The proof is given by Lee in his paper (\cite{Lee76b}) but is short of details for readers not familiar with the 
subject. It is therefore reproduced with additional details in Section~\ref{proof:Lee}. Let's finish this subsection 
with a Lemma that will be useful for the next section.

\begin{lem}\label{lem:Ns}
  The Hermite B-splines $N_0, \ldots, N_{r-1}$ have continuous Fourier transforms $\hat{N}_0, \ldots, \hat{N}_{r-1}$ 
  with the property that
  \begin{equation*}
    \forall s \in \llbracket0,r-1\rrbracket, \quad \hat{N}_s(u) = \mathcal{O}_{|u| \to \infty}\left( 
    \frac{1}{u^{2m-r+1}}\right)
  \end{equation*}
\end{lem}

\begin{proof}
  The Fourier transforms of the Hermite B-splines are continuous as $N_0, \ldots, N_{r-1}$ are continuous and have 
  compact support. Let $\matr{H_r(u)} := \matr{H_r}(\alpha_{2m}(u)), \matr{H_{r,s}(u)} := \matr{H_{r,s}}(\alpha_{2m}(u)) 
  $ the Hankel matrices and $H_r(u), H_{r,s}(u)$ their determinant.  For $(i,j) \in {\llbracket0,r-1\rrbracket}^2$, let 
  $H_r^{(i,j)}(u)$ the determinant of the matrix $\matr{H_r(u)}$ with $(i+1)^{th}$ row  and $(j+1)^{th}$ column deleted.  
  Developing the determinant $H_{r,s}(u)$ around the $(s+1)^{th}$ column leads to
  \begin{equation*}
    H_{r,s}(u)  = \sum_{i=0}^{r-1} (-1)^{i+s+1} H_{r}^{(i, s+1)}(u) \frac{1}{u^{2m-i}}
  \end{equation*}
  Therefore, 
  \begin{equation*}
    \forall u \in \mathbb{R}, \forall s \in \llbracket0,r-1\rrbracket, \quad \hat{N}_s(u) = {(-j)}^s K(m,r) {\left(2 
    \sin \frac{u}{2} \right)}^{2m} \sum_{i=0}^{r-1} (-1)^{i+s+1} H_{r}^{(i, s+1)}(u) \frac{1}{u^{2m-i}}
  \end{equation*}
  The functions $H_{r}^{(i, s+1)}(u)$ are $2\pi$-periodic as they are polynomial in quantities that are $2\pi$-periodic.  
  Therefore there exists a global bound $M$ on these functions and the following holds
  \begin{equation*}
    \forall |u| \geq 1, \forall s \in \llbracket0,r-1\rrbracket, \quad |u^{2m-r+1}\hat{N}_s(u)| \leq 4^m K(m,r) M 
  \end{equation*}
\end{proof}

\subsection{The case of compact fundamental splines}

It is clear from the Definition~\ref{def:fundamental-r} that the fundamental splines $L_0, \ldots, L_{r-1}$ are 
compactly supported when the order of the splines, $2m$, is exactly twice the multiplicity $r$ that is to say $m=r$.  
From this point it is assumed that $m=r$ for the rest of this subsection.  In such a case, the scheme resulting from the 
Hermite-Lagrange expansion (\ref{eq:LH}) becomes local in the sense that a change in the value of one data point has 
only local consequences on the resulting function. On top of that, the scheme is now computationally efficient as the 
infinite sum always reduces to a finite sum of fixed size at any evaluation point. In summary, (\ref{eq:LH}) with $m=r$ 
provides us with an interpolating function that is such that: 

\begin{enumerate}
  \item its parameters are exactly the input
  \item it can evaluated very efficiently at any point.  
  \item it depends only locally on the input data
\end{enumerate}

In view of these features, the Hermite-Lagrange expansion (\ref{eq:LH}) is a perfect candidate for a practical solution 
to a Hermite-type interpolation problem. The scheme has in fact additional good properties that we shall describe in the 
coming sections but before that we need a formula for the Fourier transform of more practical usability than Lee's 
formulas (\ref{eq:Lee1}) and (\ref{eq:Lee2}). Note first that $m=r$ simplifies the definition of fundamental splines in 
the sense that for ${s=0, \ldots, r=1}$, $L_s$ is now the function that satisfies

\begin{equation}
  L_s(t) = \begin{dcases}
    \frac{t^{s}}{s!} + a_{1,s}t^{r} + a_{2,s}t^{r+1} + \cdots + a_{r,s} t^{2r-1} & \text{if} \ 0 \leq t \leq 1 \\
    0 & \text{if} \ t \geq 1 \\
    {(-1)}^s L_s(-t)  & \text{if} \ t < 0
  \end{dcases}
\end{equation}
with
\begin{equation}
  \begin{bmatrix}
    a_{1,s} \\ a_{2,s} \\ \vdots \\ a_{r,s} 
  \end{bmatrix}
  =
  {\begin{bmatrix}
    \frac{r!}{r!} & \frac{(r+1)!}{(r+1)!} & \cdots & \frac{(2r-1)!}{(2r-1)!}  \\
    \frac{r!}{(r-1)!} & \frac{(r+1)!}{r!} & \cdots & \frac{(2r-1)!}{(2r-2)!}  \\
    \vdots & \vdots & \ddots & \vdots \\
    \frac{r!}{1!} & \frac{(r+1)!}{2!} & \cdots & \frac{(2r-1)!}{r!}  \\
  \end{bmatrix}}^{-1}
  \begin{bmatrix} -\frac{1}{s!} \\ \vdots \\ -\frac{1}{0!} \\ 0 \\ \vdots \\ 0 \end{bmatrix}
\end{equation}

The following formulas are based on the empirical observation of repeated patterns in the first Fourier transform of the 
compact fundamental splines $L_0, \ldots, L_{r-1}$ and have yet to be proven. 

\begin{con}
  In the particular case $m=r$, the Fourier transforms of the fundamental splines $L_0, \ldots, L_{r-1}$ are
  \begin{equation}
    \begin{split}
      \hat{L}_{0}(u) &= \frac{(2r)!}{r!u^{2r}} \Big[ (\cos(u)-1) \sum_{k=1}^{\lfloor \frac{r+1}{2} \rfloor} 
	\frac{(2r-2k)!}{(2k-2)!(r+1-2k)!} {(-1)}^k u^{2k-2} \\
      &\quad + \sin(u) \sum_{k=1}^{\lfloor \frac{r}{2} \rfloor} \frac{(2r-1-2k)!}{(2k-1)!(r-2k)!} {(-1)}^k u^{2k-1} 
    \Big]
    \end{split}
  \end{equation}
  \begin{equation}
    \begin{split}
      \hat{L}_1(u) &= \frac{2r(2r-2)!j}{r!u^{2r}} \Big[\cos(u) \sum_{k=1}^{\lfloor \frac{r}{2} 
	\rfloor}\frac{(2r^2-(2k+2)r+1)(2r-2-2k)!}{(2k-1)!(r-2k)!} {(-1)}^k u^{2k-1} \\
      &\quad + \sum_{k=1}^{\lfloor \frac{r}{2} \rfloor}\frac{(2r^2-(2k+2)r+2k)(2r-2-2k)!}{(2k-1)!(r-2k)!} {(-1)}^k 
    u^{2k-1} \\
      &\quad + \sin(u) \sum_{k=1}^{\lfloor \frac{r+1}{2} \rfloor} \frac{(2r^2-(2k+1)r+1)(2r-2k-1)!}{(2k-2)!(r+1-2k)!} 
    {(-1)}^{k+1} u^{2k-2} \Big]
    \end{split}
  \end{equation}
\end{con}

\section{Hermite B-splines \& Riesz-Schauder basis}

We adopt the following notations in this section.
\begin{itemize}
  \item $r, p \in \mathbb{N}^*$
  \item $\mathbb{K}$ designates the field $\mathbb{R}$ of real numbers or the field $\mathbb{C}$ of complex numbers.
  \item $I$ is a set of indices
  \item $\ell^p(I)$ denotes all collections $\bm{c} \in \mathbb{K}^{I}$ such that \begin{equation*}
      \sum_{k \in I} c_k^p < \infty
    \end{equation*}
  \item ${\left(\ell^p(I)\right)}^r$ denotes all $r$-uplet of sequences $\bm{c}=\left(\bm{c_{0}},\ldots 
    \bm{c_{r-1}}\right)$ such that $\bm{c_{s}}$ is in $\ell^{p}(I)$.
  \item $\mu$ is the Lebesgue measure on $\mathbb{R}$
  \item $\mathscr{L}^p := \mathscr{L}^p(\mathbb{R}) := \mathscr{L}^p(\mathbb{R}, \mathbb{K})$ denotes all 
    $\mu$-measurable functions $f: \mathbb{R} \to \mathbb{K}$ such that \begin{equation*}
      \int_{-\infty}^{\infty} f d\mu < \infty
    \end{equation*}
  \item $\mathcal{L}^p := \mathcal{L}^p(\mathbb{R}) := \mathcal{L}^p(\mathbb{R}, \mathbb{K})$ denote the quotient set 
    $\mathscr{L}^p/R$ with $R$ the equivalence relation $f R g$ if $f = g$ $\mu$-a.e.
  \item If $f: \mathbb{R} \to \mathbb{K}$ is a function, $\check{f}: \mathbb{R} \to \mathbb{K}$ is the opposite function 
    i.e $\check{f}(t) = f(-t)$.
\end{itemize} 

The present Section will show that the Hermite B-splines $N_0, \ldots, N_{r-1}$ give rise to a Riesz-Schauder basis, a 
property that is absolutely essential to guarantee safe and sound numerical implementations. It is also a property of a 
high theoretical interest as to quote Ingrid Daubechies, ``a Riesz-Schauder basis is the next-best thing after an 
orthogonal basis''. Before detailing further the benefits of using a Riesz-Schauder basis, let's define it.  
\begin{deftn}[Riesz-Schauder basis]
  Let $\mathcal{H}$ be a Hilbert space over the field $\mathbb{K}$.  A collection of functions ${\{\varphi_k\}}_{k \in 
  I}$ of $\mathcal{H}$ is said to be a Riesz-Schauder basis for the space it spans if there exist positive constants $0 
  < A \leq B$ such that
  \begin{equation}\label{eq:def-Riesz}
    \forall c \in \ell^2(I), \qquad A {\|c\|}_{\ell^2} \leq {\left\| \sum_{k \in I} c_k \varphi_k 
    \right\|}_{\mathcal{H}} \leq B {\|c\|}_{\ell^2}
  \end{equation}
\end{deftn} 

\begin{remark} 
  \begin{enumerate} \item In practical cases, the set of indices $I$ will be the set of all integers $\mathbb{Z}$.  
    \item The expression ``Riesz-Schauder basis'' is usually abbreviated ``Riesz basis''.  
    \item If ${\{\varphi_k\}}_{k \in I}$ is a Riesz basis for $\Span{\{\varphi_k\}}_{k \in I}$, then it is a basis in 
      the usual algebraic sense.  The Riesz basis condition indeed implies linear independence of the function as a 
      linear combination of the $\varphi_k$ can only vanish if the coefficients of that combination also vanish.
  \end{enumerate} 
\end{remark} 


It is numerically essential that the functions $\{L_s(\cdot-k)\}_{s\in \llbracket0,r-1\rrbracket, k \in \mathbb{Z}}$ or 
equivalently, when $m=r$, that the functions $\{N_s(\cdot-k)\}_{s\in \llbracket0,r-1\rrbracket, k \in \mathbb{Z}}$ used 
for representing solutions to the Hermite interpolation problem (\ref{eq:LH}) constitute a Riesz basis. Indeed, if that 
is the case, the representation of the function is unique and stable. Unicity is straightforward while stability means 
that a change in the sequence of coefficients, say  $\bm{\delta c}$, results in a comparative change in the function in 
the sense that \begin{equation}
  A {\|\bm{\delta c}\|}_{\ell^2} \leq {\left\| \sum_{k=-\infty}^{\infty} \delta c_k L_s(\cdot-k) 
  \right\|}_{\mathcal{L}^2} \leq B {\|\bm{\delta c}\|}_{\ell^2}
\end{equation} 

The set $\mathcal{L}^{2}(\mathbb{R})$ endowed with the inner product 
\begin{equation}\label{eq:inner-prod}
  \langle f,g\rangle = \int_{-\infty}^{\infty} f\bar{g} d\mu 
\end{equation}
is a Hilbert space and it is this space that we shall use to write the Riesz basis property of Hermite B-splines.  One 
should indeed observe that the functions $N_0, \ldots, N_{r-1}$ are in $\mathcal{L}^p$ as the functions $L_0, \ldots, 
L_{r-1}$ themselves are in $\mathcal{L}^p$ following Theorem~\ref{thm:LH-p}. As $\mathcal{L}^p$ is a Hilbert space for 
$p=2$, it is that space that we should consider.  Let's provide a series of general results on the Riesz basis property 
that we shall use to prove that the Hermite B-splines form a Riesz-Schauder basis.  

\subsection{Functions in \texorpdfstring{$\mathcal{L}^2$}{g}}\label{ssec:l2}

Let $\varphi_0, \ldots, \varphi_{r-1}$ be functions in $\mathcal{L}^2$ such that the map 

\begin{align}\label{def:mapK}
  K: {(\ell^2)}^r &\longrightarrow  \mathcal{L}^2 \nonumber \\
  \bm{c} &\longmapsto \sum_{k=-\infty}^{\infty} \sum_{s=0}^{r-1} c_{s,k} \varphi_s(\cdot-k)
\end{align} is well-defined.  
  
Let's denote $\bm{\varphi}$ the collection of function ${\{\varphi_s(\cdot-k)| s \in \llbracket0, r-1\rrbracket, k \in 
\mathbb{Z}\}}$. As we shall see in the next theorem, the Fourier transform is very helpful for characterizing the 
properties of the functions in $\bm{\varphi}$ and in particular the Riesz basis property. For that, we will need the 
Gram matrix as defined in
\begin{deftn}[Gram matrix]
  The Gram matrix for $(\varphi_0, \ldots, \varphi_{r-1})$ at $u \in \mathbb{R}$ is the $r \times r$ matrix
  \begin{equation}
    \matr{\hat{G}}(u) = (\matr{\hat{G}}_{i,j}(u))_{(i,j) \in {\llbracket0, r-1\rrbracket}^2}, \quad 
    \matr{\hat{G}}_{i,j}(u) = \sum_{k=-\infty}^{\infty} \hat{\varphi}_i(u+2k\pi)\overline{\hat{\varphi}_j(u+2k\pi)}
  \end{equation}
\end{deftn}

\begin{remark}
  \begin{enumerate}
    \item Why it is well-defined.
    \item At any $u \in \mathbb{R}$, the matrix $\hat{G}(u)$ is hermitian and therefore has $r$ eigenvalues that we will 
      note and order $\lambda_0(u) \leq \ldots \leq \lambda_{r-1}(u)$. Let's denote $\matr{D}(u) = \diag (\lambda_0(u), 
      \ldots, \lambda_{r-1}(u))$ and $\matr{Q}(u)$ the unitary matrix such that
      \begin{equation}\label{eq:spectral}
	\matr{\hat{G}}(u) = \matr{Q}(u) \matr{D}(u) \matr{Q}(u)^*
      \end{equation}
  \end{enumerate}
\end{remark} The following theorem is proved in (\cite[Theorem 2]{AldUns94}) for the case of a single generator 
$\varphi$ ($r=1$), and in (\cite[Theorem 3.1]{GST93}) for the general case. It is interesting to note that  
(\cite[Theorem 2]{AldUns94}) additionally proves which sequence $\bm{c}_{opt}(g)$ achieves the minimum squared-error 
approximation $K(\bm{c}_{opt}(g))$ of a function $g \in \mathcal{L}^2$.

\begin{thm}[{\cite[Theorem 3.1]{GST93}}]\label{thm:Riesz-Gram} The collection of functions $\bm{\varphi}$ is a Riesz 
  basis with Riesz bounds $0 < A \leq B$ if and only if the eigenvalues of $\matr{\hat{G}}$ are essentially bounded by 
  $A^2$ and $B^2$ i.e
  \begin{equation} A^2 \leq \essinf_{u \in [-\pi, \pi]} \lambda_{0}(u) \leq \esssup_{u \in [-\pi, \pi]} \lambda_{r-1}(u) 
    \leq B^2
  \end{equation}
\end{thm}

The proof is reproduced here as it uses key ideas that help understand the theorem and may prove fruitful for other 
proofs.
\begin{proof}
  \begin{enumerate}
    \item
      Let's first establish a relation between the $\mathcal{L}^2$ norm of $K(\bm{c})$ and the Gram matrix. For that, 
      observe that the $\mathcal{L}^2$ norm is induced by the inner product (\ref{eq:inner-prod}) of the Hilbert space 
      $\mathcal{L}^2$, hence the following
      \begin{align*}
	{\left\| \sum_{k=-\infty}^{\infty} \sum_{s=0}^{r-1} c_{s,k} \varphi_s(\cdot-k)\right\|}_{\mathcal{L}^2}^2 &= 
	\left\langle  \sum_{k=-\infty}^{\infty} \sum_{s=0}^{r-1} c_{s,k} \varphi_s(\cdot-k), \sum_{l=-\infty}^{\infty} 
	\sum_{t=0}^{r-1} c_{t,l} \varphi_t(\cdot-l) \right\rangle \\
	& = \sum_{s,t=0}^{r-1} \sum_{k,l=-\infty}^{\infty}  c_{s,k} \left\langle \varphi_s, \varphi_t(\cdot-(l-k)) 
	\right\rangle \overline{c_{t,l}}
      \end{align*}
      For $s \in \llbracket0, r-1\rrbracket$, let $\widehat{\bm{c_s}}$ be the discrete-time Fourier transform of $\bm{c_s}$ 
      i.e
      \begin{equation*}
	\forall u \in \mathbb{R}, \qquad \widehat{\bm{c_s}}(u) = \sum_{k=-\infty}^{\infty} c_{s,k} e^{-juk}
      \end{equation*}
      For $(s,t) \in {\llbracket0, r-1\rrbracket}^2$, let $\bm{g_{s,t}}$ be the sequence ${\left(\langle \varphi_s, 
      \varphi_t(\cdot-(k))\rangle\right)}_{k \in \mathbb{Z}}$ and let $\widehat{\bm{g_{s,t}}}$ be its discrete-time Fourier 
      transform. The calculations above continue in the following manner,
      \begin{align*}
	{\left\| \sum_{k=-\infty}^{\infty} \sum_{s=0}^{r-1} c_{s,k} \varphi_s(\cdot-k)\right\|}_{\mathcal{L}^2}^2
	& = \sum_{s,t=0}^{r-1} \sum_{l=-\infty}^{\infty}  \left(\bm{c_{s}}*\bm{g_{s,t}}\right)(l)\overline{\bm{c_{t}}(l)} \\
	& = \sum_{s,t=0}^{r-1} \frac{1}{2\pi} \int_{-\pi}^{\pi}  \widehat{\bm{c_{s}}}(u)\widehat{\bm{g_{s,t}}}(u) 
	\overline{\widehat{\bm{c_{t}}}(u)} du
      \end{align*}
      where in the last line we made use of Parseval's theorem for discrete-time and the property of Fourier transform on 
      convolution product. It is now time to observe that, using Poisson's summation formula, for any $(s,t) \in 
      {\llbracket0, r-1\rrbracket}^2$ and any $u \in \mathbb{R}$,
      \begin{align*}
	\widehat{\bm{g_{s,t}}}(u) &= \sum_{k=-\infty}^{\infty} \left(\langle \varphi_s, \varphi_t(\cdot-(k))\rangle\right) 
	e^{-juk} \\
	& = \sum_{k=-\infty}^{\infty} (\varphi_s * \check{\varphi_t})(k) e^{-juk} \\
	& = \sum_{k=-\infty}^{\infty} \hat{\varphi_s}(u+2k\pi)\overline{\hat{\varphi_t}(u+2k\pi)} \\
	& = \matr{\hat{G}_{s,t}}(u)
      \end{align*}
      Letting $\hat{\bm{c}}$ the $r$-dimensional row vector $\begin{bmatrix} \widehat{\bm{c_0}} & \hdots & 
      \widehat{\bm{c_{r-1}}} \end{bmatrix}$ (slight abuse of notation as $\bm{c}$ is also a $r$-uplet of sequences), we 
      eventually have
      \begin{equation}\label{eq:rel-fourier}
	{\left\| \sum_{k=-\infty}^{\infty} \sum_{s=0}^{r-1} c_{s,k} \varphi_s(\cdot-k)\right\|}_{\mathcal{L}^2}^2
	= \frac{1}{2\pi} \int_{-\pi}^{\pi}  \widehat{\bm{c}}(u)\hat{\matr{G}}(u)\widehat{\bm{c}}^*(u) du
      \end{equation}
      which is the relation we sought to establish.  

    \item 
      Let $\bm{c} \in {(\ell^{2}(\mathbb{Z}))}^r$ and let $\bm{d} \in {(\ell^{2}(\mathbb{Z}))}^r$ defined in the Fourier 
      domain by  
      \begin{equation}\label{eq:dc}
	\forall u \in \mathbb{R}, \qquad \widehat{\bm{d}}(u) = \widehat{\bm{c}}(u)\matr{Q}(u)
      \end{equation}
      with $\matr{Q}(u)$ the matrix from the spectral decomposition (\ref{eq:spectral}). As $\matr{Q(u)}$ is unitary, we 
      have $\displaystyle \sum_{s=0}^{r-1} |\widehat{c_s}|^2 = \sum_{s=0}^{r-1} |\widehat{d_s}|^2$. Combining it with 
      Parseval theorem leads to
      \begin{equation*}
      {\|\bm{c}\|}_{{(\ell^2)}^r}^2 = {\|\bm{d}\|}_{{(\ell^2)}^r}^2 = \sum_{s=0}^{r-1} \frac{1}{2\pi} \int_{-\pi}^{\pi} 
      |\widehat{d_s}(u)|^2 du
      \end{equation*}
      As $\matr{Q}$ is invertible everywhere, equation (\ref{eq:dc}) is a one-to-one correspondence in 
      ${(\ell^2(\mathbb{Z}))}^r$ and therefore,
      \begin{align*}
	&\forall \bm{c} \in {(\ell^2(\mathbb{Z}))}^r, \qquad A {\|\bm{c}\|}_{{(\ell^2)}^r} \leq {\left\| 
	\sum_{k=-\infty}^{\infty} \sum_{s=0}^{r-1} c_{s,k} \varphi_s(\cdot-k) \right\|}_{\mathcal{L}^2} \leq B 
	{\|\bm{c}\|}_{{(\ell^2)}^r} \\
	  &  \iff \\
	  & \forall \bm{d} \in {(\ell^{2}(\mathbb{Z}))}^r, \quad A^2 \sum_{s=0}^{r-1} \int_{-\pi}^{\pi} 
	  |\widehat{d_s}(u)|^2 du \leq \sum_{s=0}^{r-1} \int_{-\pi}^{\pi} |\widehat{d_s}(u)|^2 \lambda_s(u) du \leq B^2 
	  \sum_{s=0}^{r-1} \int_{-\pi}^{\pi} |\widehat{d_s}(u)|^2 du
      \end{align*}
      This last equation is equivalent to 
      \begin{equation*} A^2 \leq \essinf_{u \in [-\pi, \pi]} \lambda_{0}(u) \leq \esssup_{u \in [-\pi, \pi]} 
	\lambda_{r-1}(u) \leq B^2
      \end{equation*}
    \end{enumerate}
%  $\implies$ Suppose the collection of functions $\bm{\varphi}$ is a Riesz basis with bounds $0 < A \leq B$, that is to 
%  say (\ref{eq:def-Riesz-r}) holds. Suppose by contradiction that $\esssup_{u \in [-\pi, \pi]} \lambda_{r-1}(u) > B^2$.  
%  Then there exists ${B'}^2 > B^2$ and $E \subset [-\pi, \pi]$ such that $\mu(E) > 0$ and \begin{equation*}
%    \forall u \in E, \quad \lambda_{r-1}(u) \geq B'^{2}
%  \end{equation*}
%  Let $\bm{c}\in {(\ell^{2}(\mathbb{Z}))}^r$ defined in the Fourier domain by 
%  \begin{equation*}
%    \forall u \in \mathbb{R}, \qquad \widehat{\bm{c}}(u) = \begin{bmatrix} 0 & \hdots & 0 & 1 \end{bmatrix} 
%    \matr{Q}^*(u) \chi_{E}(u)
%  \end{equation*}
%  Then,
%  \begin{align*}
%    \frac{1}{2\pi} \int_{-\pi}^{\pi}  \widehat{\bm{c}}(u)\hat{\matr{G}}(u)\widehat{\bm{c}}^*(u) du &= \frac{1}{2\pi} 
%    \int_{E} \lambda_{r-1}(u)du  \\
%    &\geq \frac{\mu(E)}{2\pi}B'^{2}
%  \end{align*}
%  while from Parseval theorem we have
%  \begin{align*}
%    {\|\bm{c}\|}_{{(\ell^2(\mathbb{Z}))}^r}^2 &= \sum_{s=0}^{r-1} \frac{1}{2\pi} \int_{-\pi}^{\pi} 
%    |\widehat{\bm{c_s}}(u)|^2 du \\
%    &= \frac{\mu(E)}{2\pi}
%  \end{align*}
%  Combining these with (\ref{eq:rel-fourier}), we have
%  \begin{equation*}
%    {\left\| \sum_{k=-\infty}^{\infty} \sum_{s=0}^{r-1} c_{s,k} \varphi_s(\cdot-k)\right\|}_{\mathcal{L}^2} \geq {B'} 
%    {\|\bm{c}\|}_{{(\ell^2(\mathbb{Z}))}^r} 
%  \end{equation*}
%  which contradicts our hypothesis of Riesz basis for with upper bound $B$. Assuming similarly that  $\essinf_{u \in 
%  [-\pi, \pi]} \lambda_{0}(u) < A^2$ leads to a contradiction. Therefore, the eigenvalues of $\matr{G}$ are essentially 
%  bounded by $A^2$ and $B^2$ respectively.
\end{proof}

The following Proposition links the Riesz basis property to a property of the map $K$.
\begin{prop}\label{prop:Riesz-K}
  The collection of functions $\bm{\varphi}$ is a Riesz basis if and only if the map $K$ defined in (\ref{def:mapK}) is 
  a homeomorphism (continuous isomorphism with continuous inverse) onto a subspace of $\mathcal{L}^2$.
\end{prop}
\begin{proof}
  $\implies$ Suppose $\exists 0 < A \leq B$ such that
  \begin{equation}\label{eq:def-Riesz-r}
    \forall \bm{c} \in {(\ell^2(\mathbb{Z}))}^r, \qquad A {\|\bm{c}\|}_{{(\ell^2)}^r} \leq {\left\| 
    \sum_{k=-\infty}^{\infty} \sum_{s=0}^{r-1} c_{s,k} \varphi_s(\cdot-k) \right\|}_{\mathcal{L}^2} \leq B 
    {\|\bm{c}\|}_{{(\ell^2)}^r}
  \end{equation}
  Then clearly $K(\bm{c}) = 0$ if and only if $\bm{c} = 0$ i.e $K$ is injective. $K$ is thus a bijection on the subspace 
  $K\left({(\ell^2)}^r\right)$. As $K$ is a linear function on the vector space ${(\ell^2)}^r$, the following holds
  \begin{equation*}
    K \ \text{continuous} \ \iff \exists \vertiii{K} \ \text{such that} \qquad \forall \bm{c} \in {(\ell^2)}^r, 
    \|K(c)\|_{\mathcal{L}^2} \leq \vertiii{K} {\|\bm{c}\|}_{{(\ell^2)}^r}
  \end{equation*}
  The rhs is true by hypothesis with $\vertiii{K}=B$ and therefore $K$ is continuous. Similarly, $K^{-1}$ is linear and 
  continuous with $\vertiii{K^{-1}} = A^{-1}$. \\

  $\impliedby$ Assume that $K$ is a homeomorphism. Reversing the arguments just given, the continuity of $K$ and 
  $K^{-1}$ ensures the existence of $A = \vertiii{K^{-1}}^{-1}$, $B=\vertiii{K}$ such that (\ref{eq:def-Riesz-r}) holds.  
\end{proof}

\subsection{Regular functions in \texorpdfstring{$\mathcal{L}^1$}{g}}

In practice, the functions we shall consider are more than functions in $\mathcal{L}^2$ which simplifies the 
requirements for having a Riesz basis. More precisely, consider the functions $\varphi_0, \ldots, \varphi_{r-1}$ such 
that each $\varphi_s$ is in $L^1$ and is \emph{regular} in the sense that $\hat{\varphi_s}$ is continuous and satisfies 
${\hat{\varphi}_s(u) = o_{|u| \to \infty}\left(\frac{1}{|u|}\right)}$. It may not be clear yet why this is more precise 
than considering functions in $\mathcal{L}^2$ so we state and prove in (\ref{proof:lemmal1l2}) the following result.
\begin{lem}\label{lemma:l1l2}
  If $f \in \mathcal{L}^1(\mathbb{R})$ is such that $\hat{f} \in \mathcal{L}^2(\mathbb{R})$ then $f \in \mathcal{L}^2$.
\end{lem}

By hypothesis, the functions $\varphi_{0}, \ldots, \varphi_{r-1}$ are such that $\hat{\varphi}_s(u) = o_{|u|\to 
\infty}\left(\frac{1}{|u|}\right)$ which implies that $\hat{\varphi}_s \in \mathcal{L}^2$. As a consequence of 
Lemma~\ref{lemma:l1l2}, the functions $\varphi_{0}, \ldots, \varphi_{r-1}$ are all in $\mathcal{L}^2$ which means that 
all results of Subsection~\ref{ssec:l2} are valid. \textbf{Is the map $K$ well-defined?} The continuity of the Fourier 
transforms combined with Proposition~\ref{prop:Riesz-K} and Theorem~\ref{thm:Riesz-Gram} yields the following 
result.

\begin{prop}[{\cite[Corollary 3.1]{GST93}}]\label{prop:K-Gram}
  The map $K$ defined by (\ref{def:mapK}) is an homeomorphism if and only if the Gram matrix $\hat{\matr{G}}$ is 
  positive definite everywhere.
\end{prop}

\begin{proof}
  As the the functions $\hat{\varphi}_s$ are continuous, the eigenvalues $\lambda_0, \ldots, \lambda_{r-1}$ of 
  $\hat{\matr{G}}$ are continuous and $2\pi$-periodic functions.  

  $\implies$ Suppose that $K$ is a homeomorphism. Then by Theorem~\ref{thm:Riesz-Gram}, there exists $0 < A \leq B$ such 
  that the Gram matrix $\hat{\matr{G}}$ has eigenvalues essentially bounded by $A^2$ and $B^2$. However the eigenvalues 
  are continuous so that the bounds hold \emph{everywhere} which in turn implies that $\hat{\matr{G}}$ is positive 
  definite \emph{everywhere}.

  $\impliedby$ Suppose the Gram matrix is positive definite everywhere. The functions $\lambda_0, \ldots, \lambda_{r-1}$ 
  are then strictly positive and continuous on the compact $[-\pi, \pi]$. As any continuous function on a compact 
  reaches its bounds, there exists $0 < A \leq B$ such that
  \begin{equation*}
    A^2 = \inf_{u \in [-\pi, \pi]} \lambda_0(u) \leq \sup_{u \in [-\pi, \pi]} \lambda_{r-1}(u) = B^2
  \end{equation*}
  By $2\pi$-periodicity this holds everywhere. As a consequence of Theorem~\ref{thm:Riesz-Gram} and 
  Proposition~\ref{prop:Riesz-K},  $K$ is a homeomorphism.
\end{proof}

Here is an other and final result for characterizing the Riesz basis property.
\begin{prop}[{\cite[Theorem 3.2]{GST93}}]\label{prop:Gram-H}
  Let $u \in \mathbb{R}$. The Gram matrix $\hat{\matr{G}}(u)$ is positive definite if and only if the infinite matrix
  \begin{equation}
    \hat{\matr{Z}}(u) = {\left(\hat{\varphi}_s(u+2k\pi)\right)}_{s \in \llbracket0,r-1\rrbracket, k \in \mathbb{Z}}
  \end{equation}
  has linearly independent rows.
\end{prop}

\begin{proof}
  Let $u \in \mathbb{R}$. Observe that the Gram matrix $\hat{\matr{G}}(u)$ is related to $\hat{\matr{Z}}(u)$ by
  \begin{equation*}
    \hat{\matr{G}}(u) = \hat{\matr{Z}}(u) \hat{\matr{Z}}(u)^*
  \end{equation*}
  Therefore, for any $\bm{v} \in \mathbb{C}^r$, we have
  \begin{equation*}
    \bm{v}\hat{\matr{G}}(u)\bm{v}^* = \sum_{k=-\infty}^{\infty} \left| \sum_{s=0}^{r-1} v_s
    \hat{\varphi}_s(u+2k\pi)\right|^2
  \end{equation*}
  Then,
  \begin{align*}
    \hat{\matr{G}}(u) \ \text{positive definite} \ &\iff \forall v \in \mathbb{C}^r\setminus\{0\}, \quad 
    \sum_{k=-\infty}^{\infty} \left| \sum_{s=0}^{r-1} v_s \hat{\varphi}_s(u+2k\pi)\right|^2  > 0 \\
    & \iff \forall v \in \mathbb{C}^r\setminus\{0\}, \exists k \in \mathbb{Z}, \quad \sum_{s=0}^{r-1} v_s 
    \hat{\varphi}_s(u+2k\pi) \neq 0 \\
    & \iff \text{rows of $\hat{\matr{Z}}(u)$ are linearly independent}
  \end{align*}
\end{proof}

In Theorem~\ref{thm:Lee} an exact formula for the Fourier transform of the Hermite B-splines was found. This formula 
however is unnecessarily complex because the constant term ${(-j)}^s K(m,r)$ does not affect the Riesz basis property.  
More precisely, the following technical lemma holds.

\begin{lem}\label{lem:eqbasis}
  Let $\varphi_0, \ldots, \varphi_{r-1} \in \mathcal{L}^2$ such that $\bm{\varphi} := \{\varphi_s(\cdot-k) | s \in 
  \llbracket0, r-1\rrbracket, k \in \mathbb{Z}\}$ is a Riesz basis for the subspace $V$ of $\mathcal{L}^2$ it spans.  
  Let $\gamma_0, \ldots, \gamma_{r-1}$ non-zero complex numbers and define \begin{equation*}
    \forall s \in \llbracket0,r-1\rrbracket, \quad \tilde{\varphi}_s = \gamma_s \varphi_s
  \end{equation*}
  Then the collection of functions $\bm{\tilde{\varphi}} := \{\tilde{\varphi_s}(\cdot-k) | s \in \llbracket0, 
  r-1\rrbracket, k \in \mathbb{Z}\}$ is a Riesz basis for $V$.
\end{lem}

\begin{proof}
  Let $\bm{c} \in {(\ell^2)}^r$ and let $\tilde{\bm{c}} \in {(\ell^2)}^r$ defined by $\tilde{\bm{c_s}} = \gamma_s 
  \bm{c_s}$ for $s \in \llbracket0,r-1\rrbracket$. As $\bm{\varphi}$ is a Riesz basis there exist constants $0 < A \leq 
  B$ such that
  \begin{equation*}
    A {\|\tilde{\bm{c}}\|}_{{(\ell^2)}^r}\leq {\left\| \sum_{k=-\infty}^{\infty} \sum_{s=0}^{r-1} c_{s,k} 
    \tilde{\varphi}_s(\cdot-k) \right\|}_{\mathcal{L}^2} = {\left\| \sum_{k=-\infty}^{\infty} \sum_{s=0}^{r-1} 
    \tilde{c}_{s,k} \varphi_s(\cdot-k) \right\|}_{\mathcal{L}^2} \leq B{\|\tilde{\bm{c}}\|}_{{(\ell^2)}^r}
  \end{equation*}
  Observe now that
  \begin{align*}
    {\|\tilde{\bm{c}}\|}_{{(\ell^2)}^r}^2 &= \sum_{s=0}^{r-1} \sum_{k=-\infty}^{\infty} \|\tilde{c_{s,k}}\|^2 \\
    &= \sum_{s=0}^{r-1} |\gamma_s|^2 \sum_{k=-\infty}^{\infty} \|\tilde{c_{s,k}}\|^2
  \end{align*}
  which implies that
  \begin{equation*}
    \min_{s=0, \ldots, r-1} |\gamma_s| {\|{\bm{c}}\|}_{{(\ell^2)}^r} \leq {\|\tilde{\bm{c}}\|}_{{(\ell^2)}^r}
   \leq \max_{s=0, \ldots, r-1} |\gamma_s| {\|{\bm{c}}\|}_{{(\ell^2)}^r}
  \end{equation*}
\end{proof}

From Lemma~\ref{lem:Ns}, the Hermite B-splines $N_0, \ldots, N_{r-1}$ are continuous and \emph{regular} i.e they satisfy 
the assumptions we previously had on $\bm{\varphi}$. As a consequence all results of the Subsection
holds. These results allows us to prove that $\bm{N} := \{N_{s}| s\in \llbracket0,r-1\rrbracket, k \in \mathbb{Z}\}$ is 
a Riesz basis, a result that it is central to our work.

\begin{thm}
  The collection of Hermite B-splines $\bm{N} := \{N_{s}| s\in \llbracket0,r-1\rrbracket, k \in \mathbb{Z}\}$ is a Riesz 
  basis for the subspace of $\mathcal{L}^2$
  \begin{equation*}
    V_{\bm{N}} = K_{\bm{N}}\left({(\ell^2)}^r\right)
  \end{equation*}
  with $K_{\bm{N}}$ the map
  \begin{align*}
    {(\ell^2)}^r & \longrightarrow  \mathcal{L}^2 \\
    \bm{c} & \longmapsto \sum_{k=-\infty}^{\infty} \sum_{s=0}^{r-1} c_{s,k} N_s(\cdot-k)
  \end{align*}
\end{thm}

\begin{proof}
  \begin{enumerate}
    \item Given Lemma~\ref{lem:eqbasis}, it is equivalent to prove that the functions $\tilde{N}_0, \ldots, 
      \tilde{N}_{r-1}$ defined in the Fourier domain by
      \begin{equation*}
	\widehat{\tilde{N}}(u) = {\left(2\sin \frac{u}{2}\right)}^{2m} H_{r,s}(\alpha_{2m}(u))
      \end{equation*}
    are such that $\bm{\tilde{N}} := \{\tilde{N}_s(\cdot-k) | s \in \llbracket0, r-1\rrbracket, k \in \mathbb{Z}\}$ is a 
    Riesz basis.
    \item To prove that $\bm{\tilde{N}}$ is a Riesz basis, we shall prove that the infinite matrix 
      $\hat{\matr{Z}}_{\bm{\tilde{N}}}$ given by
      \begin{equation*}
	\forall u \in \mathbb{R}, \quad \hat{\matr{Z}}_{\bm{\tilde{N}}}(u) = 
	{\left(\widehat{\tilde{N}}_s(u+2k\pi)\right)}_{s \in \llbracket0,r-1\rrbracket, k \in \mathbb{Z}}
      \end{equation*}
      has linearly independent rows everywhere. Then, the successive application of 
      Propositions~\ref{prop:Riesz-K},~\ref{prop:K-Gram} and~\ref{prop:Gram-H} will prove the desired result. 

    \item  Let $u \in  \mathbb{R}\setminus{2\pi\mathbb{Z}}$, $s \in \llbracket0,r-1\rrbracket$. Let $\matr{H_r(u)} := 
      \matr{H_r}({\bm{\alpha}}_{2m}(u)), \matr{H_{r,s}(u)} := \matr{H_{r,s}}({\bm{\alpha}}_{2m}(u))$ the Hankel and 
      modified Hankel matrices for $\bm{\alpha}$ and $H_{r}(u), H_{r,s}(u)$ the associated determinants.  For $(i,j) \in 
      {\llbracket0,r-1\rrbracket}^2$, let $H^{(i,j)}_r(u)$ the determinant of the submatrix of $\matr{H_r(u)}$ with the 
      $(i+1)^{th}$ row and the $(j+1)^{th}$ column deleted.  Developing the determinant, the Fourier transform of 
      $\tilde{N}_{s}$ at u is
      \begin{equation*}
	\widehat{\tilde{N}}_s(u) =\left(2\sin \frac{u}{2}\right)^{2m} \sum_{i=0}^{r-1} {(-1)}^{i+s} H_r^{(i,s)}(u) 
	\frac{1}{u^{2m-i}}
      \end{equation*}
      Let's rewrite this with the help of the transpose of the comatrix.
      \begin{align*}
	\left[\com \matr{H_r(u)}^T \begin{bmatrix} \frac{1}{u^{2m}} \\ \vdots \\ \frac{1}{u^{2m-r+1}} 
	\end{bmatrix}\right]_s &= \sum_{i=0}^{r-1} \left[\com \matr{H_r(u)}^T\right]_{s,i} \frac{1}{u^{2m-i}} \\
	&= \sum_{i=0}^{r-1} {(-1)}^{i+s} H_r^{(i,s)}(u) \frac{1}{u^{2m-i}} \\
	&= \widehat{\tilde{N}}_s(u)
    \end{align*}
    From (\ref{eq:relHrPi}), the determinant $H_r(u)$ can only vanish if $\Pi_{2m-1,r}(e^{ju})$ vanish which does not 
    happen for $u \in  \mathbb{R}\setminus{2\pi\mathbb{Z}}$ as none of the roots of $\Pi_{2m-1,r}$ has modulus 1.  
    Consequently, $\matr{H_r}(u)$ is invertible and ${\matr{H_r}(u)}^{-1} = \frac{1}{H_r(u)} \com \matr{H_r}(u)^T$.  
    Eventually, the following holds
    \begin{equation*}
      \begin{bmatrix} \widehat{\tilde{N}}_0(u) \\ \vdots \\ \widehat{\tilde{N}}_{r-1}(u) \end{bmatrix} = \left(2\sin 
      \frac{u}{2}\right)^{2m} {\matr{H_r}(u)}^{-1} H_r(u) \begin{bmatrix} \frac{1}{u^{2m}} \\ \vdots \\ 
      \frac{1}{u^{2m-r+1}} \end{bmatrix}
    \end{equation*}
    Rewriting this equation at $2\pi$-translations of $u$ we have
    \begin{equation*}\scriptstyle
      \begin{bsmallmatrix} \widehat{\tilde{N}}_0(u) & \hdots & \widehat{\tilde{N}}_0(u+2(r-1)\pi)  \\ \vdots & \ddots & 
      \vdots \\ \widehat{\tilde{N}}_{r-1}(u) & \hdots & \widehat{\tilde{N}}_{r-1}(u + 2(r-1)\pi) \end{bsmallmatrix} = 
      \left(2\sin \frac{u}{2}\right)^{2m} {\matr{H_r}(u)}^{-1} H_r(u) \begin{bmatrix} \frac{1}{u^{2m}} & \hdots &  
      \frac{1}{{(u+2(r-1)\pi)}^{2m}} \\ \vdots & \ddots & \vdots \\ \frac{1}{u^{2m-r+1}} & \hdots & 
    \frac{1}{{(u+2(r-1)\pi)}^{2m-r+1}} \end{bmatrix}
    \end{equation*}
    The Vandermondian matrix on the right has full rank and therefore so does the matrix on the left. Noticing that the 
    matrix on the left is a submatrix of $\hat{\matr{Z}}_{\bm{\tilde{N}}}(u)$,  we conclude that 
    $\hat{\matr{Z}}_{\bm{\tilde{N}}}(u)$ must have independent rows at all $u \in \mathbb{R}\setminus{2\pi\mathbb{Z}}$.
  \end{enumerate}
\end{proof}

\subsection{Vectorial sequences of coefficients}

\section{Support and approximation}

\begin{example}
In the Fourier domain the basis functions are expressed by
\begin{align*}
  \hat{L}_0(u) &= \int_{\mathbb{R}} \phi_1(t) e^{-jut} dt = \frac{-12}{u^4}(u\sin(u) + 2\cos(u) - 2) \\
  \hat{L}_1(u) &= \int_{\mathbb{R}} \phi_2(t) e^{-jut} dt = \frac{-4j}{u^4}(u\cos(u) - 3\sin(u) + 2u) \\
\end{align*}
Define $\rho_{D^4}, \rho_{D^3}$ the Green's functions (\ref{def:Green}) of the operators $D^4, D^3$. They are given in 
the Fourier domain by
\begin{align*}
  \hat{\rho}_{D^4}(u) &= \frac{1}{(ju)^4} \\
  \hat{\rho}_{D^3}(u) &= \frac{1}{(ju)^3} \\
\end{align*}
Let $\matr{L} = \begin{bmatrix} L_0 \\ L_1 \end{bmatrix}$, $\bm{\rho} = \begin{bmatrix} \rho_{D^4} \\ \rho_{D^3} 
\end{bmatrix}$.  Then,
\begin{equation*}
  \forall u \in \mathbb{R}, \qquad \hat{\matr{L}}(u) = \hat{\matr{R}}(u) \hat{\bm{\rho}}(u), \quad \hat{\bm{\rho}}(u) = 
  \hat{\matr{S}}(u) \hat{\matr{L}}(u)
\end{equation*}
with $\hat{\matr{S}}=\hat{\matr{R}}^{-1}$ the inverse of $\hat{\matr{R}}$. Decomposing we have
\begin{align*}
  \hat{\rho}_{D^4} &= \hat{s_{0,0}} \hat{L}_0 + \hat{s_{0,1}} \hat{L}_1, \\
  \rho_{D^4} &= s_{0,0} * L_0 + s_{0,1} * L_0.
\end{align*}

\begin{remark}
  The convolutions above are are mixed convolutions, because $s_{0,0}, s_{0,1} : \mathbb{Z} \to \mathbb{C}$ and $L_{0}, 
  L_{1} : \mathbb{R} \to \mathbb{C}$.  Mixed convolution is defined as
  \begin{equation*}
    s_{0,0} * L_0 (t) = \sum_{k \in \mathbb{Z}} s_{0,0}[k] L_0(t-k).  
  \end{equation*}
    The property of Fourier transform over a convolution product holds.
\end{remark}

As the cubic splines are generated by ${(\rho_{D^4}(.-k))}_{k \in \mathbb{Z}}$ and the quadratic are generated by 
${(\rho_{D^2}(.-k))}_{k \in \mathbb{Z}}$, cubic and quadratic splines are reproduced by cubic Hermite splines.

In the same fashion, cubic B-splines $\beta^3_+ = \Delta^4_+ \rho_{D^4}$ and quadratic B-splines $\beta^2_+ = \Delta^3_+ 
\rho_{D^3}$ are readily expressed as finite combinations of $L_0$ and $L_1$.
\end{example}



