\documentclass[a4paper, 11pt]{article}

\usepackage[left=1.5cm, right=1.5cm, top=2cm, bottom=2cm]{geometry}

\usepackage[utf8]{inputenc}
\usepackage[T1]{fontenc}
\usepackage[english]{babel}  
\usepackage{lmodern}

\usepackage{amsmath, mathtools, amsthm, amssymb} % amsthm contains the proof environment
\usepackage{amsfonts} % for the \mathfrak command
\usepackage{mathrsfs} % for the \mathscr command
\usepackage{stmaryrd} % for \llbracket
\SetSymbolFont{stmry}{bold}{U}{stmry}{m}{n}

\usepackage{graphicx}
\usepackage{geometry}
\usepackage{dsfont}
\usepackage{bm}

\newtheorem{innercustomgeneric}{\customgenericname}
\providecommand{\customgenericname}{}
\newcommand{\newcustomtheorem}[2]{%
  \newenvironment{#1}[1]
  {%
   \renewcommand\customgenericname{#2}%
   \renewcommand\theinnercustomgeneric{##1}%
   \innercustomgeneric}
  {\endinnercustomgeneric}
}

\newtheorem{thm}{Theorem}
\newtheorem{lem}{Lemma}
\newtheorem{cor}{Corollary}
\newtheorem{deftn}{Definition}
\newtheorem{prop}{Proposition}
\newtheorem{remark}{Remark}
\newtheorem{example}{Example}

\DeclareMathOperator*{\argmin}{argmin} 
\DeclareMathOperator*{\argmax}{argmax}
\DeclareMathOperator*{\essinf}{ess\ inf}
\DeclareMathOperator*{\esssup}{ess\ sup}
\DeclareMathOperator*{\Supp}{Supp}
\DeclareMathOperator*{\com}{com}
\DeclareMathOperator*{\Tr}{Tr} 

\renewcommand{\thesection}{\Roman{section}}

\usepackage[backend=bibtex, citestyle=authoryear, sorting=nyt]{biblatex} % Use the bibtex backend 
\usepackage[autostyle=true]{csquotes} % Required to generate language-dependent quotes in the bibliography
\usepackage{filecontents}

\begin{filecontents}{xmpl.bib}
@article{Me02,
  shorthand = {Me02},
  author = {Meijering, E.},
  title = {A Chronology of Interpolation: From Ancient Astronomy to Modern and Signal Image processing},
  year = {2002},
  journal = {Proceedings of the IEEE},
  volume = {90},
  number = {3},
  pages = {319-342},
}

@book{Bo01,
  shorthand = {Bo01},
  author = {De Boor, C.},
  title = {A practical guide to splines},
  year = {2001},
  publisher = {Springer},
  edition = {Revised edition}
}

@article{CS66,
  shorthand = {CS66},
  author = {Curry, H.B. and Schoenberg, I.J.},
  title = {On Pólya frequency functions IV: the fundamental spline functions and their limits},
  year = {1966},
  journal = {J. Analyse Math.},
  volume = {17},
  pages = {71-107},
}

@article{Sch46,
  shorthand = {Sch46},
  author = {Schoenberg, I.J.},
  title = {Contributions to the problem of approximation of equidistant data by analytic functions.},
  year = {1946},
  journal = {Quart. Appl. Math.},
  volume = {IV},
  number = {1},
  pages = {45-99},
}

@article{Sch69,
  shorthand = {Sch69},
  author = {Schoenberg, I.J.},
  title = {Cardinal interpolation and spline functions},
  year = {1969},
  journal = {J. Approx. Theory},
  volume = {2},
  pages = {167-206},
}

@article{Sch72a,
  shorthand = {Sch72a},
  author = {Schoenberg, I.J.},
  title = {Cardinal interpolation and spline functions II: interpolation of data of power growth},
  year = {1972},
  journal = {J. Approx. Theory},
  volume = {6},
  pages = {404-420},
}

@article{LS73,
  shorthand = {LS73},
  author = {Lipow, P.R. and Schoenberg, I.J.},
  title = {Cardinal Interpolation and Spline Functions. III\@. Cardinal Hermite Interpolation},
  year = {1973},
  journal = {Linear Algebra and its Applications},
  volume = {6},
  pages = {273-304},
}

@article{Sch72b,
  shorthand = {Sch72b},
  author = {Schoenberg, I.J.},
  title = {Cardinal interpolation and spline functions IV: the exponential Euler splines},
  year = {1972},
  journal = {International series of numerical mathematics},
  volume = {20},
  pages = {382-404},
}

@article{SS73,
  shorthand = {SS73},
  author = {Schoenberg, I.J. and Sharma, A.},
  title = {Cardinal Interpolation and Spline Functions. V\@. The B-splines for cardinal Hermite interpolation},
  year = {1973},
  journal = {Linear Algebra and its Applications},
  volume = {7},
  number = {1},
  pages = {1-42},
}

@article{Sch73,
  shorthand = {Sch73},
  author = {Schoenberg, I.J.},
  title = {Cardinal Spline Interpolation},
  year = {1973},
  journal = {CBMS-NSF Regional conference series in applied mathematics},
}

@article{Her78,
  shorthand = {Her78},
  author = {Hermite, C.},
  title = {Sur la formule d'interpolation de Lagrange},
  year = {1878},
  journal = {Journal für die Reine und Angewandte Mathematik},
  volume = {84},
  number = {1},
  pages = {70-79}
}

@article{Lee73,
  shorthand = {Lee73},
  author = {Lee, S. L.},
  title = {B-splines for cardinal Hermite interpolation},
  year = {1975},
  journal = {Linear Algebra and its Applications},
  volume = {12},
  pages = {269-280}
}

@article{Lee76a,
  shorthand = {Lee76a},
  author = {Lee, S. L.},
  title = {Exponential Hermite-Euler splines},
  year = {1976},
  journal = {J. Approx. Theory.},
  volume = {18},
  pages = {205-212}
}

@article{Lee76b,
  shorthand = {Lee76b},
  author = {Lee, S. L.},
  title = {Fourier transforms of B-splines and fundamental splines for cardinal Hermite interpolation},
  year = {1976},
  journal = {Proceedings of the American mathematical society},
  volume = {57},
  pages = {291-296}
}

@article{LeeSh76,
  shorthand = {LeeSh76},
  author = {Lee, S. L. and Sharma, A.},
  title = {Cardinal Lacunary interpolation by g-splines. I. The characteristic polynomial},
  year = {1976},
  journal = {J. Approx. Theory},
  volume = {16},
  pages = {85-96}
}

@article{FAUU19,
  shorthand = {FAUU73},
  author = {Fageot, Julien and Aziznejad, Shayan and Unser, Michael and Uhlmann, Virginie},
  title = {Support and approximation properties of Hermite splines},
  year = {2019},
  journal = {arXiv:1902.02565 [math.NA]}
}

\end{filecontents}

\addbibresource{xmpl.bib} % The filename of the bibliography


\begin{document}
\title{Fourier transforms of $N_s$ and Lee's formula}
\author{Yoann Pradat}
\maketitle
\tableofcontents

\section{The proof}

\underline{Notations}
\begin{itemize}
  \item $m, r \in \mathbb{N}^*$ with $m \geq r$
  \item $N_s := N_{2m,r,s}$, $s=0, \ldots, r-1$ are Hermite B-splines
  \item $L_s := L_{2m,r,s}$, $s=0, \ldots, r-1$ are Schoenberg fundamental splines
\end{itemize}

Given these notations, Lee stated and proved the following theorem
\begin{thm}[{\cite[Theorem 1]{Lee76b}}]\label{thm:Lee1}
  The Fourier transform of the fundamental splines is
  \begin{equation}\label{eq:Lee1}
    \hat{L}_{2m,r,s}(u) = {(-j)}^{s} \frac{H_{r,s}(\alpha_{2m}(u))}{H_r(\alpha_{2m}(u))}
  \end{equation}
\end{thm}

\begin{proof}
  Let's first prove a property of the Hankel determinant that we will use later. For any sequence ${(a_n)}_n$ and any 
  complex $\mu$ the following holds
  \begin{equation}\label{prop:hankel}
    H_r(a_n \mu^n) = \mu^{r(n-r+1)} H_r(a_n)
  \end{equation}
  This is easily proved using the Leibniz formula for the determinant as follows
  \begin{align*}
    H_r(a_n \mu^n) &= \sum_{\sigma \in \mathfrak{S}_r} \epsilon(\sigma) \prod_{j=1}^r \mu^{n-\sigma(j) - j+2} 
    a_{n-\sigma(j)-j+2} \\
    &= \mu^{rn-(\sum_{j=1}^r \sigma(j) + j) + 2r} \sum_{\sigma \in \mathfrak{S}_r} \epsilon(\sigma) \prod_{j=1}^r  
    a_{n-\sigma(j)-j+2} \\
    &= \mu^{rn-r(r+1) + 2r} \sum_{\sigma \in \mathfrak{S}_r} \epsilon(\sigma) \prod_{j=1}^r  a_{n-\sigma(j)-j+2} \\
    &= \mu^{r(n-r+1)} H_r(a_n)
  \end{align*}
  
  Let $m, r \in \mathbb{N}^*$ such that $m \geq r$ and let $s \in \llbracket0, r-1\rrbracket$. The key observation for 
  the proof of the Fourier transform of $L_{2m,r,s}$ is that the $(s+1)^{th}$ fundamental spline can be written as the 
  integral of the $(s+1)^{th}$ exponential Euler-Hermite splines of order $n+1=2m$.  By definition, the $(s+1)^{th}$ 
  exponential Euler-Hermite splines of order $n$ for the base $\lambda$ ($\Pi_{n,r}(\lambda) \neq 0$) is given by
  \begin{align}
    S_{n+1,r,s}(x;\lambda) &= \frac{A_{n,r,s}(x; \lambda)}{A_{n,r,s}^{(s)}(x; \lambda)}, \quad 0 \leq x \leq 1 \\
    S_{n+1,r,s}(x+1;\lambda) &= \lambda S_{n+1,r,s}(x;\lambda), \quad \forall x \in \mathbb{R}
  \end{align}
   
  Consider the $r$ functions
  \begin{equation}\label{eq:def-I}
    I_{2m,r,s}(x) = \begin{dcases}
      \frac{1}{2\pi} \int_0^{2\pi} S_{2m, r, s}(x;e^{ju}) du & \text{if $r$ even} \\
      \frac{1}{2\pi} \int_{-\pi}^{\pi} S_{2m, r, s}(x;e^{ju}) du & \text{if $r$ odd} \\
    \end{dcases}
  \end{equation}
  The functions $S_{2m,r,s}(\cdot;\lambda)$ being in $\mathscr{S}_{2m,r}^{(s)}$ so is $I_{2m,r,s}$. Given the properties 
  of the derivatives of $S_{2m,r,s}$, the following holds 
  \begin{equation*}
    \forall k \in \mathbb{Z}, \quad 
    \begin{dcases}
      I_{2m,r,s}^{(\rho)}(k) = 0 & \ \rho=0, \ldots, r-1, \rho \neq s,  \\
      I_{2m,r,s}^{(s)}(k) = \delta_k &
    \end{dcases}
  \end{equation*}
  
  However there is only one element in $\mathscr{S}_{2m,r}$ that satisfies such conditions and that element is 
  $L_{2m,r,s}$ by definition. As a consequence,
  \begin{equation}
    L_{2m,r,s} = I_{2m,r,s}, \quad s=0, \ldots, r-1
  \end{equation}
  
  From (\cite[(7.14)]{Sch72b}), the Euler-Frobenius polynomial can be written as \begin{equation}\label{eq:sch-an}
      \frac{A_n(x;e^{ju})}{n!} = (e^{ju}-1)e^{-ju}e^{jux} \sum_{k=-\infty}^{\infty} \frac{e^{2j\pi kx}}{{(ju + 
      2jk\pi)}^{n+1}}, \quad e^{ju} \neq 1, 0 \leq x \leq 1
  \end{equation}
  
  Using the multilinearity of the determinant $A_{n,r,s}$ and using the equation (\ref{prop:hankel}), the numerator of 
  $S_{n+1,r,s}$ can be written as \begin{align*}
    A_{n,r,s}(x;e^{ju}) &= \frac{(e^{ju}-1)}{e^{ju}}e^{jux} \sum_{k=-\infty}^{\infty} e^{2j\pi kx}
    \begin{vmatrix}
      \frac{A_n(0;e^{ju})}{n!}  &  \hdots & \frac{1}{{(ju+2jk\pi)}^{n+1}} & \hdots &  
      \frac{A_{n-r+1}(0;e^{ju})}{(n-r+1)!} \\
      \vdots & & \vdots & & \vdots \\
      \frac{A_{n-r+1}(0;e^{ju})}{(n-r+1)!}  & \hdots & \frac{1}{{(ju+2jk\pi)}^{n-r+2}} & \hdots &  
      \frac{A_{n-2r+2}(0;e^{ju})}{(n-2r+2)!} \\
    \end{vmatrix} \\
    &= \frac{{(e^{ju}-1)}^r}{e^{jur}}e^{jux} {(-j)}^{s} j^{r(n-r+1)} \sum_{k=-\infty}^{\infty} e^{2j\pi kx} 
    \Delta_{n+1,k,s}(u)
  \end{align*}
  with $\Delta_{n+1,k,s}(u)$ the Hankel determinant of order $r$ of $\alpha_{n+1}(u)$ with its ${(s+1)}^{th}$ column 
  replaced by the vector
  \begin{equation*}
    \begin{bmatrix} \frac{1}{{(u+2k\pi)}^{n+1}} & \frac{1}{{(u+2k\pi)}^{n}} & \hdots & \frac{1}{{(u+2k\pi)}^{n-r+2}} 
    \end{bmatrix}^T
  \end{equation*}
  Similarly, the denominator of $S_{n,r,s}$ can be written as 
  \begin{align*}
    A^{(s)}_{n,r,s}(0;e^{ju}) &= H_r\left(\frac{A_n(0;\lambda)}{n!}\right) \\
    & = \frac{{(e^{ju}-1)}^r}{e^{jur}}{(-j)}^{r(n-r+1)} H_r(\alpha_{n+1}(u))
  \end{align*}
  
  As a consequence, \begin{equation}\label{eq:exp-s2mrs}
    S_{n+1,r,s}(x;e^{ju}) = {(-j)}^s e^{jux} \sum_{k=-\infty}^{\infty}  e^{2j\pi kx} 
    \frac{\Delta_{n+1,k,s}(u)}{H_r(\alpha_{n+1}(u))}
  \end{equation}
  
  Let now replace the expression (\ref{eq:exp-s2mrs}) into (\ref{eq:def-I}). The details are given for $r$ even as the 
  calculations are similar for $r$ odd.
  \begin{align*}
    I_{2m,r,s}(x) &= \frac{1}{2\pi} \int_0^{2\pi} S_{2m, r, s}(x;e^{ju}) du \\
    &= \frac{{(-j)}^s}{2\pi} \int_0^{2\pi} e^{jux} \sum_{k=-\infty}^{\infty}  e^{2j\pi kx} 
    \frac{\Delta_{2m,k,s}(u)}{H_r(\alpha_{2m}(u))} du \\
    &= \frac{{(-j)}^s}{2\pi} \sum_{k=-\infty}^{\infty} \int_0^{2\pi} e^{j(u+2k\pi) x}  
    \frac{\Delta_{2m,k,s}(u)}{H_r(\alpha_{2m}(u+2k\pi))} du \quad (\alpha_{2m} \text{is $2\pi$-periodic}) \\
    &= \frac{{(-j)}^s}{2\pi} \int_{-\infty}^{\infty} e^{jux} \frac{H_{r, s}(\alpha_{2m}(u))}{H_r(\alpha_{2m}(u))} du
  \end{align*}
\end{proof}

\begin{thm}[{\cite[Theorem 2]{Lee76b}}]\label{thm:Lee2}
  The Fourier transform of the Hermite B-splines is
  \begin{equation}\label{eq:Lee2}
    \hat{N}_s(u) = {(-j)}^s K(m,r) {\left(2 \sin \frac{u}{2} \right)}^{2m} H_{r,s}(\alpha_{2m}(u))
  \end{equation}
  with $K(m,r) = {(-1)}^{m(r+1)} \frac{(2m-1)!(2m-2)!\ldots(2m-r)!}{1!2!\ldots(r-1)!}$
\end{thm}

\begin{remark}
  From Theorem~\ref{thm:Lee1} and the definition
  \begin{equation}
    N_s = \sum_{k=-(m-r)}^{(m-r)} c_k L_s(\cdot-k)
  \end{equation}
  $N_s$ can be written in the following integral representation
  \begin{equation}
    N_s(x) = \frac{{(-j)}^s}{2\pi} \int_{-\infty}^{\infty} e^{-ju(m-r)} \Pi_{2m-1,r}(e^{ju}) e^{jux} \frac{H_{r, 
    s}(\alpha_{2m}(u))}{H_r(\alpha_{2m}(u))} du
  \end{equation}
\end{remark}

\begin{example}
  \begin{enumerate}
    \item In the case where $m=r$, we know that $N_0 = (2m-1)! M_{2m}$ i.e we expect \begin{equation}
	\hat{N_0}(u) = (2m-1)!{\left(\frac{2\sin(\frac{u}{2})}{u}\right)}^{2m}
      \end{equation}

    \item Assume $m \geq 1, r=1$. Lee's formula is
      \begin{equation}
	\hat{N_0}(u) = (2m-1)! \times {\left(2 \sin \frac{u}{2} \right)}^{2m} \frac{1}{u}^{2m} 
      \end{equation}

    \item Assume $m=2, r=2$. Lee's formula is 
      \begin{align*}
	\hat{N_0}(u) &= 3!2! {\left(2 \sin \frac{u}{2} \right)}^{4}   
	\begin{vmatrix}
	  \frac{1}{u^4} & \sum_{k} \frac{1}{{(u+2k\pi)}^{3}} \\
	  \frac{1}{u^3} & \sum_{k} \frac{1}{{(u+2k\pi)}^{2}} \\
	\end{vmatrix} \\
      &=  3!2! {\left(2 \sin \frac{u}{2} \right)}^{4}  \frac{1}{u^6} \left[ 1+ \sum_{k \neq 0} 
      \frac{u^2}{{(u+2k\pi)}^{2}} - (1+ \sum_{k \neq 0} \frac{u^3}{{(u+2k\pi)}^{3}})  \right] \\
      &=  3!2! {\left(2 \sin \frac{u}{2} \right)}^{4}  \frac{1}{u^4} \left[\sum_{k \neq 0} \frac{1}{{(u+2k\pi)}^{2}} - 
      \sum_{k \neq 0} \frac{u}{{(u+2k\pi)}^{3}} \right]
      \end{align*}
      while
      \begin{align*}
	\hat{N_1}(u) &= (-j) 3!2! {\left(2 \sin \frac{u}{2} \right)}^{4}   \begin{vmatrix}
	  \sum_{k} \frac{1}{{(u+2k\pi)}^{4}} & \frac{1}{u^4}  \\
	  \sum_{k} \frac{1}{{(u+2k\pi)}^{3}} & \frac{1}{u^3} \\
	\end{vmatrix} \\
	&= (-j) 3!2! {\left(2 \sin \frac{u}{2} \right)}^{4}  \frac{1}{u^7} \left[ 1+ \sum_{k \neq 0} 
	\frac{u^4}{{(u+2k\pi)}^{4}} - (1+ \sum_{k \neq 0} \frac{u^3}{{(u+2k\pi)}^{3}})  \right] \\
	&= (-j) 3!2! {\left(2 \sin \frac{u}{2} \right)}^{4}  \frac{1}{u^4} \left[\sum_{k \neq 0} 
	\frac{u}{{(u+2k\pi)}^{4}} - \sum_{k \neq 0} \frac{1}{{(u+2k\pi)}^{3}} \right]
      \end{align*}

    \item Assume $m \geq 2, r=2$.
      \begin{itemize}
	\item \underline{Around 0}
	  \begin{align*}
	    \hat{N_0}(u) &= K(m,2) {\left(2 \sin \frac{u}{2} \right)}^{2m}   
	      \begin{vmatrix}
		\frac{1}{u^{2m}} & \sum_{k} \frac{1}{{(u+2k\pi)}^{2m-1}} \\
		\frac{1}{u^{2m-1}} & \sum_{k} \frac{1}{{(u+2k\pi)}^{2m-2}} \\
	      \end{vmatrix} \\
	      &=  K(m,2) {\left(2 \sin \frac{u}{2} \right)}^{2m} \frac{1}{u^{4m-2}} \left[ 1+ \sum_{k \neq 0} 
	      \frac{u^{2m-2}}{{(u+2k\pi)}^{2m-2}} - (1+ \sum_{k \neq 0} \frac{u^{2m-1}}{{(u+2k\pi)}^{2m-1}})  \right] \\
	      &=  K(m,2) {\left(2 \sin \frac{u}{2} \right)}^{2m} \frac{1}{u^{2m}}\left[\beta_{2m-2}(u) - u 
	      \beta_{2m-1}(u) \right]
	    \end{align*}
	    while
	    \begin{align*}
	      \hat{N_1}(u) &= (-j)K(m,2) {\left(2 \sin \frac{u}{2} \right)}^{2m}   
	      \begin{vmatrix}
		\sum_{k} \frac{1}{{(u+2k\pi)}^{2m}} & \frac{1}{u^{2m}}  \\
		\sum_{k} \frac{1}{{(u+2k\pi)}^{2m-1}} & \frac{1}{u^{2m-1}}  \\
	      \end{vmatrix} \\
	      &=  (-j)K(m,2) {\left(2 \sin \frac{u}{2} \right)}^{4m-2} \frac{1}{u^{4m-1}}\left[ 1+ \sum_{k \neq 0} 
	      \frac{u^{2m}}{{(u+2k\pi)}^{2m}} - (1+ \sum_{k \neq 0} \frac{u^{2m-1}}{{(u+2k\pi)}^{2m-1}})  \right] \\
	      &=  (-j)K(m,2) {\left(2 \sin \frac{u}{2} \right)}^{2m} \frac{1}{u^{2m}}\left[u\beta_{2m}(u) - 
	    \beta_{2m-1}(u) \right]
	  \end{align*}
	\item \underline{Around $2l\pi$}
	  \begin{align*}
	    \hat{N_0}(u) &= K(m,2) {\left(2 \sin \frac{u}{2} \right)}^{2m}   \begin{vmatrix}
		\frac{1}{u^{2m}} & \sum_{k} \frac{1}{{(u+2k\pi)}^{2m-1}} \\
		\frac{1}{u^{2m-1}} & \sum_{k} \frac{1}{{(u+2k\pi)}^{2m-2}} \\
	      \end{vmatrix} \\
	      &=  K(m,2) {\left(2 \sin \frac{u}{2} \right)}^{2m} \frac{1}{(u-2l\pi)^{2m}} \bigg[ 
		\frac{{(u-2l\pi)}^2}{u^{2m}} + \frac{{(u-2l\pi)}^{2m}}{u^{2m}} \beta_{2m-2}(u-2l\pi) \\
	      & \quad - \Big( \frac{{(u-2l\pi)}^1}{u^{2m-1}} + \frac{{(u-2l\pi)}^{2m}}{u^{2m-1}} 
	    \beta_{2m-1}(u-2l\pi)\Big) \bigg] \\
	    \end{align*}
	    i.e $\displaystyle \lim_{u \to 2l\pi} \hat{N}_0(u) = 0$ while
	    \begin{align*}
	      \hat{N_1}(u) &= (-j)K(m,2) {\left(2 \sin \frac{u}{2} \right)}^{2m}   
	      \begin{vmatrix}
		\sum_{k} \frac{1}{{(u+2k\pi)}^{2m}} & \frac{1}{u^{2m}}  \\
		\sum_{k} \frac{1}{{(u+2k\pi)}^{2m-1}} & \frac{1}{u^{2m-1}}  \\
	      \end{vmatrix} \\
	      &=  K(m,2) {\left(2 \sin \frac{u}{2} \right)}^{2m} \frac{1}{(u-2l\pi)^{2m}} \bigg[ \frac{1}{u^{2m-1}} + 
		\frac{{(u-2l\pi)}^{2m}}{u^{2m-1}} \beta_{2m}(u-2l\pi) \\
	      & \quad - \Big( \frac{{(u-2l\pi)}^1}{u^{2m}} + \frac{{(u-2l\pi)}^{2m}}{u^{2m}} \beta_{2m-1}(u-2l\pi)\Big) 
	    \bigg] \\
	    \end{align*}
	    i.e $\displaystyle \lim_{u \to 2l\pi} \hat{N}_1(u) = -\frac{jK(m,2)}{{(2l\pi)}^{2m-1}}$
      \end{itemize}

    \item Assume $m=3, r=3$.
      \begin{itemize}
	\item \underline{Around 0}
	  \begin{align*}
	    \hat{N_0}(u) &= \frac{5!4!3!}{2!} {\left(2 \sin \frac{u}{2} \right)}^{6}   
	    \begin{vmatrix}
	      \frac{1}{u^6} & \sum_{k} \frac{1}{{(u+2k\pi)}^{5}}& \sum_{k} \frac{1}{{(u+2k\pi)}^{4}} \\
	      \frac{1}{u^5} & \sum_{k} \frac{1}{{(u+2k\pi)}^{4}}& \sum_{k} \frac{1}{{(u+2k\pi)}^{3}} \\
	      \frac{1}{u^4} & \sum_{k} \frac{1}{{(u+2k\pi)}^{3}}& \sum_{k} \frac{1}{{(u+2k\pi)}^{2}} \\
	    \end{vmatrix} \\
	    &=  \frac{5!4!3!}{2!} {\left(2 \sin \frac{u}{2} \right)}^{6}  \frac{1}{u^{12}} \Big[ 1+ \sum_{k \neq 0} 
	      \frac{u^4}{{(u+2k\pi)}^{4}} + \frac{u^2}{{(u+2k\pi)}^{2}} + \sum_{k \neq 0}\frac{u^4}{{(u+2k\pi)}^{4}} \sum_{k 
	      \neq 0}\frac{u^2}{{(u+2k\pi)}^{2}} \\
	    & \quad - (1+ \sum_{k \neq 0} \frac{u^3}{{(u+2k\pi)}^{3}} + \frac{u^3}{{(u+2k\pi)}^{3}} + \sum_{k \neq 0} 
	    \frac{u^3}{{(u+2k\pi)}^{3}}\sum_{k \neq 0} \frac{u^3}{{(u+2k\pi)}^{3}})  \\
	    & \quad - (1+ \sum_{k \neq 0} \frac{u^5}{{(u+2k\pi)}^{5}} + \frac{u^2}{{(u+2k\pi)}^{2}} + \sum_{k \neq 
	    0}\frac{u^5}{{(u+2k\pi)}^{5}} \sum_{k \neq 0}\frac{u^2}{{(u+2k\pi)}^{2}}) \\
	    & \quad + (1+ \sum_{k \neq 0} \frac{u^4}{{(u+2k\pi)}^{4}} + \frac{u^3}{{(u+2k\pi)}^{3}} + \sum_{k \neq 0} 
	  \frac{u^3}{{(u+2k\pi)}^{3}}\sum_{k \neq 0} \frac{u^4}{{(u+2k\pi)}^{4}}) \\
	   &\quad + 1+ \sum_{k \neq 0} 
	      \frac{u^5}{{(u+2k\pi)}^{5}} + \frac{u^3}{{(u+2k\pi)}^{3}} + \sum_{k \neq 0}\frac{u^5}{{(u+2k\pi)}^{5}} \sum_{k 
	      \neq 0}\frac{u^3}{{(u+2k\pi)}^{3}} \\
	    & \quad - (1+ \sum_{k \neq 0} \frac{u^4}{{(u+2k\pi)}^{4}} + \frac{u^4}{{(u+2k\pi)}^{4}} + \sum_{k \neq 0} 
	    \frac{u^4}{{(u+2k\pi)}^{4}}\sum_{k \neq 0} \frac{u^4}{{(u+2k\pi)}^{4}}) \Big] \\
	    &= \frac{5!4!3!}{2!} {\left(2 \sin \frac{u}{2} \right)}^{6}  \frac{1}{u^6} \Big[\beta_4(u)\beta_2(u) - 
	    \beta_3(u)\beta_3(u) + \mathcal{O}(u) \Big]
	  \end{align*}

	\item \underline{Around $2l\pi$, $v=u-2l\pi$}
	  \begin{align*}
	    \hat{N_0}(u) &= K(m,3) {\left(2 \sin \frac{v}{2} \right)}^{2m}   \begin{vmatrix}
	      \frac{1}{u^{2m}} & \frac{1}{v^{2m-1}}(1+v^{2m-1}\beta_{2m-1}(v)) & 
	      \frac{1}{v^{2m-2}}(1+v^{2m-2}\beta_{2m-2}(v)) \\
	      \frac{1}{u^{2m-1}} & \frac{1}{v^{2m-2}}(1+v^{2m-2}\beta_{2m-2}(v)) & 
	      \frac{1}{v^{2m-3}}(1+v^{2m-3}\beta_{2m-3}(v)) \\
	      \frac{1}{u^{2m-2}} & \frac{1}{v^{2m-3}}(1+v^{2m-3}\beta_{2m-3}(v)) & 
	      \frac{1}{v^{2m-4}}(1+v^{2m-4}\beta_{2m-4}(v)) \\
	    \end{vmatrix} \\
	    &=  K(m,3) {\left(2 \sin \frac{u}{2} \right)}^{2m}  \frac{1}{v^{r(2m-r+1)}} \Big[ 1+ \sum_{k \neq 0} 
	      \frac{u^4}{{(u+2k\pi)}^{4}} + \frac{u^2}{{(u+2k\pi)}^{2}} + \sum_{k \neq 0}\frac{u^4}{{(u+2k\pi)}^{4}} 
	      \sum_{k \neq 0}\frac{u^2}{{(u+2k\pi)}^{2}} \\
	    & \quad - (1+ \sum_{k \neq 0} \frac{u^3}{{(u+2k\pi)}^{3}} + \frac{u^3}{{(u+2k\pi)}^{3}} + \sum_{k \neq 0} 
	    \frac{u^3}{{(u+2k\pi)}^{3}}\sum_{k \neq 0} \frac{u^3}{{(u+2k\pi)}^{3}})  \\
	    & \quad - (1+ \sum_{k \neq 0} \frac{u^5}{{(u+2k\pi)}^{5}} + \frac{u^2}{{(u+2k\pi)}^{2}} + \sum_{k \neq 
	    0}\frac{u^5}{{(u+2k\pi)}^{5}} \sum_{k \neq 0}\frac{u^2}{{(u+2k\pi)}^{2}}) \\
	    & \quad + (1+ \sum_{k \neq 0} \frac{u^4}{{(u+2k\pi)}^{4}} + \frac{u^3}{{(u+2k\pi)}^{3}} + \sum_{k \neq 0} 
	  \frac{u^3}{{(u+2k\pi)}^{3}}\sum_{k \neq 0} \frac{u^4}{{(u+2k\pi)}^{4}}) \\
	   &\quad + 1+ \sum_{k \neq 0} 
	      \frac{u^5}{{(u+2k\pi)}^{5}} + \frac{u^3}{{(u+2k\pi)}^{3}} + \sum_{k \neq 0}\frac{u^5}{{(u+2k\pi)}^{5}} \sum_{k 
	      \neq 0}\frac{u^3}{{(u+2k\pi)}^{3}} \\
	    & \quad - (1+ \sum_{k \neq 0} \frac{u^4}{{(u+2k\pi)}^{4}} + \frac{u^4}{{(u+2k\pi)}^{4}} + \sum_{k \neq 0} 
	    \frac{u^4}{{(u+2k\pi)}^{4}}\sum_{k \neq 0} \frac{u^4}{{(u+2k\pi)}^{4}}) \Big] \\
	    &= \frac{5!4!3!}{2!} {\left(2 \sin \frac{u}{2} \right)}^{6}  \frac{1}{u^6} \Big[\beta_4(u)\beta_2(u) - 
	    \beta_3(u)\beta_3(u) o(u) \Big]
	  \end{align*}
      \end{itemize}
  \end{enumerate}
\end{example}

\begin{proof}
Using the definition of Hermite B-splines
\begin{equation*}
  N_s(x) = \sum_{k=-(m-r)}^{m-r} c_k L_{2m,r,s}(x-k)
\end{equation*}
and given the integral representation of $I_{2m,r,s} = L_{2m,r,s}$, the following holds
\begin{equation}
  N_s(x) = \frac{{(-j)}^s}{2\pi} \int_{-\infty}^{\infty} e^{-ju(m-r)} \Pi_{2m-1,r}(e^{ju}) e^{jux} \frac{H_{r, 
  s}(\alpha_{2m}(u))}{H_r(\alpha_{2m}(u))} du
\end{equation}

It was proven by Lee and Sharma in (\cite[Theorem 4]{LeeSh76}) that the following holds
  \begin{equation}\label{eq:LeeSh}
    H_r\left(\frac{\Pi_{n}(\lambda)}{n!}\right) = {(-1)}^{\lfloor \frac{r}{2}\rfloor} C(n,r) 
    {(1-\lambda)}^{(r-1)(n-r+1)} \Pi_{n,r}(\lambda)
\end{equation}
with $C(n,r)$ the quantity
\begin{equation*}
  C(n,r) = \frac{1!2!\ldots(r-1)!}{n!(n-1)!\ldots(n-r+1)!}
\end{equation*}
Therefore,
\begin{equation*}
  \Pi_{2m-1,r}(e^{ju}) = {(-1)}^{\lfloor \frac{r}{2} \rfloor + m(r+1)} K(m,r) {(1-e^{ju})}^{-(r-1)(2m-r)} H_r\left( 
  \frac{\Pi_{2m-1}(e^{ju})}{(2m-1)!} \right)
\end{equation*}

We have $\Pi_{n}(\lambda) = A_n(0;\lambda){(1-\lambda)}^n$.  Using this and (\ref{prop:hankel}), we have
\begin{equation*}
  H_r\left(\frac{\Pi_{2m-1}(e^{ju})}{(2m-1)!}\right) = H_r\left(\frac{A_{2m-1}(0;e^{ju})}{(2m-1)!}\right) 
  {(1-e^{ju})}^{r(2m-r)}
\end{equation*}
while we previously established
\begin{equation*}
  H_r\left(\frac{A_{2m-1}(0;e^{ju})}{(2m-1)!}\right) = {(e^{ju}-1)}^{r}e^{-jur} {(-j)}^{r(2m-r)} 
  H_r(\alpha_{2m} (u))
\end{equation*}
Combining the previous relations we have
\begin{equation}
  \Pi_{2m-1,r}(e^{ju}) = {(-1)}^{\lfloor \frac{r}{2} \rfloor + m(r+1) + r} {(-j)}^{r(2m-r)} 
  \frac{{\left(e^{ju}-1\right)}^{2m}}{e^{jur}} K(m,r) H_r(\alpha_{2m}(u))
\end{equation}

Eventually,
\begin{equation}
  N_s(x) = \frac{{(-j)}^{s}}{2\pi} \int_{-\infty}^{\infty} e^{jux} {\left(2\sin \frac{u}{2}\right)}^{2m} K(m,r)   
  H_{r,s}(\alpha_{2m}(u)) du
\end{equation}

\end{proof}

\end{document}


