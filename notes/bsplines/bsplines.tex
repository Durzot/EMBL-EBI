\documentclass[a4paper, 11pt]{article}

\usepackage[left=1.5cm, right=1.5cm, top=2cm, bottom=2cm]{geometry}

\usepackage[utf8]{inputenc}
\usepackage[T1]{fontenc}
\usepackage[english]{babel}  
\usepackage{lmodern}

\usepackage{amsmath, amsthm, amssymb}
\usepackage{bm}
\usepackage{mathtools}
\usepackage{dsfont}

\newtheorem{innercustomgeneric}{\customgenericname}
\providecommand{\customgenericname}{}
\newcommand{\newcustomtheorem}[2]{%
  \newenvironment{#1}[1]
  {%
   \renewcommand\customgenericname{#2}%
   \renewcommand\theinnercustomgeneric{##1}%
   \innercustomgeneric}
  {\endinnercustomgeneric}
}

\newtheorem{thm}{Theorem}
\newtheorem{lem}{Lemma}
\newtheorem{cor}{Corollary}
\newtheorem{deftn}{Definition}
\newtheorem{prop}{Proposition}
\newtheorem{remark}{Remark}
\newtheorem{example}{Example}

\DeclareMathOperator*{\argmin}{argmin} 
\DeclareMathOperator*{\argmax}{argmax}
\DeclareMathOperator*{\essinf}{ess\ inf}
\DeclareMathOperator*{\esssup}{ess\ sup} 

\renewcommand{\thesection}{\Roman{section}}

\usepackage[url=true,isbn=false, maxbibnames=99,backend=bibtex]{biblatex}
\usepackage{csquotes}
\usepackage{filecontents}

\begin{filecontents}{xmpl.bib}
@book{Bo01,
  shorthand = {Bo01},
  author = {Carl de Boor},
  title = {A practical guide to splines},
  year = {2001},
  publisher = {Springer},
  edition = {Revised edition}
}

@article{CS66,
  shorthand = {CS66},
  author = {Curry, H.B and Schoenberg, I.J},
  title = {On Pólya frequency functions IV: the fundamental spline functions and their limits},
  year = {1966},
  journal = {J. Analyse Math.},
  volume = {17},
  pages = {71-107},
}
\end{filecontents}

\addbibresource{xmpl.bib}

\begin{document}
\title{B-splines}
\author{Yoann Pradat}
\maketitle

\section{B-splines}
\subsection{Motivation}

In the problem of interpolating data, may it be values or derivatives of some unknown function, the intuitive method 
consisting in interpolating all the data at once is prone to large errors when the data exceeds a few points. This is 
the consequence of resorting to high order polynomials which tend to oscillate a lot when forced to perfectly 
interpolate the data. A \emph{natural} way of approaching problems deemed too hard consist in splitting them into 
several subproblems of lesser complexity then solving each of these before putting solutions of all subproblems 
together.  In our case, the less difficult problems are that of interpolating a subset of the data and then joining 
interpolating pieces together so that the global solution interpolates all the data. The price to pay for doing so is 
the decrease in the smoothness of the interpolating function at the break points we will have chosen. As we will see 
later, the maximum degree of smoothness achievable is a decreasing function of the number of derivatives we try to 
interpolate. In polynomial interpolation, the subproblems are that of interpolating on subintervals with small degree 
polynomials so that the general solution belongs to the space of piecewise polynomial functions of given order defined 
as follows

\begin{deftn}
  Let $k$ an integer and $\bm{\xi}$ a strictly increasing sequence of real numbers. The set of all piecewise polynomials 
  of order $k$ with breaks at $\bm {\xi}$ is denoted $\Pi_{<k, \bm{\xi}}$. It consists in all functions that are 
  polynomials of order $k$ on all intervals $(\xi_i, \xi_{i+1})$. The elements of $\bm{\xi}$ are called \emph{knots}.
\end{deftn}

For the needs of further results, let's introduce here the subspaces where a certain number of derivatives are made be 
continuous at the knots. 

\begin{deftn}
  Let $k$ an integer, $\bm{\xi}$ a strictly increasing sequence of real numbers and $\bm{\nu}$ a sequence of nonnegative 
  integers. Define
  \begin{equation}
    \Pi_{<k, \bm{\xi}, \bm{\nu}} = \{f \in \Pi_{<k, \bm{\xi}} \ | \ \text{jump}_{\xi_i} D^{j-1}f = 0, j=1, \ldots, 
    \nu_i\}
  \end{equation}
\end{deftn}

The maximum degree of continuity achievable at a knot is the order of the polynomials on each side of the knots, $k$ in 
our notation. Indeed, in the case $\nu_i = k$, writing the polynomials in their Taylor expansion at $\xi_i$ up to order 
$k$ we see that both polynomials share the same coefficients in the expansion. Consequently polynomials on each side of 
the knot join \emph{perfectly} in the sense that they are subparts of the same polynomial of order $k$. The value 
$\nu_i=0$ means we impose no continuity condition at the knot $\xi_i$.  If $\bm{\nu_1}, \bm{\nu_2}$ are two sequences 
such that $\bm{\nu_1} \leq \bm{\nu_2}$ it follows from the definition that $\Pi_{<k, \bm{\xi}, \bm{\nu_2}} \subset 
\Pi_{<k, \bm{\xi}, \bm{\nu_1}}$.

\subsection{Definitions}

Let $\bm{t}$ a sequence of nondecreasing real numbers. This sequence can be finite on both sides or only one side or it 
can be biinfinite. In his book (\cite{Bo01}), De Boor defines \emph{normalized} B-splines using the divided difference 
operator as follows

\begin{deftn}
  The $j^{th}$ \emph{normalized} B-spline of order $k$ is
  \begin{equation}
    B_{j, k, \bm{t}}(t) = (t_{j+k}-t_j)[t_j, \ldots, t_{j+k}]{(\cdot-t)}_+^{k-1}
  \end{equation}
\end{deftn}

In order to ligthen notations we will usually drop the dependence in the knots sequence $\bm{t}$ when it is clear from 
the context what these knots are. In the definition above we adopt the convention that $0^0 = 0$ which has the 
consequence of making our B-splines right-continuous.

\begin{example}
  \begin{itemize}
    \item \underline{$k=1$}
      \begin{align*}
        B_{j,1}(t) &= (t_{j+1}-t_j)[t_j, t_{j+1}]{(\cdot-t)}_+^0 \\
        &= {(t_{j+1}-t)}_+^0 - {(t_j-t)}_+^0 \\
        &= \begin{dcases} 1 & t_j \leq t < t_{j+1} \\ 0 & \text{elsewhere} \end{dcases}
      \end{align*}
    \item \underline{$k=2$}
      \begin{align*}
        B_{j,2}(t) &= (t_{j+2}-t_j)\frac{[t_{j+1}, t_{j+2}]{(\cdot-t)}_+^1-[t_{j}, 
        t_{j+1}]{(\cdot-t)}_+^1}{t_{j+2}-t_j}\\
        &= \frac{{(t_{j+2}-t)}_+^1-{(t_{j+1}-t)}_+^1}{t_{j+2}-t_{j+1}}- 
        \frac{{(t_{j+1}-t)}_+^1-{(t_{j}-t)}_+^1}{t_{j+1}-t_{j}} \\
        &= 
        \begin{dcases}
          \frac{t-t_j}{t_{j+1}-t_j} & t_j \leq t < t_{j+1} \\
          \frac{t_{j+2}-t}{t_{j+2}-t_{j+1}} & t_{j+1} \leq t < t_{j+2} \\
          0 & \text{elsewhere} 
        \end{dcases}
    \end{align*}
\end{itemize}
\end{example}

For the needs of incoming properties we need to define the \textbf{basic interval} $I_{k, \bm{t}}$  as follows

\begin{equation}
  I_{k, \bm{t}} = (t_-, t_+), \quad t_- := \begin{dcases} t_k & \text{if} \ \bm{t} = (t_1, \ldots) \\ \inf t_j & 
\text{otherwise} \end{dcases}, \quad t_+ := \begin{dcases} t_{n+1} & \text{if} \ \bm{t} = (\ldots, t_{n+k}) \\ \sup t_j 
& \text{otherwise} \end{dcases}, \end{equation}

These B-splines are normalized in the sense that they satisfy $\sum_{j} B_{j,k, \bm{t}} = 0$ on $I_{k, \bm{t}}$. The 
definition De Boor uses is different from that previously used by Curry and Schoenberg in (\cite{CS66}) which is the 
following

\begin{deftn}
  The $j^{th}$ B-spline of Curry and Schoenberg is
  \begin{equation}
    M_{j,k,\bm{t}}(t) = k [t_j, \ldots, t_{j+k}]{(\cdot-t)}_+^{k-1}
  \end{equation}
\end{deftn}

These B-splines are closely related to the previously defined B-splines as $B_{j, k, \bm{t}} = \frac{t_{j+k}-t_j}{k} 
M_{j,k,\bm{t}}$. In the following paragraphs we will focus on the properties of $B_{j,k,\bm{t}}$ as they can be written 
more elegantly than with $M_{j,k,\bm{t}}$. However they can easily be translated in terms $M_{j,k,\bm{t}}$ given the 
relation above. For the sake of completeness let at least mention that $M_{j,k,\bm{t}}$ have the property of being
kernels for the divided difference operator (\cite{CS66}, eq\.  (1.5)) in the sense that for any function $f \in 
\mathcal{C}^{(k)}$  we have 

\begin{equation*}
  [t_j, \ldots, t_{j+k}]f = \frac{1}{k!} \int_{t_j}^{t_{j+k}} M_{j,k,\bm{t}} f^{(k)}(t) dt
\end{equation*}


\subsection{Properties}

\nocite{*}
\printbibliography%
%\bibliographystyle{unsrt}
%\bibliography{xmpl}
%\addcontentsline{toc}{section}{References}

\end{document}
